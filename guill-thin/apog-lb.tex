\section{Lower Bound on APoG}
\label{sec:apog-lb}

\begin{lemma}
\label{thm:guill-hardex-area}
Let $m$ and $k$ be positive integers and $\eps$ be a positive real number.
Let $J$ be a set of items packed into a bin, where each item has the longer dimension
equal to $(1+\eps)/2$ and the shorter dimension equal to $(1-\eps)/2k$.
If the bin is guillotine-separable, then $a(J) \le 3/4 + \eps/2 - \eps^2/4$.
\end{lemma}
\begin{proof}
For an item packed in the bin, if the height is $(1-\eps)/2k$, call it a wide item,
and if the width is $(1-\eps)/2k$, call it a tall item.
Let $W$ be the set of wide items in $J$.

The packing of items in the bin can be represented as a tree,
called the \emph{guillotine tree} of the bin,
where each node $u$ represents a rectangular region of the bin
and the child nodes $v_1, v_2, \ldots, v_p$ of node $u$
represent the sub-regions obtained by parallel guillotine cuts.
The ordering of the children has a significance here:
if the guillotine cuts were vertical, children are ordered by increasing $x$-coordinate,
and if the cuts were horizontal, children are ordered by increasing $y$-coordinate.

We will now see how to rearrange the items in the bin so that the packing remains
guillotine-separable but becomes more structured.
We will exploit this structure to show that the packing has a large unpacked area.
See \cref{fig:shift-thin} for an example.

\begin{figure}[htb]
\centering
\ifcsname pGameL\endcsname\else\newlength{\pGameL}\fi
\setlength{\pGameL}{0.3cm}
\tikzset{bin/.style={draw,thick}}
%\tikzset{binGrid/.style={draw,step=1\pGameL,{textColor!20!bgColor}}}
\tikzset{item/.style={draw,fill={textColor!15!bgColor}}}
\tikzset{myarrow/.style={->,>={Stealth}}}
\tikzset{gsep/.style={draw,dashed,semithick,fill={textColor!85!bgColor}}}
\begin{tikzpicture}
\begin{scope}
%\path[binGrid] (0\pGameL, 0\pGameL) grid (16\pGameL, 16\pGameL);
\path[item] (0\pGameL, 15\pGameL) rectangle +(9\pGameL, 1\pGameL);
\path[item] (1\pGameL, 14\pGameL) rectangle +(9\pGameL, 1\pGameL);
\path[item] (5\pGameL, 4\pGameL) rectangle +(9\pGameL, 1\pGameL);
\path[item] (5\pGameL, 3\pGameL) rectangle +(9\pGameL, 1\pGameL);
\path[item] (1\pGameL, 2\pGameL) rectangle +(9\pGameL, 1\pGameL);
\path[item] (0\pGameL, 1\pGameL) rectangle +(9\pGameL, 1\pGameL);
\path[item] (7\pGameL, 0\pGameL) rectangle +(9\pGameL, 1\pGameL);
\path[item] (0\pGameL, 6\pGameL) rectangle +(1\pGameL, 9\pGameL);
\path[item] (1\pGameL, 5\pGameL) rectangle +(1\pGameL, 9\pGameL);
\path[item] (2\pGameL, 5\pGameL) rectangle +(1\pGameL, 9\pGameL);
\path[item] (3\pGameL, 5\pGameL) rectangle +(1\pGameL, 9\pGameL);
\path[item] (4\pGameL, 5\pGameL) rectangle +(1\pGameL, 9\pGameL);
\path[item] (5\pGameL, 5\pGameL) rectangle +(1\pGameL, 9\pGameL);
\path[item] (7\pGameL, 5\pGameL) rectangle +(1\pGameL, 9\pGameL);
\path[item] (8\pGameL, 5\pGameL) rectangle +(1\pGameL, 9\pGameL);
\path[item] (9\pGameL, 5\pGameL) rectangle +(1\pGameL, 9\pGameL);
\path[item] (10\pGameL, 5\pGameL) rectangle +(1\pGameL, 9\pGameL);
\path[item] (12\pGameL, 5\pGameL) rectangle +(1\pGameL, 9\pGameL);
\path[item] (13\pGameL, 5\pGameL) rectangle +(1\pGameL, 9\pGameL);
\path[item] (14\pGameL, 4\pGameL) rectangle +(1\pGameL, 9\pGameL);
\path[item] (15\pGameL, 2\pGameL) rectangle +(1\pGameL, 9\pGameL);
\path[bin] (0\pGameL, 0\pGameL) rectangle (16\pGameL, 16\pGameL);
\path[gsep] (0\pGameL, 2\pGameL) -- +(16\pGameL, 0\pGameL);
\path[gsep] (0\pGameL, 15\pGameL) -- +(16\pGameL, 0\pGameL);
\path[gsep] (1\pGameL, 2\pGameL) -- (1\pGameL, 15\pGameL);
\path[gsep] (14\pGameL, 2\pGameL) -- (14\pGameL, 15\pGameL);
\path[gsep] (1\pGameL, 14\pGameL) -- (14\pGameL, 14\pGameL);
\path[gsep] (1\pGameL, 5\pGameL) -- (14\pGameL, 5\pGameL);
\end{scope}
\draw[myarrow,semithick] (17\pGameL, 8\pGameL) -- +(6\pGameL, 0);
\begin{scope}[xshift=24\pGameL]
%\path[binGrid] (0\pGameL, 0\pGameL) grid (16\pGameL, 16\pGameL);
\path[item] (0\pGameL, 15\pGameL) rectangle +(9\pGameL, 1\pGameL);
\path[item] (0\pGameL, 14\pGameL) rectangle +(9\pGameL, 1\pGameL);
\path[item] (0\pGameL, 13\pGameL) rectangle +(9\pGameL, 1\pGameL);
\path[item] (0\pGameL, 12\pGameL) rectangle +(9\pGameL, 1\pGameL);
\path[item] (0\pGameL, 11\pGameL) rectangle +(9\pGameL, 1\pGameL);
\path[item] (0\pGameL, 10\pGameL) rectangle +(9\pGameL, 1\pGameL);
\path[item] (0\pGameL, 9\pGameL) rectangle +(9\pGameL, 1\pGameL);
\path[item] (0\pGameL, 0\pGameL) rectangle +(1\pGameL, 9\pGameL);
\path[item] (1\pGameL, 0\pGameL) rectangle +(1\pGameL, 9\pGameL);
\path[item] (2\pGameL, 0\pGameL) rectangle +(1\pGameL, 9\pGameL);
\path[item] (3\pGameL, 0\pGameL) rectangle +(1\pGameL, 9\pGameL);
\path[item] (4\pGameL, 0\pGameL) rectangle +(1\pGameL, 9\pGameL);
\path[item] (6\pGameL, 0\pGameL) rectangle +(1\pGameL, 9\pGameL);
\path[item] (7\pGameL, 0\pGameL) rectangle +(1\pGameL, 9\pGameL);
\path[item] (8\pGameL, 0\pGameL) rectangle +(1\pGameL, 9\pGameL);
\path[item] (9\pGameL, 0\pGameL) rectangle +(1\pGameL, 9\pGameL);
\path[item] (11\pGameL, 0\pGameL) rectangle +(1\pGameL, 9\pGameL);
\path[item] (12\pGameL, 0\pGameL) rectangle +(1\pGameL, 9\pGameL);
\path[item] (13\pGameL, 0\pGameL) rectangle +(1\pGameL, 9\pGameL);
\path[item] (14\pGameL, 0\pGameL) rectangle +(1\pGameL, 9\pGameL);
\path[item] (15\pGameL, 0\pGameL) rectangle +(1\pGameL, 9\pGameL);
\path[bin] (0\pGameL, 0\pGameL) rectangle (16\pGameL, 16\pGameL);
\path[gsep] (0\pGameL, 15\pGameL) -- (16\pGameL, 15\pGameL);
\path[gsep] (0\pGameL, 13\pGameL) -- (16\pGameL, 13\pGameL);
\path[gsep] (13\pGameL, 0\pGameL) -- (13\pGameL, 13\pGameL);
\path[gsep] (14\pGameL, 0\pGameL) -- (14\pGameL, 13\pGameL);
\path[gsep] (0\pGameL, 12\pGameL) -- (13\pGameL, 12\pGameL);
\path[gsep] (0\pGameL, 9\pGameL) -- (13\pGameL, 9\pGameL);
\end{scope}
\end{tikzpicture}

\caption{Structuring a guillotine-separable packing.}
\label{fig:shift-thin}
\end{figure}

In the guillotine tree, suppose there is a node $u$
that has children $v_1, v_2, \ldots, v_p$.
\WLoG, assume that the children are obtained by making vertical cuts.
At most one of these children can contain items from $W$.
We can assume \wLoG{} that the other children contain only one item,
because otherwise we can separate them by vertical cuts.
We can reorder the children (which is equivalent to repacking the guillotine partitions)
so that the child containing items from $W$ (if any) is the first child.
Therefore, we can assume \wLoG{} that at any level in the guillotine tree,
only the first node has children.

Based on the argument above, we can see that the first node in each level
touches the bottom-left corner of the bin. All the other nodes either
contain a single wide item and touch the left edge of the bin but not the bottom edge,
or they contain a single tall item and touch the bottom edge of the bin but not the left edge.
In each node containing a wide item, shift the item leftwards,
and in each node containing a tall item, shift the item downwards.
Then each wide item touches the left edge of the bin
and each tall item touches the bottom edge of the bin.

Therefore, the square region of side length $(1-\eps)/2$
at the top-right corner of the bin is empty.
Hence, the area occupied in each bin is at most $3/4 + \eps/2 - \eps^2/4$.
\end{proof}

\rdefHardItems*

\rthmAPoGLB*
\begin{proof}
For an item $i \in I$, if $h(i) = (1-\eps)/2k$, call it a wide item,
and if $w(i) = (1-\eps)/2k$, call it a tall item.
Let $W$ be the set of wide items and $H$ be the set of tall items.

We will show that $\opt(I)$ and $\opt_g(I)$ have a big difference,
which will give us a lower-bound on $\APoG$.

Partition $W$ into groups of $k$ elements.
In each group, stack items one-over-the-other.
This gives us $2m$ containers of width $(1+\eps)/2$ and height $(1-\eps)/2$.
Similarly, get $2m$ containers of height $(1+\eps)/2$ and height $(1-\eps)/2$
by stacking items from $H$ side-by-side.
We can pack 4 containers in one bin, so $I$ can be packed into $m$ bins.
See \cref{fig:thin-gadget} for an example.
Therefore, $\opt(I) \le m$.

\begin{figure}[htb]
\centering
\ifcsname pGameL\endcsname\else\newlength{\pGameL}\fi
\setlength{\pGameL}{0.3cm}
\tikzset{bin/.style={draw,thick}}
%\tikzset{binGrid/.style={draw,step=1\pGameL,{black!20}}}
\tikzset{item/.style={draw,fill={black!15}}}
\begin{tikzpicture}
%\path[binGrid] (0\pGameL, 0\pGameL) grid (16\pGameL, 16\pGameL);
\path[item] (7\pGameL, 15\pGameL) rectangle +(9\pGameL, 1\pGameL);
\path[item] (0\pGameL, 0\pGameL) rectangle +(9\pGameL, 1\pGameL);
\path[item] (7\pGameL, 14\pGameL) rectangle +(9\pGameL, 1\pGameL);
\path[item] (7\pGameL, 13\pGameL) rectangle +(9\pGameL, 1\pGameL);
\path[item] (7\pGameL, 12\pGameL) rectangle +(9\pGameL, 1\pGameL);
\path[item] (7\pGameL, 11\pGameL) rectangle +(9\pGameL, 1\pGameL);
\path[item] (7\pGameL, 10\pGameL) rectangle +(9\pGameL, 1\pGameL);
\path[item] (7\pGameL, 9\pGameL) rectangle +(9\pGameL, 1\pGameL);
\path[item] (0\pGameL, 1\pGameL) rectangle +(9\pGameL, 1\pGameL);
\path[item] (0\pGameL, 2\pGameL) rectangle +(9\pGameL, 1\pGameL);
\path[item] (0\pGameL, 3\pGameL) rectangle +(9\pGameL, 1\pGameL);
\path[item] (0\pGameL, 4\pGameL) rectangle +(9\pGameL, 1\pGameL);
\path[item] (0\pGameL, 5\pGameL) rectangle +(9\pGameL, 1\pGameL);
\path[item] (0\pGameL, 6\pGameL) rectangle +(9\pGameL, 1\pGameL);
\path[item] (0\pGameL, 7\pGameL) rectangle +(1\pGameL, 9\pGameL);
\path[item] (15\pGameL, 0\pGameL) rectangle +(1\pGameL, 9\pGameL);
\path[item] (1\pGameL, 7\pGameL) rectangle +(1\pGameL, 9\pGameL);
\path[item] (2\pGameL, 7\pGameL) rectangle +(1\pGameL, 9\pGameL);
\path[item] (3\pGameL, 7\pGameL) rectangle +(1\pGameL, 9\pGameL);
\path[item] (4\pGameL, 7\pGameL) rectangle +(1\pGameL, 9\pGameL);
\path[item] (5\pGameL, 7\pGameL) rectangle +(1\pGameL, 9\pGameL);
\path[item] (6\pGameL, 7\pGameL) rectangle +(1\pGameL, 9\pGameL);
\path[item] (14\pGameL, 0\pGameL) rectangle +(1\pGameL, 9\pGameL);
\path[item] (13\pGameL, 0\pGameL) rectangle +(1\pGameL, 9\pGameL);
\path[item] (12\pGameL, 0\pGameL) rectangle +(1\pGameL, 9\pGameL);
\path[item] (11\pGameL, 0\pGameL) rectangle +(1\pGameL, 9\pGameL);
\path[item] (10\pGameL, 0\pGameL) rectangle +(1\pGameL, 9\pGameL);
\path[item] (9\pGameL, 0\pGameL) rectangle +(1\pGameL, 9\pGameL);
\path[bin] (0\pGameL, 0\pGameL) rectangle (16\pGameL, 16\pGameL);
\end{tikzpicture}

\caption{Packing $4k$ items in one bin. Here $k = 7$.}
\label{fig:thin-gadget}
\end{figure}

We will now show a lower-bound on $\opt_g(I)$.
In any guillotine-separable packing of $I$,
the area occupied by each bin is at most $3/4 + \eps/2 - \eps^2/4$
(by \cref{thm:guill-hardex-area}).
Note that $a(I) = m(1 - \eps^2)$. Therefore,
\begin{align*}
& \opt_g(I) \ge \frac{m(1-\eps^2)}{3/4 + \eps/2 - \eps^2/4}
\\ &\implies \frac{\opt_g(I)}{\opt(I)}
    \ge \frac{4}{3} \times \frac{1-\eps^2}{1 + 2\eps/3 - \eps^2/3}
    = \frac{4}{3} \times \frac{1-\eps}{1 - \eps/3}
    \ge \frac{4}{3}(1-\eps)
\qedhere
\end{align*}
\end{proof}
