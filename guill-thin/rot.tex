\section{APoG for the Rotational Case}
\label{sec:guill-rot}

\begin{theorem}
\label{thm:guill-rot}
For a set $I$ of rectangular items,
let $\opt\nonrot(I)$ and $\opt\rot(I)$ be the minimum number of bins
needed to pack $I$ in the non-rotational and rotational versions, respectively.
Let $\opt\nonrot_g(I)$ and $\opt\rot_g(I)$ be the minimum number of guillotinable bins
needed to pack $I$ in the non-rotational and rotational versions, respectively.

Let $\Scal$ be a family of inputs that is closed under rotation, i.e.,
for a set $I \in \Scal$ of items, if we rotate some items in $I$ to get a set $J$ of items,
then $J \in \Scal$.
Let $\APoG\nonrot$ and $\APoG\rot$ be the APoG for the non-rotational
and rotational versions, respectively, restricted to the family $\Scal$.
Then $\APoG\rot \le \APoG\nonrot$.
\end{theorem}
\begin{proof}
Let $I$ be any set of items in $\Scal$. Let $K$ be the corresponding rotated items
in the optimal rotational packing of $I$, i.e., $\opt\rot(I) = \opt\nonrot(K)$. Then
\begin{align*}
\opt\rot_g(I) &\le \opt\nonrot_g(K)
\\ &\le \APoG\nonrot\opt\nonrot(K) + O(1)
\\ &= \APoG\nonrot\opt\rot(I) + O(1)
\end{align*}
Since this is true for all $I \in \Scal$, we get $\APoG\rot \le \APoG\nonrot$.
\end{proof}

Assume without loss of generality that bins have width and height at least 1.
The class of $(\delta, \delta)$-\thin{} rectangles is closed under rotation,
so by \cref{thm:guill-rot}, the APoG for the rotational case is upper-bounded by
\[ \frac{4}{3}\left(1+\frac{4\delta}{1-\delta}\right). \]
When $\delta$ is very small, this is close to $4/3$.
