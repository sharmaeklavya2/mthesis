\section{Algorithm \texorpdfstring{$\thinGPack$}{thin4Pack}}
\label{sec:thin-gpack}

\subsection{Packing With Slicing}

Before we look at the problem of computing a (guillotinable) packing of \thin{} rectangles,
let us first look at a closely-related variant of this problem,
called the \emph{2D sliceable bin packing problem}, denoted as 2D SBP.
In this problem, we are given two sets of rectangular items, $\Wtild$ and $\Htild$, where
items in $\Wtild$ have width more than $1/2$, and items in $\Htild$ have height more than $1/2$.
$\Wtild$ is called the set of wide items and $\Htild$ is called the set of tall items.
We are allowed to \emph{slice} items in $\Wtild$ using horizontal cuts
and slice items in $\Htild$ using vertical cuts, and our task is to pack
$\Wtild \cup \Htild$ into the minimum number of bins without rotating the items.
See \cref{fig:bp-vs-sbp} for an example that illustrates the difference
between 2D GBP and 2D SBP.

\begin{figure}[htb]
\begin{subfigure}{0.45\textwidth}
\centering
\begin{tikzpicture}[
witem/.style={draw,fill={black!20}},
hitem/.style={draw,fill={black!10}},
bin/.style={draw,thick},
myarrow/.style={->,>={Stealth},thick},
]
\begin{scope}
\node at (0.5, 2.0) {$W:$};
\node at (0.5, 0.6) {$H:$};
\path[hitem] (1.0, 0.0) rectangle +(1.2, 1.2);
\path[hitem] (2.4, 0.0) rectangle +(1.2, 1.2);
\path[witem] (1.0, 1.4) rectangle +(1.2, 1.2);
\path[witem] (2.4, 1.4) rectangle +(1.2, 1.2);
\end{scope}
\draw[myarrow] (2.1, -0.3) -- (2.1, -1.5);
\begin{scope}[yshift={-6cm}]
\path[hitem] (0.0, 0.0) rectangle +(1.2, 1.2);
\path[hitem] (2.2, 0.0) rectangle +(1.2, 1.2);
\path[witem] (0.0, 2.2) rectangle +(1.2, 1.2);
\path[witem] (2.2, 2.2) rectangle +(1.2, 1.2);
\path[bin] (0.0, 0.0) rectangle +(2, 2);
\path[bin] (2.2, 0.0) rectangle +(2, 2);
\path[bin] (0.0, 2.2) rectangle +(2, 2);
\path[bin] (2.2, 2.2) rectangle +(2, 2);
\end{scope}
\end{tikzpicture}

\caption{Packing items into 4 bins without slicing.}
\end{subfigure}
\hfil
\begin{subfigure}{0.45\textwidth}
\centering
\begin{tikzpicture}[
witem/.style={draw,fill={black!20}},
hitem/.style={draw,fill={black!10}},
bin/.style={draw,thick},
myarrow/.style={->,>={Stealth},thick},
cutline/.style={draw={black!50!red},dashed,semithick},
]
\begin{scope}
\node at (0.5, 2.0) {$\Wtild:$};
\node at (0.5, 0.6) {$\Htild:$};
\path[hitem] (1.0, 0.0) rectangle +(1.2, 1.2);
\path[hitem] (2.4, 0.0) rectangle +(1.2, 1.2);
\path[witem] (1.0, 1.4) rectangle +(1.2, 1.2);
\path[witem] (2.4, 1.4) rectangle +(1.2, 1.2);
\path[cutline] (2.3, 2.0) -- (3.7, 2.0);
\node[xscale=-1,transform shape] at (3.9, 1.98) {\large\ding{34}};
\path[cutline] (3.0, -0.1) -- (3.0, 1.3);
\node[rotate=90,transform shape] at (3.02, -0.3) {\large\ding{34}};
\end{scope}
\draw[myarrow] (2.1, -0.3) -- (2.1, -1.6);
\begin{scope}[yshift={-4cm}]
\path[witem] (0.0, 0.0) rectangle +(1.2, 1.2);
\path[hitem] (2.2, 0.0) rectangle +(1.2, 1.2);
\path[witem] (0.0, 1.2) rectangle +(1.2, 0.6);
\path[witem] (2.2, 1.2) rectangle +(1.2, 0.6);
\path[hitem] (1.2, 0.0) rectangle +(0.6, 1.2);
\path[hitem] (3.4, 0.0) rectangle +(0.6, 1.2);
\path[bin] (0.0, 0.0) rectangle +(2, 2);
\path[bin] (2.2, 0.0) rectangle +(2, 2);
\end{scope}
\end{tikzpicture}

\caption{Packing items into 2 bins by horizontally slicing an item in $\Wtild$
and vertically slicing an item in $\Htild$.}
\end{subfigure}
\caption[2D GBP vs.~2D SBP]%
{Example to illustrate the difference between 2D geometric bin packing
and 2D sliceable bin packing. There are 2 wide items ($\Wtild$) and 2 tall items ($\Htild$).
The items are squares of side length 0.6 and the bins are squares of side length 1.}
\label{fig:bp-vs-sbp}
\end{figure}

We first describe a fast and simple $4/3$-asymptotic-approximation algorithm
for 2D SBP, called $\greedyPack$, that outputs a 2-stage packing.
Later, we will show how to use $\greedyPack$ to design $\thinGPack$.

We assume that the bin is a square of side length 1. Since we can slice items,
we allow items in $\Wtild$ to have height more than 1
and items in $\Htild$ to have width more than 1.

For a set $X \subseteq \Wtild$ of items, define
\[ \hsum(X) \defeq \sum_{i \in X} h(i) \quad\textrm{and}\quad
\wmax(X) \defeq \begin{cases}\max_{i \in X} w(i) & \textrm{ if } X \neq \emptyset
\\ 0 & \textrm{ if } X = \emptyset\end{cases}. \]

For a set $X \subseteq \Htild$ of items, define
\[ \wsum(X) \defeq \sum_{i \in X} w(i) \quad\textrm{and}\quad
\hmax(X) \defeq \begin{cases}\max_{i \in X} h(i) & \textrm{ if } X \neq \emptyset
\\ 0 & \textrm{ if } X = \emptyset\end{cases}. \]

In the algorithm $\greedyPack(\Wtild, \Htild)$, we first sort the items $\Wtild$ in decreasing order
of width and sort the items $\Htild$ in decreasing order of height.
If $\hsum(\Wtild) \ge \wsum(\Htild)$, then we pack the largest possible prefix of $\Wtild$
into a bin such that the items touch the right edge of the bin.
Then we pack a prefix of $\Htild$ into the remaining space in the side of the bin.
We call this a type-1 bin. See \cref{fig:greedy-pack} for an example.
If $\hsum(\Wtild) < \wsum(\Htild)$, we proceed analogously in a coordinate-swapped way,
i.e., we first pack tall items in the bin and then pack wide items in the remaining space.
Call this bin a type-2 bin.
We pack the rest of the items into bins in the same way.
See \cref{algo:greedyPack} for a more precise description of $\greedyPack$.

\begin{figure}[!ht]
\begin{subfigure}{0.45\textwidth}
    \centering
    \tikzset{mytransform/.style={}}
    \tikzset{wItem/.style={draw,fill={textColor!20!bgColor}}}
    \tikzset{hItem/.style={draw,fill={textColor!10!bgColor}}}
    \ifcsname pGameL\endcsname\else\newlength{\pGameL}\fi
\setlength{\pGameL}{0.2cm}
\tikzset{bin/.style={draw,thick}}
%\tikzset{binGrid/.style={draw,step=1\pGameL,{black!20}}}
\begin{tikzpicture}[mytransform]
%\path[binGrid] (0\pGameL, 0\pGameL) grid (20\pGameL, 20\pGameL);
\path[wItem] (5\pGameL, 0\pGameL) rectangle +(15\pGameL, 2\pGameL);
\path[wItem] (6\pGameL, 2\pGameL) rectangle +(14\pGameL, 3\pGameL);
\path[wItem] (6\pGameL, 5\pGameL) rectangle +(14\pGameL, 2\pGameL);
\path[wItem] (7\pGameL, 7\pGameL) rectangle +(13\pGameL, 3\pGameL);
\path[wItem] (8\pGameL, 10\pGameL) rectangle +(12\pGameL, 4\pGameL);
\path[wItem] (8\pGameL, 14\pGameL) rectangle +(12\pGameL, 2\pGameL);
\path[wItem] (9\pGameL, 16\pGameL) rectangle +(11\pGameL, 2\pGameL);
\path[wItem] (10\pGameL, 18\pGameL) rectangle +(10\pGameL, 2\pGameL);
\path[hItem] (1\pGameL, 4\pGameL) rectangle +(2\pGameL, 16\pGameL);
\path[hItem] (0\pGameL, 2\pGameL) rectangle +(1\pGameL, 18\pGameL);
\path[hItem] (3\pGameL, 5\pGameL) rectangle +(2\pGameL, 15\pGameL);
\draw[semithick,dashed] (5\pGameL, 0\pGameL) -- +(0\pGameL, 20\pGameL);
\path[bin] (0\pGameL, 0\pGameL) rectangle (20\pGameL, 20\pGameL);
\end{tikzpicture}

    \caption{A type-1 bin. Wide items are packed on the right. Tall items are packed on the left.}%
\label{fig:greedy-pack:1}
\end{subfigure}
\hfil
\begin{subfigure}{0.45\textwidth}
    \centering
    \tikzset{mytransform/.style={xscale=-1,rotate=90}}
    \tikzset{wItem/.style={draw,fill={textColor!10!bgColor}}}
    \tikzset{hItem/.style={draw,fill={textColor!20!bgColor}}}
    \ifcsname pGameL\endcsname\else\newlength{\pGameL}\fi
\setlength{\pGameL}{0.2cm}
\tikzset{bin/.style={draw,thick}}
%\tikzset{binGrid/.style={draw,step=1\pGameL,{black!20}}}
\begin{tikzpicture}[mytransform]
%\path[binGrid] (0\pGameL, 0\pGameL) grid (20\pGameL, 20\pGameL);
\path[wItem] (5\pGameL, 0\pGameL) rectangle +(15\pGameL, 2\pGameL);
\path[wItem] (6\pGameL, 2\pGameL) rectangle +(14\pGameL, 3\pGameL);
\path[wItem] (6\pGameL, 5\pGameL) rectangle +(14\pGameL, 2\pGameL);
\path[wItem] (7\pGameL, 7\pGameL) rectangle +(13\pGameL, 3\pGameL);
\path[wItem] (8\pGameL, 10\pGameL) rectangle +(12\pGameL, 4\pGameL);
\path[wItem] (8\pGameL, 14\pGameL) rectangle +(12\pGameL, 2\pGameL);
\path[wItem] (9\pGameL, 16\pGameL) rectangle +(11\pGameL, 2\pGameL);
\path[wItem] (10\pGameL, 18\pGameL) rectangle +(10\pGameL, 2\pGameL);
\path[hItem] (1\pGameL, 4\pGameL) rectangle +(2\pGameL, 16\pGameL);
\path[hItem] (0\pGameL, 2\pGameL) rectangle +(1\pGameL, 18\pGameL);
\path[hItem] (3\pGameL, 5\pGameL) rectangle +(2\pGameL, 15\pGameL);
\draw[semithick,dashed] (5\pGameL, 0\pGameL) -- +(0\pGameL, 20\pGameL);
\path[bin] (0\pGameL, 0\pGameL) rectangle (20\pGameL, 20\pGameL);
\end{tikzpicture}

    \caption{A type-2 bin. Tall items are packed above. Wide items are packed below.}%
\label{fig:greedy-pack:2}
\end{subfigure}
\caption{Examples of type-1 and type-2 bins produced by $\greedyPack$.}
\label{fig:greedy-pack}
\end{figure}

\begin{definition}
\label{defn:hw-prefix}
Let $\Wtild$ be a sequence of wide items.
Define $\hprefix(\Wtild, \gamma)$ as the prefix of $\Wtild$ of total height $\gamma$
if $\hsum(\Wtild) > \gamma$ (slice items if necessary).
If $\hsum(\Wtild) \le \gamma$, define $\hprefix(\Wtild, \gamma)$ to be $\Wtild$.
Let $\Htild$ be a sequence of tall items.
Define $\wprefix(\Htild, \gamma)$ as the prefix of $\Htild$ of total width $\gamma$
if $\wsum(\Htild) > \gamma$ (slice items if necessary).
If $\wsum(\Htild) \le \gamma$, define $\wprefix(\Htild, \gamma)$ to be $\Htild$.
\end{definition}

\begin{algorithm}[!ht]
\caption{$\greedyPack(\Wtild, \Htild)$: Packs items $\Wtild \cup \Htild$ into bins.
The items $\Wtild$ have width more than $1/2$ and can be sliced using horizontal cuts.
The items $\Htild$ have width more than $1/2$ and can be sliced using vertical cuts.}
\label{algo:greedyPack}
\begin{algorithmic}[1]
\State Sort the items in $\Wtild$ in decreasing order of width.
\State Sort the items in $\Htild$ in decreasing order of height.
\While{$\Wtild \neq \emptyset$ or $\Htild \neq \emptyset$}
    \State Create an empty bin.
    \If{$\hsum(\Wtild) \ge \wsum(\Htild)$}
        \State Let $X \defeq \hprefix(\Wtild, 1)$. \Comment{see \cref{defn:hw-prefix}}
        \State Pack $X$ in a region of width $\wmax(X)$ on the right side of the bin.
        \State Remove $X$ from $\Wtild$.
        \State Let $Y \defeq \wprefix(\Htild, 1-\wmax(X))$. \Comment{see \cref{defn:hw-prefix}}
        \State Pack $Y$ in a region of width $1 - \wmax(X)$ on the left side of the bin.
        \State Remove $Y$ from $\Htild$.
        \State Label the bin as a type-1 bin.
    \Else
        \parState{Proceed analogous to the previous case, i.e., $X$ is a prefix of $\Htild$ of width
            at most 1 and $Y$ is a prefix of $\Wtild$ of total height at most $1 - \hmax(X)$.}
        \parState{Label the bin as a type-2 bin.}
    \EndIf
\EndWhile
\end{algorithmic}
\end{algorithm}

\begin{claim}
\label{thm:greedy-pack}
$\greedyPack(\Wtild, \Htild)$ always outputs a 2-stage packing of $\Wtild \cup \Htild$.
It runs in time $O(m + |\Wtild|\log|\Wtild| + |\Htild|\log|\Htild|)$,
where $m$ is the number of bins used.
Furthermore, it slices items in $\Wtild$ by making at most $m-1$ horizontal cuts
and slices items in $\Htild$ by making at most $m-1$ vertical cuts.
\end{claim}

If all bins are of type 1, then the number of bins used is $\smallceil{\hsum(\Wtild)}$.
If all bins are of type 2, then the number of bins used is $\smallceil{\wsum(\Htild)}$.
Since items in $\Wtild$ have width more than $1/2$, no two items can be placed side-by-side.
If $\Htild = \{\}$, then the optimal solution is to stack the items $\Wtild$ one-over-the-other.
Therefore, $\smallceil{\hsum(\Wtild)} \le \opt(\Wtild \cup \Htild)$.
Similarly, $\smallceil{\wsum(\Htild)} \le \opt(\Wtild \cup \Htild)$.
Hence, if all bins are of the same type, the number of bins used is at most $\opt(\Wtild \cup \Htild)$.

We will now focus on the more interesting case, i.e.,
some bins are of type 1 and some are of type 2.

\begin{definition}[Full bin]
In a type-1 bin, let $X$ be the items from $\Wtild$ and $Y$ be the items from $\Htild$.
The bin is said to be \emph{full} iff $\hsum(X) = 1$ and $\wsum(Y) = 1 - \wmax(X)$.
Define fullness for a type-2 bin analogously.
\end{definition}

We first show that full bins pack items of a large total area,
and then we show that if some bins are of type 1 and some bins are of type 2,
then there can be at most 2 non-full bins.
This will help us get an upper-bound on the number of bins used by $\greedyPack(\Wtild, \Htild)$
in terms of $a(\Wtild \cup \Htild)$.

\begin{lemma}
\label{thm:area-bound}
Let there be $m_1$ full bins of type 1.
Let $J_1$ be the items inside those bins.
Then $m_1 \le 4a(J_1)/3 + 1/3$.
\end{lemma}
\begin{proof}
In the $j\Th$ full bin of type 1, let $X_j$ be the items from $\Wtild$
and $Y_j$ be the items from $\Htild$. Let
\[ \ell_j \defeq \begin{cases}\wmax(X_j) & \textrm{ if } j \le m_1
\\ 1/2 & \textrm{ if } j = m_1 + 1 \end{cases}. \]
Since all items have their larger dimension more than $1/2$,
$\ell_j \ge 1/2$ and $\hmax(Y_j) > 1/2$.

$a(X_j) \ge \ell_{j+1}$, since $X_j$ has height 1 and width at least $\ell_{j+1}$.
$a(Y_j) \ge (1-\ell_j)/2$, since $Y_j$ has width $1 - \ell_j$ and height more than $1/2$.
Therefore,
\begin{align*}
a(J_1) &= \sum_{j=1}^{m_1} (a(X_j) + a(Y_j))
\ge \sum_{j=1}^{m_1} (\ell_{j+1} + (1-\ell_j)/2)
\\ &\ge \sum_{j=1}^{m_1} \left(\frac{\ell_{j+1}}{2} + \frac{1}{4}
    + \frac{1}{2} - \frac{\ell_j}{2}\right)  \tag{$\ell_{j+1} \ge 1/2$}
\\ &= \frac{3m_1}{4} + \frac{1}{4} - \frac{\ell_1}{2}
\ge \frac{3m_1-1}{4}.
\end{align*}
Therefore, $m_1 \le 4a(J_1)/3 + 1/3$.
% This is tight.
\end{proof}

An analogue of \cref{thm:area-bound} can be proven for type-2 bins.

Let $m$ be the number of bins used by $\greedyPack(\Wtild, \Htild)$.
After $j$ bins have been packed, let $A_j$ be the height of the remaining items in $\Wtild$
and $B_j$ be the width of the remaining items in $\Htild$.
Let $t_j$ be the type of the $j\Th$ bin (1 for type-1 bin and 2 for type-2 bin).
So $t_j = 1 \iff A_{j-1} \ge B_{j-1}$.

We first show that $|A_{j-1} - B_{j-1}| \le 1 \implies |A_j - B_j| \le 1$.
This means that once the difference between $h(\Wtild)$ and $w(\Htild)$ becomes at most 1,
it continues to stay at most 1.
Next, we show that $t_j \neq t_{j+1} \implies |A_{j-1} - B_{j-1}| \le 1$.
This means that if there are some bins of type 1 and some bins of type 2,
then the difference between $h(\Wtild)$ and $w(\Htild)$ will eventually become at most 1.
In the first non-full bin, we will use up all the wide items or the tall items.
Then the remaining items will have total height or total width at most 1,
so we can have at most 1 more non-full bin.
This would imply that there can be at most 2 non-full bins
when we have both type-1 and type-2 bins.

In the $j\Th$ bin, let $a_j$ be the height of items from $\Wtild$
and $b_j$ be the width of items from $\Htild$.
Hence, for all $j \in [m]$,
$A_{j-1} = A_j + a_j$ and $B_{j-1} = B_j + b_j$.

\begin{lemma}
\label{thm:diff-capture}
$|A_{j-1} - B_{j-1}| \le 1 \implies |A_j - B_j| \le 1$.
\end{lemma}
\begin{proof}
\textbf{Case 1:} $A_{j-1} - B_{j-1} \in [0, 1]$.
\\ This means that $t_j = 1$.

Assume (for the sake of proof by contradiction) that $a_j < b_j$.
Then $a_j < 1$, so we used up all of $\Wtild$ in the $j\Th$ bin.
Therefore, $A_j = 0$ and $A_{j-1} = a_j$. Therefore,
\[ A_{j-1} = a_j < b_j \le b_j + B_j = B_{j-1} \implies \bot. \]
Therefore, $a_j \ge b_j$. Since $A_{j-1} - B_{j-1} \in [0, 1]$
and $a_j - b_j \in [0, 1]$, we get
\[ A_j - B_j = (A_{j-1} - B_{j-1}) - (a_j - b_j) \in [-1, 1]. \]
Therefore, $|A_j - B_j| \le 1$.

\textbf{Case 2:} $A_{j-1} - B_{j-1} \in [-1, 0)$.
\\ This means that $t_j = 2$.
By an analysis similar to case 1, we get $|A_j - B_j| \le 1$.
\end{proof}

\begin{lemma}
\label{thm:tdiff-implies-adiff}
$t_j \neq t_{j+1} \implies |A_{j-1} - B_{j-1}| \le 1$.
\end{lemma}
\begin{proof}
\begin{align*}
& t_j = 1 \textrm{ and } t_{j+1} = 2
\\ &\implies A_{j-1} \ge B_{j-1} \textrm{ and } A_j < B_j
\\ &\implies B_{j-1} \le A_{j-1} < B_{j-1} + a_j - b_j
\\ &\implies A_{j-1} - B_{j-1} \in \ropenInterval{0, 1}.
\end{align*}
\begin{align*}
& t_j = 2 \textrm{ and } t_{j+1} = 1
\\ &\implies A_{j-1} < B_{j-1} \textrm{ and } A_j \ge B_j
\\ &\implies A_{j-1} < B_{j-1} \le A_{j-1} + (b_j - a_j)
\\ &\implies A_{j-1} - B_{j-1} \in \ropenInterval{-1, 0}.
\qedhere \end{align*}
\end{proof}

\begin{lemma}
\label{thm:non-full-bounded-diff}
Let there be $p$ full bins. If all bins don't have the same type, then $|A_p - B_p| \le 1$.
\end{lemma}
\begin{proof}
If $m = p$, then $A_p = B_p = 0$, so $|A_p - B_p| \le 1$ trivially holds.
If $m = p+1$, then $A_p \le 1$ and $B_p \le 1$, so $|A_p - B_p| \le 1$ trivially holds.
So now assume $m \ge p+2$.

Suppose that in the $(p+1)\Th$ bin, we used up all items from $\Wtild$ but not $\Htild$.
Then $\forall i \ge p+2$, $t_i = 2$.
Since all bins don't have the same type, $\exists k \le p+1$ such that
$t_k = 1$ and $t_{k+1} = 2$.
By \cref{thm:tdiff-implies-adiff,thm:diff-capture}, we get $|A_p - B_p| \le 1$.
Similarly, if we used up all items from $\Htild$ in the $(p+1)\Th$ bin, then $|A_p - B_p| \le 1$.
\end{proof}

\begin{lemma}
\label{thm:non-full-ub}
If all bins don't have the same type, then there can be at most 2 non-full bins.
\end{lemma}
\begin{proof}
Let there be $p$ full bins. Assume that there are more than 2 non-full bins.
\WLoG, assume that the first non-full bin used up all wide items.
Hence, $A_{p+1} = 0$.
By \cref{thm:non-full-bounded-diff}, we get $|A_p - B_p| \le 1$.
By \cref{thm:diff-capture}, we get $|A_{p+1} - B_{p+1}| \le 1$,
which implies that $B_{p+1} \le 1$.
Hence, the $(p+1)\Th$ bin will have type 2 and will use up all tall items,
so there can be at most 2 non-full bins.
\end{proof}

\begin{theorem}
\label{thm:greedy-pack-bins}
The number of bins used by $\greedyPack$ is at most
\[ \max\left(\smallceil{\hsum(\Wtild)}, \smallceil{\wsum(\Htild)},
    \frac{4}{3}a(\Wtild \cup \Htild) + \frac{8}{3}\right). \]
\end{theorem}
\begin{proof}
Let there be $m$ bins in the output of $\greedyPack(\Wtild, \Htild)$.
If all bins have the same type, then $m \le \max(\smallceil{\hsum(\Wtild)}, \smallceil{\wsum(\Htild)})$.

Let there be $m_1$ full bins of type 1 and let $J_1$ be the items inside those bins.
Let there be $m_2$ full bins of type 2 and let $J_2$ be the items inside those bins.
Then by \cref{thm:area-bound}, we get $m_1 \le 4a(J_1)/3 + 1/3$ and $m_2 \le 4a(J_2)/3 + 1/3$.
Hence, $m_1 + m_2 \le 4a(\Wtild \cup \Htild)/3 + 2/3$.
If all bins don't have the same type, then by \cref{thm:non-full-ub},
there can be at most 2 non-full bins, so $\greedyPack(\Wtild, \Htild)$
uses at most $4a(\Wtild \cup \Htild)/3 + 8/3$ bins.
\end{proof}

\subsection{Overview of \texorpdfstring{$\thinGPack$}{thin4Pack}}

$\thinGPack$ takes as input a set $I$ of rectangular items
and a parameter $\eps \in (0, 1)$.
It outputs a 4-stage bin packing of $I$.

$\thinGPack$ has the following outline:
\begin{enumerate}
\item Use linear grouping to round up the width or height of each item in $I$.
    This gives us a new instance $\Ihat$.
\item Pack $\Ihat$ into $1/\eps^2 + 1$ shelves,
    after possibly \emph{slicing} some items.
    Each shelf has width or height more than $1/2$ and is fully packed, i.e.,
    the total area of items in a shelf equals the area of the shelf.
    If we treat each shelf as an item, we get a new instance $\Itild$.
\item Compute a packing of $\Itild$ into bins, after possibly slicing some items,
    using $\greedyPack$.
\item Pack most of the items of $I$ into the shelves in the bins. We will prove that
    the remaining items have very small area, so they can be packed separately.
\end{enumerate}

To simplify our algorithm, we assume that $\eps^{-1} \in \mathbb{Z}$.

\subsection{Item Classification and Rounding}
\label{sec:bp-algo:round}

Define $W \defeq \{i \in I: h(i) \le \delta_H \}$ and $H \defeq I - W$.
Items in $W$ are called \emph{wide} and items in $H$ are called \emph{tall}.
Let $\WLarge \defeq \{i \in W: w(i) > \eps \}$ and $\WSmall \defeq W - \WLarge$.
Similarly, let $\HLarge \defeq \{i \in H: h(i) > \eps \}$ and $\HSmall \defeq H - \HLarge$.

We will now use \emph{linear grouping}~\cite{bp-aptas,kenyon1996strip}
to round up the widths of items in $\WLarge$ and the heights of items in $\HLarge$.
Arrange the items of $\WLarge$ in decreasing order of width and stack them
one-over-the-other (i.e., the widest item in $\WLarge$ is at the bottom).
Let $h_L$ be the height of the stack.
Let $y(i)$ be the $y$-coordinate of the bottom edge of item $i$.
Split the stack into sections of height $\eps^2h_L$ each.
For $j \in [1/\eps^2]$, let $w_j$ be the width of the
widest item intersecting the $j\Th$ section from the bottom, i.e.,
\[ w_j \defeq \max(\{w(i): i \in \WLarge \textrm{ and } (y(i), y(i) + h(i))
    \cap ((j-1)\eps^2 h_L, j\eps^2 h_L) \neq \emptyset\}). \]
Round up the width of each item $i$ to the smallest $w_j$ that is at least $w(i)$
(see \cref{fig:tall-lingroup-2}).
Let $\WLarge_j$ be the items whose width got rounded to $w_j$
and let $\WhatLarge_j$ be the resulting rounded items.
(There may be ties, i.e., there may exist $j_1 < j_2$ such that $w_{j_1} = w_{j_2}$.
In that case, define $\WLarge_{j_2} \defeq \WhatLarge_{j_2} = \emptyset$.
This ensures that all $\WLarge_j$ are disjoint.)
Let $\WhatLarge \defeq \bigcup_j \WhatLarge_j$.

\begin{figure}[htb]
\centering
\ifcsname pGameL\endcsname\else\newlength{\pGameL}\fi
\ifcsname pGameM\endcsname\else\newlength{\pGameM}\fi
\setlength{\pGameL}{0.3cm}
\setlength{\pGameM}{0.1cm}
\tikzset{bin/.style={draw,thick}}
\tikzset{binGrid/.style={draw,xstep=1\pGameL,ystep=1\pGameM,{textColor!20!bgColor}}}
\tikzset{item/.style={draw,fill={textColor!20!bgColor}}}
\tikzset{sepline/.style={draw,dashed,thick}}
\tikzset{limline/.style={densely dashed,ultra thick}}
\tikzset{myarrow/.style={->,>={Stealth},thick}}
\tikzset{mybrace/.style={decoration={amplitude=7pt,brace,mirror,raise=2pt},semithick,decorate}}
\definecolor{color1}{Hsb}{0,0.4,0.9}
\definecolor{color1d}{Hsb}{0,0.8,0.6}
\definecolor{color2}{Hsb}{60,0.4,0.9}
\definecolor{color2d}{Hsb}{60,0.8,0.6}
\definecolor{color3}{Hsb}{135,0.4,0.9}
\definecolor{color3d}{Hsb}{135,0.8,0.6}
\definecolor{color4}{Hsb}{240,0.4,0.9}
\definecolor{color4d}{Hsb}{240,0.8,0.6}
\begin{tikzpicture}[rotate=90]
\begin{scope}
%\path[binGrid] (0\pGameL, 0\pGameM) grid (50\pGameL, 50\pGameM);
\path[item] (0\pGameL, 0\pGameM) rectangle +(2\pGameL, 48\pGameM);
\path[item] (2\pGameL, 0\pGameM) rectangle +(3\pGameL, 46\pGameM);
\path[item] (5\pGameL, 0\pGameM) rectangle +(3\pGameL, 46\pGameM);
\path[item] (8\pGameL, 0\pGameM) rectangle +(2\pGameL, 42\pGameM);
\path[item] (10\pGameL, 0\pGameM) rectangle +(3\pGameL, 39\pGameM);
\path[item] (13\pGameL, 0\pGameM) rectangle +(3\pGameL, 37\pGameM);
\path[item] (16\pGameL, 0\pGameM) rectangle +(2\pGameL, 32\pGameM);
\path[item] (18\pGameL, 0\pGameM) rectangle +(4\pGameL, 27\pGameM);
\path[item] (22\pGameL, 0\pGameM) rectangle +(2\pGameL, 26\pGameM);
\path[item] (24\pGameL, 0\pGameM) rectangle +(3\pGameL, 23\pGameM);
\path[item] (27\pGameL, 0\pGameM) rectangle +(5\pGameL, 20\pGameM);
\path[item] (32\pGameL, 0\pGameM) rectangle +(2\pGameL, 20\pGameM);
\path[item] (34\pGameL, 0\pGameM) rectangle +(3\pGameL, 15\pGameM);
\path[item] (37\pGameL, 0\pGameM) rectangle +(3\pGameL, 14\pGameM);
%\path[bin] (0\pGameL, 0\pGameM) rectangle (50\pGameL, 50\pGameM);
\path[sepline] (10\pGameL, -2\pGameM) -- (10\pGameL, 50\pGameM);
\path[sepline] (20\pGameL, -2\pGameM) -- (20\pGameL, 50\pGameM);
\path[sepline] (30\pGameL, -2\pGameM) -- (30\pGameL, 50\pGameM);
\draw[limline,color1d] (0\pGameL, 48\pGameM) -- (40\pGameL, 48\pGameM);
\draw[limline,color2d] (0\pGameL, 39\pGameM) -- (40\pGameL, 39\pGameM);
\draw[limline,color3d] (0\pGameL, 27\pGameM) -- (40\pGameL, 27\pGameM);
\draw[limline,color4d] (0\pGameL, 20\pGameM) -- (40\pGameL, 20\pGameM);
\node[anchor=north] at (0\pGameL, 48\pGameM) {$w_1$};
\node[anchor=north] at (0\pGameL, 39\pGameM) {$w_2$};
\node[anchor=north] at (0\pGameL, 27\pGameM) {$w_3$};
\node[anchor=north] at (0\pGameL, 20\pGameM) {$w_4$};
\fill[color1d] (0\pGameL, 48\pGameM) circle [radius=2.5pt];
\fill[color2d] (10\pGameL, 39\pGameM) circle [radius=2.5pt];
\fill[color3d] (20\pGameL, 27\pGameM) circle [radius=2.5pt];
\fill[color4d] (30\pGameL, 20\pGameM) circle [radius=2.5pt];
\draw[mybrace] (0,0) -- node[right=7pt] {$\eps^2 h_L$} +(10\pGameL,0);
\end{scope}
\draw[myarrow] (22\pGameL, -5\pGameM) -- (22\pGameL, -25\pGameM);
\begin{scope}[yshift=-7.5cm]
%\path[binGrid] (0\pGameL, 0\pGameM) grid (50\pGameL, 50\pGameM);
\path[item,fill=color1] (0\pGameL, 0\pGameM) rectangle +(2\pGameL, 48\pGameM);
\path[item,fill=color1] (2\pGameL, 0\pGameM) rectangle +(3\pGameL, 48\pGameM);
\path[item,fill=color1] (5\pGameL, 0\pGameM) rectangle +(3\pGameL, 48\pGameM);
\path[item,fill=color1] (8\pGameL, 0\pGameM) rectangle +(2\pGameL, 48\pGameM);
\path[item,fill=color2] (10\pGameL, 0\pGameM) rectangle +(3\pGameL, 39\pGameM);
\path[item,fill=color2] (13\pGameL, 0\pGameM) rectangle +(3\pGameL, 39\pGameM);
\path[item,fill=color2] (16\pGameL, 0\pGameM) rectangle +(2\pGameL, 39\pGameM);
\path[item,fill=color3] (18\pGameL, 0\pGameM) rectangle +(4\pGameL, 27\pGameM);
\path[item,fill=color3] (22\pGameL, 0\pGameM) rectangle +(2\pGameL, 27\pGameM);
\path[item,fill=color3] (24\pGameL, 0\pGameM) rectangle +(3\pGameL, 27\pGameM);
\path[item,fill=color4] (27\pGameL, 0\pGameM) rectangle +(5\pGameL, 20\pGameM);
\path[item,fill=color4] (32\pGameL, 0\pGameM) rectangle +(2\pGameL, 20\pGameM);
\path[item,fill=color4] (34\pGameL, 0\pGameM) rectangle +(3\pGameL, 20\pGameM);
\path[item,fill=color4] (37\pGameL, 0\pGameM) rectangle +(3\pGameL, 20\pGameM);
\path[sepline] (10\pGameL, -2\pGameM) -- (10\pGameL, 50\pGameM);
\path[sepline] (20\pGameL, -2\pGameM) -- (20\pGameL, 50\pGameM);
\path[sepline] (30\pGameL, -2\pGameM) -- (30\pGameL, 50\pGameM);
%\path[bin] (0\pGameL, 0\pGameM) rectangle (50\pGameL, 50\pGameM);
\end{scope}
\end{tikzpicture}

\caption{Linear grouping of $\WLarge$ for $\eps = 1/2$.}
\label{fig:tall-lingroup-2}
\end{figure}

Allow horizontally slicing each item in $\WhatLarge$.
Let $\WhatSmall$ be the same as $\WSmall$, except that we are allowed
to slice items in $\WhatSmall$ both horizontally and vertically.
Let $\What \defeq \WhatLarge \cup \WhatSmall$.
Define $\Hhat$ analogously. Let $\Ihat \defeq \What \cup \Hhat$.

\begin{claim}
Items in $\WhatLarge$ have at most $1/\eps^2$ distinct widths.
Items in $\HhatLarge$ have at most $1/\eps^2$ distinct heights.
\end{claim}

\begin{lemma}
\label{thm:lingroup-opt-compare}
$\opt(\Ihat) < (1+\eps)\opt(I) + 2$.
\end{lemma}
\begin{proof}
Consider the optimal packing of $I$.
To convert this to a packing of $\Ihat - (\WhatLarge_1 \cup \HhatLarge_1)$,
unpack $\WLarge_1$ and $\HLarge_1$, and for each $j \in [1/\eps^2-1]$,
pack $\WhatLarge_{j+1}$ in the place of $\WLarge_j$
and pack $\HhatLarge_{j+1}$ in the place of $\HLarge_j$,
possibly after slicing the items.
Therefore,
\begin{equation}
\label{eqn:lingroup-1}
\opt(\Ihat - (\WhatLarge_1 \cup \HhatLarge_1)) \le \opt(I).
\end{equation}

We can pack $\HhatLarge_1$ in a bin by stacking the items side-by-side on the base of bins.
We can pack $\WhatLarge_1$ in a bin by stacking the items one-over-the-other.
Let $w_L$ be the total width of items in $\HhatLarge$. The number of bins used is
$\smallceil{\eps^2h_L} + \smallceil{\eps^2w_L}$. Also,
\[ \opt(I) \ge \opt(\WLarge \cup \HLarge)
\ge a(\WLarge) + a(\HLarge) \ge \eps(h_L + w_L). \]
Therefore,
\begin{equation}
\label{eqn:lingroup-2}
\opt(\WhatLarge_1 \cup \HhatLarge_1)
\le \smallceil{\eps^2h_L} + \smallceil{\eps^2w_L} < \eps\opt(I) + 2.
\end{equation}
On combining \eqref{eqn:lingroup-1} and \eqref{eqn:lingroup-2}, we get
\[ \opt(\Ihat) \le \opt(\Ihat - (\WhatLarge_1 \cup \HhatLarge_1))
    + \opt(\WhatLarge_1 \cup \HhatLarge_1)
< (1+\eps)\opt(I) + 2.
\qedhere \]
\end{proof}

\subsection{Creating Containers}
\label{sec:bp-algo:create-containers}

We will use ideas from Kenyon and R\'emila's strip packing algorithm~\cite{kenyon1996strip}
to pack $\Ihat$ into containers and pack the containers into shelves.
In the strip packing problem, we are given a set of rectangular items,
and we have to pack them into a bin of width 1 and minimum height.

Since we allow horizontally slicing items in $\What$,
a packing of $\What$ into $m$ bins gives us a
packing of $\What$ into a strip of height $m$,
and a packing of $\What$ into a strip of height $h'$ gives us a
packing of $\What$ into $\smallceil{h'}$ bins.
Hence, if we denote the optimal strip packing of $\What$ by $\optSP(\What)$,
then $\opt(\What) = \smallceil{\optSP(\What)}$.
We will now try to compute a near-optimal strip packing of $\What$.

Define a horizontal configuration $S$ as a tuple of $1/\eps^2+1$ non-negative integers,
where $S_0 \in \{0, 1\}$ and $\sum_{j=1}^{1/\eps^2} S_jw_j \le 1$.
For any horizontal line at height $y$ in a strip packing of $\What$,
the multiset of items intersecting the line corresponds to a configuration.
$S_0$ indicates whether the line intersects items from $\WhatSmall$,
and $S_j$ is the number of items from $\WhatLarge_j$ that the line intersects.
Let $\Scal$ be the set of all horizontal configurations. Let $N \defeq |\Scal|$.

To obtain an optimal packing, we need to determine the height of each configuration.
This can be done with the following linear program.
\[ \optprog{\min_{x \in \mathbb{R}^N}}{\sum_{S \in \Scal} x_S}{
\\[1.5em] \textrm{where} & \sum_{S \in \Scal} S_jx_S = h(\WhatLarge_j)
    & \forall j \in [1/\eps^2]
\\ \textrm{and} & \sum_{S: S_0=1} \left(1 - \sum_{j=1}^{1/\eps^2}S_jw_j\right)x_S
    = a(\WhatSmall)
\\ \textrm{and} & x_S \ge 0 & \forall S \in \Scal
} \]
Let $x^*$ be an optimal extreme-point solution to the above LP.
This gives us a packing where the strip is divided into rectangular regions
called \emph{shelves} that are stacked on top of each other.
Each shelf has a configuration $S$ associated with it
and has height $h(S) \defeq x^*_S$ and contains $S_j$ \emph{containers} of width $w_j$.
Containers of width $w_j$ only contain items from $\WhatLarge_j$,
and we call them \emph{type-$j$} containers.
If $S_0 = 1$, $S$ also contains a container of width $1 - \sum_{j=1}^{1/\eps^2} S_jw_j$
that contains small items. We call this container a \emph{type-$0$} container.
Each container is fully filled with items.
Let $w(S)$ denote the width of shelf $S$, i.e., the sum of widths of all containers in $S$.
Note that if $S_0 = 1$, then $w(S) = 1$. Otherwise, $w(S) = \sum_{j=1}^{1/\eps^2} S_jw_j$.

\begin{lemma}
\label{thm:nonneg-entries}
$x^*$ contains at most $1/\eps^2+1$ positive entries.
\end{lemma}
\begin{proof}[Proof sketch]
Follows by applying rank lemma (\cref{thm:rank-lemma-corr}) to the linear program.
\end{proof}

\begin{lemma}
\label{thm:width-gt-half}
$x_S^* > 0 \implies w(S) > 1/2$.
\end{lemma}
\begin{proof}
Suppose $w(S) \le 1/2$. Then we could have split $S$ into two parts by making a horizontal cut
in the middle and packed the parts side-by-side, reducing the height of the strip by $x^*_S/2$.
But that would contradict the fact that $x^*$ is optimal.
\end{proof}

Now treat each shelf $S$ as an item of width $w(S)$ and height $h(S)$.
Allow each such item to be sliced using horizontal cuts.
This gives us a new set $\Wtild$ of items such that $\What$ can be packed inside $\Wtild$.
By \cref{thm:width-gt-half}, each item in $\Wtild$ has width more than $1/2$.

By applying an analogous approach to $\Hhat$, we get a new set $\Htild$ of items.
Let $\Itild \defeq \Wtild \cup \Htild$.
We call the shelves of $\Wtild$ \emph{horizontal shelves} and the shelves of $\Htild$
\emph{vertical shelves}. The containers in horizontal shelves are called \emph{wide containers}
and the containers in vertical shelves are called \emph{tall containers}.

\begin{claim}
\label{thm:area-eq}
$a(\Itild) = a(\Ihat)$.
\end{claim}
\begin{lemma}
\label{thm:hw-opt}
Let $h(\Wtild)$ be the sum of heights of all items in $\Wtild$.
Let $w(\Htild)$ be the sum of widths of all items in $\Htild$.
%Let $\opt(\Ihat)$ be the optimal bin packing of $\Ihat$.
Then $\max(\smallceil{h(\Wtild)}, \smallceil{w(\Htild)}) \le \opt(\Ihat)$.
\end{lemma}
\begin{proof}
Since $x^*$ is the optimal solution to the linear program for strip packing $\What$,
$h(\Wtild) = \sum_{S \in \Scal} x^*_S = \optSP(\What)$.
Therefore, $\smallceil{h(\Wtild)} = \opt(\What) \le \opt(\Ihat)$.
Similarly, $\smallceil{w(\Htild)} = \opt(\Hhat) \le \opt(\Ihat)$.
\end{proof}

\subsection{Packing Shelves Into Bins}
\label{sec:bp-algo:pack-shelves-into-bins}

So far, we have packed $\Ihat$ into shelves $\Wtild$ and $\Htild$.
We will now use $\greedyPack(\Wtild, \Htild)$ to pack the shelves into bins.
By \cref{thm:greedy-pack}, we get a 2-stage packing of $\Wtild \cup \Htild$
into $m$ bins, where we make at most $m-1$ horizontal cuts in $\Wtild$
and at most $m-1$ vertical cuts in $\Htild$.

By \cref{thm:nonneg-entries}, we get a packing of $m+1/\eps^2$ horizontal shelves
and $m+1/\eps^2$ vertical shelves into $m$ bins.

By \cref{thm:greedy-pack-bins,thm:hw-opt,thm:area-eq}, we get that
\[ m \le \max\left(\smallceil{h(\Wtild)}, \smallceil{w(\Htild)},
    \frac{4}{3}a(\Itild) + \frac{8}{3}\right)
\le \frac{4}{3}\opt(\Ihat) + \frac{8}{3}. \]

\subsection{Packing Items Into Containers}
\label{sec:bp-algo:pack-into-containers}

We will now try to pack a large subset of items into the containers.
See \cref{fig:thin-gpack-output} for an example output.

\begin{figure}[htb]
\centering
\ifcsname myu\endcsname\else\newlength{\myu}\fi
\setlength{\myu}{0.4cm}
\tikzset{mytransform/.style={}}
\tikzset{sepline/.style={draw,thick}}
\tikzset{halfsepline/.style={draw}}
\tikzset{wShelf/.style={draw,thick}}
\tikzset{hShelf/.style={draw,thick}}
\tikzset{wItem/.style={draw,fill={textColor!20!bgColor},very thin}}
\tikzset{hItem/.style={draw,fill={textColor!10!bgColor},very thin}}
\tikzset{bin/.style={draw,thick}}
\tikzset{binGrid/.style={draw,step=1\myu,{textColor!20!bgColor}}}
\begin{tikzpicture}[mytransform]
%\path[binGrid] (0\myu, 0\myu) grid (20\myu, 20\myu);

\path[hItem] (0.0\myu, 0.0\myu) rectangle +(1.0\myu, 1.0\myu);
\path[hItem] (1.0\myu, 0.0\myu) rectangle +(1.0\myu, 1.0\myu);
\path[hItem] (2.0\myu, 0.0\myu) rectangle +(0.8\myu, 0.9\myu);
\path[hItem] (0.0\myu, 1.0\myu) rectangle +(1.0\myu, 0.8\myu);
\path[hItem] (1.0\myu, 1.0\myu) rectangle +(0.8\myu, 0.7\myu);
\path[hItem] (1.8\myu, 1.0\myu) rectangle +(0.8\myu, 0.7\myu);
\path[hItem] (0.0\myu, 1.8\myu) rectangle +(0.7\myu, 0.7\myu);
\path[hItem] (0.7\myu, 1.8\myu) rectangle +(0.8\myu, 0.65\myu);
\path[hItem] (1.5\myu, 1.8\myu) rectangle +(0.6\myu, 0.6\myu);
\path[hItem] (2.1\myu, 1.8\myu) rectangle +(0.6\myu, 0.55\myu);
\path[hItem] (0.0\myu, 2.5\myu) rectangle +(0.7\myu, 0.5\myu);
\path[hItem] (0.7\myu, 2.5\myu) rectangle +(0.8\myu, 0.5\myu);
\path[hItem] (1.5\myu, 2.5\myu) rectangle +(0.7\myu, 0.5\myu);
\path[hItem] (2.2\myu, 2.5\myu) rectangle +(0.7\myu, 0.5\myu);
\path[hItem] (0.0\myu, 3.0\myu) rectangle +(0.6\myu, 0.5\myu);
\path[hItem] (0.6\myu, 3.0\myu) rectangle +(0.7\myu, 0.48\myu);
\path[hItem] (1.3\myu, 3.0\myu) rectangle +(0.5\myu, 0.46\myu);
\path[hItem] (1.8\myu, 3.0\myu) rectangle +(0.4\myu, 0.44\myu);
\path[hItem] (2.2\myu, 3.0\myu) rectangle +(0.5\myu, 0.42\myu);
\path[hItem] (0.0\myu, 3.5\myu) rectangle +(0.5\myu, 0.4\myu);
\path[hItem] (0.5\myu, 3.5\myu) rectangle +(0.5\myu, 0.37\myu);
\path[hItem] (1.0\myu, 3.5\myu) rectangle +(0.5\myu, 0.34\myu);
\path[hItem] (1.5\myu, 3.5\myu) rectangle +(0.5\myu, 0.31\myu);
\path[hItem] (2.0\myu, 3.5\myu) rectangle +(0.4\myu, 0.28\myu);
\path[hItem] (2.4\myu, 3.5\myu) rectangle +(0.4\myu, 0.25\myu);

\path[hItem] (0.0\myu, 4\myu) rectangle +(0.5\myu, 6\myu);
\path[hItem] (0.5\myu, 4\myu) rectangle +(0.3\myu, 6\myu);
\path[hItem] (0.8\myu, 4\myu) rectangle +(0.7\myu, 6\myu);
\path[hItem] (1.5\myu, 4\myu) rectangle +(0.5\myu, 6\myu);
\path[hItem] (2.0\myu, 4\myu) rectangle +(0.6\myu, 6\myu);
\path[hItem] (0.0\myu, 10\myu) rectangle +(0.4\myu, 10\myu);
\path[hItem] (0.4\myu, 10\myu) rectangle +(0.6\myu, 10\myu);
\path[hItem] (1.0\myu, 10\myu) rectangle +(0.3\myu, 10\myu);
\path[hItem] (1.3\myu, 10\myu) rectangle +(0.8\myu, 10\myu);
\path[hItem] (2.1\myu, 10\myu) rectangle +(0.6\myu, 10\myu);

\path[hItem] (3.0\myu, 5\myu) rectangle +(0.6\myu, 7\myu);
\path[hItem] (3.6\myu, 5\myu) rectangle +(0.6\myu, 7\myu);
\path[hItem] (4.2\myu, 5\myu) rectangle +(0.7\myu, 7\myu);
\path[hItem] (3.0\myu, 12\myu) rectangle +(0.4\myu, 8\myu);
\path[hItem] (3.4\myu, 12\myu) rectangle +(0.5\myu, 8\myu);
\path[hItem] (3.9\myu, 12\myu) rectangle +(0.4\myu, 8\myu);
\path[hItem] (4.3\myu, 12\myu) rectangle +(0.4\myu, 8\myu);

\path[wItem] (5\myu, 0.0\myu) rectangle +(5\myu, 0.5\myu);
\path[wItem] (5\myu, 0.5\myu) rectangle +(5\myu, 0.8\myu);
\path[wItem] (5\myu, 1.3\myu) rectangle +(5\myu, 0.6\myu);
\path[wItem] (10\myu, 0.0\myu) rectangle +(5\myu, 0.5\myu);
\path[wItem] (10\myu, 0.5\myu) rectangle +(5\myu, 0.6\myu);
\path[wItem] (10\myu, 1.1\myu) rectangle +(5\myu, 0.4\myu);
\path[wItem] (10\myu, 1.6\myu) rectangle +(5\myu, 0.4\myu);
\path[wItem] (15\myu, 0.0\myu) rectangle +(5\myu, 0.7\myu);
\path[wItem] (15\myu, 0.7\myu) rectangle +(5\myu, 0.8\myu);
\path[wItem] (15\myu, 1.5\myu) rectangle +(5\myu, 0.3\myu);

\path[wItem] (6\myu, 2.0\myu) rectangle +(7\myu, 0.5\myu);
\path[wItem] (6\myu, 2.5\myu) rectangle +(7\myu, 0.8\myu);
\path[wItem] (6\myu, 3.3\myu) rectangle +(7\myu, 0.5\myu);
\path[wItem] (6\myu, 3.8\myu) rectangle +(7\myu, 0.5\myu);
\path[wItem] (6\myu, 4.3\myu) rectangle +(7\myu, 0.4\myu);
\path[wItem] (13\myu, 2.0\myu) rectangle +(7\myu, 0.8\myu);
\path[wItem] (13\myu, 2.8\myu) rectangle +(7\myu, 0.9\myu);
\path[wItem] (13\myu, 3.7\myu) rectangle +(7\myu, 0.2\myu);
\path[wItem] (13\myu, 3.9\myu) rectangle +(7\myu, 0.5\myu);
\path[wItem] (13\myu, 4.4\myu) rectangle +(7\myu, 0.3\myu);

\path[wItem] (6\myu, 5.0\myu) rectangle +(6\myu, 0.6\myu);
\path[wItem] (6\myu, 5.6\myu) rectangle +(6\myu, 0.3\myu);
\path[wItem] (6\myu, 5.9\myu) rectangle +(6\myu, 0.9\myu);
\path[wItem] (12\myu, 5.0\myu) rectangle +(4\myu, 0.3\myu);
\path[wItem] (12\myu, 5.3\myu) rectangle +(4\myu, 0.7\myu);
\path[wItem] (12\myu, 6.0\myu) rectangle +(4\myu, 0.4\myu);
\path[wItem] (12\myu, 6.4\myu) rectangle +(4\myu, 0.2\myu);
\path[wItem] (16\myu, 5.0\myu) rectangle +(4\myu, 0.5\myu);
\path[wItem] (16\myu, 5.5\myu) rectangle +(4\myu, 0.8\myu);
\path[wItem] (16\myu, 6.3\myu) rectangle +(4\myu, 0.4\myu);

\path[wItem] (7\myu, 7.0\myu) rectangle +(4\myu, 0.5\myu);
\path[wItem] (7\myu, 7.5\myu) rectangle +(4\myu, 0.8\myu);
\path[wItem] (7\myu, 8.3\myu) rectangle +(4\myu, 0.6\myu);
\path[wItem] (7\myu, 8.9\myu) rectangle +(4\myu, 0.8\myu);
\path[wItem] (11\myu, 7.0\myu) rectangle +(5\myu, 0.5\myu);
\path[wItem] (11\myu, 7.5\myu) rectangle +(5\myu, 0.6\myu);
\path[wItem] (11\myu, 8.1\myu) rectangle +(5\myu, 0.4\myu);
\path[wItem] (11\myu, 8.5\myu) rectangle +(5\myu, 0.8\myu);
\path[wItem] (11\myu, 9.3\myu) rectangle +(5\myu, 0.2\myu);
\path[wItem] (11\myu, 9.5\myu) rectangle +(5\myu, 0.3\myu);
\path[wItem] (16\myu, 7.0\myu) rectangle +(4\myu, 0.7\myu);
\path[wItem] (16\myu, 7.7\myu) rectangle +(4\myu, 0.8\myu);
\path[wItem] (16\myu, 8.5\myu) rectangle +(4\myu, 0.3\myu);
\path[wItem] (16\myu, 8.8\myu) rectangle +(4\myu, 0.5\myu);
\path[wItem] (16\myu, 9.3\myu) rectangle +(4\myu, 0.4\myu);

\path[wItem] (8\myu, 10.0\myu) rectangle +(5\myu, 0.5\myu);
\path[wItem] (8\myu, 10.5\myu) rectangle +(5\myu, 0.8\myu);
\path[wItem] (8\myu, 11.3\myu) rectangle +(5\myu, 0.5\myu);
\path[wItem] (8\myu, 11.8\myu) rectangle +(5\myu, 0.5\myu);
\path[wItem] (8\myu, 12.3\myu) rectangle +(5\myu, 0.4\myu);
\path[wItem] (8\myu, 12.6\myu) rectangle +(5\myu, 0.4\myu);
\path[wItem] (8\myu, 13.0\myu) rectangle +(5\myu, 0.8\myu);
\path[wItem] (13\myu, 10.0\myu) rectangle +(7\myu, 0.8\myu);
\path[wItem] (13\myu, 10.8\myu) rectangle +(7\myu, 0.9\myu);
\path[wItem] (13\myu, 11.7\myu) rectangle +(7\myu, 0.2\myu);
\path[wItem] (13\myu, 11.9\myu) rectangle +(7\myu, 0.5\myu);
\path[wItem] (13\myu, 12.4\myu) rectangle +(7\myu, 0.3\myu);
\path[wItem] (13\myu, 12.7\myu) rectangle +(7\myu, 0.9\myu);

\path[wItem] (8\myu, 14.0\myu) rectangle +(6\myu, 0.5\myu);
\path[wItem] (8\myu, 14.5\myu) rectangle +(6\myu, 0.3\myu);
\path[wItem] (8\myu, 14.8\myu) rectangle +(6\myu, 0.7\myu);
\path[wItem] (14\myu, 14.0\myu) rectangle +(6\myu, 0.7\myu);
\path[wItem] (14\myu, 14.7\myu) rectangle +(6\myu, 0.6\myu);
\path[wItem] (14\myu, 15.3\myu) rectangle +(6\myu, 0.6\myu);

\path[wItem] (9\myu, 16.0\myu) rectangle +(6\myu, 0.7\myu);
\path[wItem] (9\myu, 16.7\myu) rectangle +(6\myu, 0.6\myu);
\path[wItem] (9\myu, 17.3\myu) rectangle +(6\myu, 0.7\myu);
\path[wItem] (15\myu, 16.0\myu) rectangle +(5\myu, 0.6\myu);
\path[wItem] (15\myu, 16.6\myu) rectangle +(5\myu, 0.8\myu);
\path[wItem] (15\myu, 17.4\myu) rectangle +(5\myu, 0.2\myu);

\path[wItem] (10\myu, 18.0\myu) rectangle +(10\myu, 0.6\myu);
\path[wItem] (10\myu, 18.6\myu) rectangle +(10\myu, 0.7\myu);
\path[wItem] (10\myu, 19.3\myu) rectangle +(10\myu, 0.4\myu);

\path[sepline] (0\myu, 10\myu) -- +(3\myu, 0);
\path[halfsepline] (0\myu, 1.0\myu) -- +(3\myu, 0);
\path[halfsepline] (0\myu, 1.8\myu) -- +(3\myu, 0);
\path[halfsepline] (0\myu, 2.5\myu) -- +(3\myu, 0);
\path[halfsepline] (0\myu, 3.0\myu) -- +(3\myu, 0);
\path[halfsepline] (0\myu, 3.5\myu) -- +(3\myu, 0);

\path[sepline] (0\myu, 4\myu) -- +(3\myu, 0);
\path[sepline] (3\myu, 12\myu) -- +(2\myu, 0);
\path[sepline] (10\myu, 0\myu) -- +(0, 2\myu);
\path[sepline] (15\myu, 0\myu) -- +(0, 2\myu);
\path[sepline] (13\myu, 2\myu) -- +(0, 3\myu);
\path[sepline] (12\myu, 5\myu) -- +(0, 2\myu);
\path[sepline] (16\myu, 5\myu) -- +(0, 2\myu);
\path[sepline] (11\myu, 7\myu) -- +(0, 3\myu);
\path[sepline] (16\myu, 7\myu) -- +(0, 3\myu);
\path[sepline] (13\myu, 10\myu) -- +(0, 4\myu);
\path[sepline] (14\myu, 14\myu) -- +(0, 2\myu);
\path[sepline] (15\myu, 16\myu) -- +(0, 2\myu);

\path[wShelf] (5\myu, 0\myu) rectangle +(15\myu, 2\myu);
\path[wShelf] (6\myu, 2\myu) rectangle +(14\myu, 3\myu);
\path[wShelf] (6\myu, 5\myu) rectangle +(14\myu, 2\myu);
\path[wShelf] (7\myu, 7\myu) rectangle +(13\myu, 3\myu);
\path[wShelf] (8\myu, 10\myu) rectangle +(12\myu, 4\myu);
\path[wShelf] (8\myu, 14\myu) rectangle +(12\myu, 2\myu);
\path[wShelf] (9\myu, 16\myu) rectangle +(11\myu, 2\myu);
\path[wShelf] (10\myu, 18\myu) rectangle +(10\myu, 2\myu);
\path[hShelf] (0\myu, 0\myu) rectangle +(3\myu, 20\myu);
\path[hShelf] (3\myu, 5\myu) rectangle +(2\myu, 15\myu);
\draw[semithick,dashed] (5\myu, 0\myu) -- +(0\myu, 20\myu);
\path[bin] (0\myu, 0\myu) rectangle (20\myu, 20\myu);
\end{tikzpicture}

\caption[A type-1 bin in the packing of $\Ihat$ computed by $\thinGPack$.]%
{A type-1 bin in the packing of $\Ihat$ computed by $\thinGPack$.
The packing contains 5 tall containers in 2 tall shelves
and 18 wide containers in 8 wide shelves.}
\label{fig:thin-gpack-output}
\end{figure}

\begin{lemma}
\label{thm:discard-area-ub}
Let $P$ be a packing of $\Itild$ into $m$ bins, where we sliced horizontal shelves by making
at most $m-1$ horizontal cuts and sliced vertical shelves by making at most $m-1$ vertical cuts.
Then we can pack a large subset of items $I$ into the containers in $P$ such that
the unpacked items from $W$ have area less than
\[ \eps h(\Wtild) + \delta_H(1 + \eps)(m + 1/\eps^2), \]
and the unpacked items from $H$ have area less than
\[ \eps w(\Htild) + \delta_W(1 + \eps)(m + 1/\eps^2). \]
\end{lemma}
\begin{proof}
For each $j \in [1/\eps^2]$, number the type-$j$ wide containers arbitrarily,
and number the items in $\WLarge_j$ arbitrarily.
Now greedily assign items from $\WLarge_j$ to the first container $C$ until the total height
of the items exceeds $h(C)$. Then move to the next container and repeat.
As per the constraints of the linear program, all items in $\WLarge_j$
will get assigned to some type-$j$ wide container.
Similarly, number the type-0 wide containers arbitrarily
and number the items in $\WSmall$ arbitrarily.
Greedily assign items from $\WSmall$ to the first container $C$ until the total area
of the items exceeds $a(C)$. Then move to the next container and repeat.
As per the constraints of the linear program, all items in $\WSmall$
will get assigned to some type-0 wide container.
Similarly, assign all items from $H$ to tall containers.

Let $C$ be a type-$j$ wide container and $J$ be the items assigned to it.
If we discard the last item from $J$, then the items can be packed into $C$.
The area of the discarded item is at most $w(C)\delta_H$.
Let $C$ be a type-0 wide container and $J$ be the items assigned to it.
Arrange the items in $J$ in decreasing order of height and pack the largest
prefix $J' \subseteq J$ into $C$ using NFDW (Next-Fit Decreasing Width),
which is an analogue of NFDH with the coordinate axes swapped.

Discard the items $J - J'$. By \cref{thm:nfdh-small},
$a(J - J') < \eps h(C) + \delta_H w(C) + \eps\delta_H$.
Therefore, for a horizontal shelf $S$, the total area of discarded items is less than
$\eps h(S) + \delta_H(1 + \eps)$.

After slicing the shelves in $\Itild$ to get $P$,
we get at most $m + 1/\eps^2$ horizontal shelves
and at most $m + 1/\eps^2$ vertical shelves.
Therefore, the total area of discarded items from $W$ is less than
\[ \eps h(\Wtild) + \delta_H(1 + \eps)(m + 1/\eps^2), \]
and the total area of discarded items from $H$ is less than
\[ \eps w(\Htild) + \delta_W(1 + \eps)(m + 1/\eps^2). \qedhere \]
\end{proof}

We will pack the discarded items into new bins using NFDH (or NFDW),
and NFDH always outputs a 2-stage packing.
Since $\greedyPack$ outputs a 2-stage packing of the shelves
and the packing of items into the shelves is a 2-stage packing,
the bin packing of non-discarded items is a 4-stage packing.

\subsection{Summary}
\label{sec:thin-gpack:summary}

A summary of $\thinGPack$ is given in \cref{algo:thinGPack}.

\begin{algorithm}[!ht]
\caption{$\thinGPack_{\eps}(I)$: Packs items $I$ into square bins of side length 1,
where each item in $I$ has width at most $\delta_W$ or height at most $\delta_H$.}
\label{algo:thinGPack}
\begin{algorithmic}[1]
\State Let $W \defeq \{i \in I: h(i) \le \delta_H\}$ and $H \defeq I - W$.
\State Compute $\Ihat$ using linear grouping with parameter $\eps$ as per \cref{sec:bp-algo:round}.
\State Create shelves $\Itild$ from items $\Ihat$ as per \cref{sec:bp-algo:create-containers}.
\State Pack $\Itild$ into bins using $\greedyPack$.
\State Pack a large subset of $I$ into the shelves using \cref{thm:discard-area-ub}.
Let $W^d$ be the unpacked items from $W$ and $H^d$ be the unpacked items from $H$.
\State Pack $W^d$ into new bins using NFDH.
\State Pack $H^d$ into new bins using NFDW.
\end{algorithmic}
\end{algorithm}

\begin{restatable}{lemma}{rthmNfdhWide}
\label{thm:nfdh-wide}
Let $I$ be a set of rectangular items where each item has height at most $\delta$.
Then the number of bins required by NFDH to pack $I$ is less than
$(2a(I)+1)/(1-\delta)$.
\end{restatable}
\begin{proof}
The bin packing version of NFDH first packs $I$ into shelves
and then packs the shelves into bins using Next-Fit.
Let $H$ be the sum of heights of all the shelves.
By \cref{thm:nfdh-strip}, $H < 2a(I) + \delta$.
By \cref{thm:next-fit-small}, the number of bins
is less than $1 + H/(1-\delta) < (2a(I)+1)/(1-\delta)$.
\end{proof}

\rthmThinGPackProps*
\begin{proof}
$\greedyPack$ outputs a 2-stage packing, so $\thinGPack$ outputs a 4-stage packing.

Linear grouping takes $O(n\log n + 1/\eps^2)$ time.
Computing the shelves requires solving a linear program in at most
$2(1/\eps^2)^{1/\eps}$ variables and $1+1/\eps^2$ constraints.
$\greedyPack$ takes $O(n\log n)$ time.
Packing $I$ into containers takes $O(n\log n)$ time.
NFDH and NFDW take $O(n\log n)$ time.
\end{proof}

\rthmThinGPackAppx*
\begin{proof}
Suppose $\greedyPack$ uses at most $m$ bins. Then by \cref{thm:greedy-pack-bins},
\[ m \le 4\opt(\Ihat)/3 + 8/3. \]
Let $W^d$ and $H^d$ be the unpacked items from $W$ and $H$, respectively.
By \cref{thm:discard-area-ub,thm:hw-opt},
\[ a(W^d) < \eps\opt(\Ihat) + \delta_H(1 + \eps)(m + 1/\eps^2), \]
\[ a(H^d) < \eps\opt(\Ihat) + \delta_W(1 + \eps)(m + 1/\eps^2). \]
By \cref{thm:nfdh-wide}, the number of bins used by $\thinGPack_{\eps}(I)$ is less than
\begin{align*}
& m + \frac{2a(W^d)+1}{1-\delta_H} + \frac{2a(H^d)+1}{1-\delta_W}
\\ &\le (1 + 4\Delta(1+\eps))m + 4\eps(1+\Delta)\opt(\Ihat)
    + 2(1+\Delta) + 4\Delta(1+\eps)/\eps^2
\\ &\le \alpha\opt(\Ihat) + 2(\beta - 1)
< \alpha(1+\eps)\opt(I) + 2\beta.
\tag{by \cref{thm:lingroup-opt-compare}}
\end{align*}
\end{proof}
