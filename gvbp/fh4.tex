\section{The \texorpdfstring{$\fhk[4]$}{fullh4} Algorithm}
\label{sec:fh4}

In this section, we describe the $\fhk[4]$ algorithm
and prove an important result about it.
Although a more detailed analysis is given in \cref{sec:fhk},
the analysis here is simpler because we only focus on the part relevant
to \geomvec{$d_g$}{$d_v$} BP.

Define $f_4: [0, 1] \mapsto [0, 1]$ and $\type: [0, 1] \mapsto [4]$ as
\begin{align*}
f_4(x) &= \begin{cases}
\frac{1}{q} & x \in \left(\left.\frac{1}{q+1}, \frac{1}{q}\right]\right. \textrm{ for } q \in [3]
\\ 2x & x \le \frac{1}{4} \end{cases}
& \type(x) &= \begin{cases}
q & x \in \left(\left.\frac{1}{q+1}, \frac{1}{q}\right]\right. \textrm{ for } q \in [3]
\\ 4 & x \le \frac{1}{4} \end{cases}
\end{align*}
\begin{comment}
\right)\right) % to fix syntax highlighting in Vim
\end{comment}

Let $i$ be a $d_g$-dimensional cuboid.
Define $f_4(i)$ as the cuboid of length $f_4(\ell_j(i))$ in the $j\Th$ dimension.
Note that $x \le f_4(x) \le 2x$.
Hence, $\vol(i) \le \vol(f_4(i)) \le 2^{d_g}\vol(i)$.
For a set $I$ of cuboids, $f_4(I) \defeq \{f_4(i): i \in I\}$.
Define $\type(i) \in [4]^{d_g}$ as a vector
where $\type(i)_j \defeq \type(\ell_j(i))$.

Caprara~\cite{caprara2008} (implicitly) defines a recursive algorithm $\hdhkunit^{[t]}(I)$
(see \cref{sec:hdhk-prelims:hdhkunit} for more details),
that takes as input a sequence $I$ of $d$D cuboidal items, such that
all items have the same $\type$ $t$ and $\vol(f_4(I-\{\last(I)\})) < 1$,
and returns a packing of $I$ into a $d$D bin.
Here $\last(I)$ is the last item in sequence $I$.
$\hdhkunit^{[t]}(I)$ runs in $O(|I|\log|I|)$ time.

We now describe the $\fhk[4]$ algorithm.
First, partition the items $I$ by $\type$. The number of partitions is at most $Q = 4^{d_g}$.
Let $I^{[q]}$ be the partition containing items of $\type$ $q$.
Order the items in $I^{[q]}$ arbitrarily.
Then repeatedly pick the smallest prefix $J$ of $I^{[q]}$ such that
either $J = I^{[q]}$ or $\vol(f_4(J)) \ge 1$, and pack $J$ into a bin using
$\hdhkunit^{[q]}(J)$.

\begin{lemma}
\label{lem:fh4}
For a non-empty $d_g$D GBP instance $I$, $|\fhk[4](I)| < 4^{d_g} + 2^{d_g}\vol(I)$.
\end{lemma}
\begin{proof}
Suppose $\fhk[4](I^{[q]})$ produces $m^{[q]}$ bins. Let $B_j^{[q]}$ be the $j\Th$ of these bins.
Given the way we choose prefixes, $\vol(f_4(B_j^{[q]})) \ge 1$ for $j \in [m^{[q]}-1]$,
i.e., at most 1 bin is partially-filled. Hence,
\[ \vol(f_4(I^{[q]})) = \sum_{j=1}^{m^{[q]}} \vol(f_4(B_j^{[q]})) > m^{[q]}-1. \]
So, the total number of bins used is
\[ \sum_{q=1}^Q m^{[q]} < \sum_{q=1}^Q (1 + \vol(f_4(I^{[q]})))
= Q + \vol(f_4(I)) \le 4^{d_g} + 2^{d_g}\vol(I). \qedhere \]
\end{proof}

Therefore, $\vol(I) \le 1 \implies |\fhk[4](I)| \le 4^{d_g} + 2^{d_g} - 1$.
