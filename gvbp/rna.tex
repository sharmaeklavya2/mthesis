\section{Round-and-Approx Framework}
\label{sec:rna}

The R\&A framework is a simple but powerful technique,
originally given by Bansal, Caprara and Sviridenko~\cite{rna},
to improve the AAR of a bin packing algorithm by combining it with
randomized rounding of the \config{} LP
(recall the definition of \config{} LP from \cref{sec:config-lp}).
R\&A may have the potential to improve the AARs of several packing problems,
but its applicability is limited because
it only works with \emph{subset-oblivious} bin packing algorithms,
and proving that an algorithm is subset-oblivious is difficult.

Bansal and Khan~\cite{bansal2014binpacking} partially removed this limitation
by proving that a large class of algorithms for geometric and vector bin packing,
called \emph{rounding-based} algorithms, is subset-oblivious.
We make further improvements on this front by showing that an even larger class
of algorithms is subset-oblivious.
This class includes some of our algorithms for \geomvec{$d_g$}{$d_v$} BP,
like $\simplePack$ and $\cbPack$.
Algorithms in this class are characterized as a combination of
three simpler subroutines, called $\round$, $\complexPack$ and $\unround$.

We describe our version of the R\&A framework as a \emph{meta-algorithm},
i.e., the R\&A framework takes four subroutines as input---%
$\cLPsolve$, $\round$, $\complexPack$ and $\unround$---%
and returns an algorithm for bin packing, called $\rnaPack$.
$\rnaPack(I)$ first uses $\cLPsolve$ to (approximately) solve the \config{} LP of $I$.
Based on the (approximate) solution to the \config{} LP,
it packs a large fraction of the items in $I$.
It then packs the remaining items using the subroutines
$\round$, $\complexPack$ and $\unround$.
We prove that if $\round$, $\complexPack$ and $\unround$ satisfy some special properties,
then $\rnaPack$ has a low AAR.

In \cref{sec:rna:algo}, we describe the $\rnaPack$ algorithm in detail.
In \cref{sec:rna:old-compare}, we highlight the differences between our version of R\&A
and the previous versions of R\&A~\cite{rna,bansal2014binpacking}.
In \cref{sec:rna:frac-pack,sec:rna:round,sec:rna:complex-pack,sec:rna:unround},
we describe the subroutines $\round$, $\complexPack$ and $\unround$,
and mention the properties that they should satisfy.
In \cref{sec:rna:aar}, we prove an upper-bound on the AAR of $\rnaPack$.
In \cref{sec:rna:simple-pack}, we show how to apply the R\&A framework
to the $\simplePack$ algorithm.

\subsection{Comparison to Previous Versions of R\&A}
\label{sec:rna:old-compare}

To use R\&A, we need to solve the \config{} LP.
All previous applications ($d$D VBP and 2D GBP) of R\&A solved the \config{} LP
$(1+\eps)$-approximately using a $(1+O(\eps))$-approximate solution to
the corresponding knapsack problem.
Due to the unavailability of a PTAS for \geomvec{2}{$d$} KS,
we had to adapt and use a different linear programming algorithm~\cite{eku-pst}
that uses an $\eta$-approximation algorithm for the knapsack problem to
$(1+\eps)\eta$-approximately solve the \config{} LP,
for any constants $\eta > 1$ and $0 < \eps < 1$.

Second, we introduce more freedom in choosing the packing structure.
Unlike the R\&A framework in \cite{bansal2014binpacking},
that worked only for container-based packing,
we allow either relaxing the packing structure to non-container-based (like in $\simplePack$)
or imposing packing constraints in addition to being container-based (like in $\cbPack$).
This generalization can help in finding better algorithms for other variants of bin packing.

The \emph{rounding-based} algorithms in \cite{bansal2014binpacking}
work by rounding up the large dimensions of items to $O(1)$ different types.
In addition, we also allow \emph{rounding down} some dimensions, if we can find
a suitable way of \emph{unrounding} a packing of rounded items.
For example, in $\cbPack$, we round down the width and height of some items to 0.
It was shown in \cite{khan-thesis} that if the large coordinates of items are
rounded to $O(1)$ types, we cannot obtain AARs better than $d$ and $4/3$
for $d$D VBP and 2D GBP, respectively.
However, as we now allow rounding down, we may be able to use the R\&A framework
with algorithms having better approximation ratios.

We also fix a minor error in the R\&A framework of \cite{bansal2014binpacking}
(see \cref{sec:rna-extra:bugfix} for details).

\subsection{Description of the R\&A Algorithm}
\label{sec:rna:algo}

The algorithm $\rnaPack(I, \beta, \eps)$
takes as input a set $I$ of \geomvecdim{$d_g$}{$d_v$} items
and parameters $\beta \ge 1$ and $\eps \in (0, 1)$.
The steps of the algorithm are as follows
(see \cref{algo:rna-pack} for a more formal description).

\begin{enumerate}
\item \textbf{Solve the \Config{} LP} of $I$.
Use $\cLPsolve$ to obtain a $\mu$-asymptotic-approximate solution $\xhat$ to the
\config{} LP of $I$. Note that each index of $\xhat$ corresponds to a \config{}.

\item \textbf{Randomized rounding of \config{} LP}:
For $T \defeq \ceil{(\ln\beta)\smallnorm{\xhat}_1}$ steps, do the following:
Select a \config{} $C$ with probability $\xhat_C/\smallnorm{\xhat}_1$.
Pack $T$ bins according to each of these selected $T$ \config{}s.
Let $S$ be the remaining items that are not packed, called the \emph{residual instance}.

\item \textbf{Rounding of items}:
We define a subroutine $\round$ that takes items $I$ and parameter $\eps$ as input%
\footnote{The input to $\round$ is $I$ instead of $S$ because $S$ is random
and we want to round items deterministically, i.e.,
the rounding of each item $i \in S$ should
not depend on which other items from $I$ lie in $S$.
In fact, this is where the old R\&A framework~\cite{khan-thesis} introduced an error.
See \cref{sec:rna-extra:bugfix} for details.}.
It \emph{discards} a set $D \subseteq I$ of items such that $\Span(D) \le \eps\Span(I)$
and then \emph{modifies} each item in $I-D$ to get a set $\Itild$ of items.
We denote the output of $\round(I, \eps)$ as $(\Itild, D)$, where
items in $\Itild$ are called \emph{rounded items}.
Intuitively, after rounding, the items in $\Itild$ are of $O(1)$ types,
which makes packing easier.
Also, since $\Span(D)$ is small, $D \cap S$ can be packed
into a small number of bins using $\simplePack$.

We impose some restrictions on $\round$, which we denote as conditions C1 and C2,
that we describe in \cref{sec:rna:round}.
We also allow $\round$ to output a $O(\poly(n))$-sized list of guesses of $(\Itild, D)$.

\item \textbf{Pack rounded items}:
Let $\Stild$ be the rounded items corresponding to $S - D$.
Pack $\Stild$ into bins using any bin packing algorithm that satisfies `condition C3',
which we describe in \cref{sec:rna:complex-pack}.
Let us name this algorithm $\complexPack$.

\item \textbf{Unrounding}:
Given a bin packing of $\Stild$, let $\unround$ be a subroutine that
computes a bin packing of $S - D$.
$\unround$ is trivial in previous versions of R\&A, because they only increase dimensions
of items during rounding. In our applications, we may round down items,
so $\unround$ can be non-trivial.
$\unround$ can be any algorithm that satisfies `condition C4',
which we describe in \cref{sec:rna:unround}.
\end{enumerate}

\begin{algorithm}[htb]
\caption{$\rnaPack(I, \beta, \eps)$: Computes a bin packing of $I$.
$I$ is a set of \geomvecdim{$d_g$}{$d_v$} items and $\beta \ge 1$}
\label{algo:rna-pack}
\begin{algorithmic}[1]
\State $\xhat = \cLPsolve(I)$
\RepeatN{$T \defeq \ceil{(\ln\beta)\smallnorm{\xhat}_1}$}
    \State \label{alg-line:rna-pack:rround-choose}Select
        a \config{} $C$ with probability $\xhat_C/\smallnorm{\xhat}_1$.
    \State \label{alg-line:rna-pack:rround-pack}Pack a bin according to $C$.
\EndRepeatN
\State Let $S$ be the unpacked items from $I$.
    \Comment{$S$ is called the set of \emph{residual items}.}
\State Initialize $J_{\best}$ to \texttt{null}.
\For{$(\Itild, D) \in \round(I)$}
    \Comment{$\round(I)$ outputs a set of pairs.}
    \State $J_D = \simplePack(S \cap D)$
    \State Let $\pi$ be a bijection from $I-D$ to $\Itild$.
        Let $\Stild \defeq \{\pi(i): i \in S-D\}$.
    \State $\Jtild = \complexPack(\Stild)$
    \State $J = \unround(\Jtild)$
    \If{$J_{\best}$ is \texttt{null} or $|J_D \cup J| < |J_{\best}|$}
        \State $J_{\best} = J_D \cup J$
    \EndIf
\EndFor
\State Pack $S$ according to $J_{\best}$.
\end{algorithmic}
\end{algorithm}

The R\&A framework requires that $\round$, $\complexPack$ and $\unround$
satisfy four conditions C1, C2, C3, C4, which we describe in
\cref{sec:rna:round,sec:rna:complex-pack,sec:rna:unround}.
Prospective users of the R\&A framework need to design these three subroutines
and prove that they satisfy these four conditions.
In \cref{sec:rna:aar}, we prove that if these conditions are satisfied,
then $\rnaPack$ has a small AAR.

Intuitively, $\rnaPack$ first packs some items into $T$ bins using randomized rounding of $\xhat$.
We can prove that $\Pr[i \in S] \le 1/\beta$,
so $S$ contains a small fraction of the items in $I$
(see \cref{thm:pr-elem-in-residual} in \cref{sec:rna-extra}).
We will then try to prove that if the rest of the algorithm
($\round + \complexPack + \unround$)
packs $I$ into $m$ bins, then it will pack $S$ into roughly $m/\beta$ bins.
This notion was referred to in \cite{rna} as \emph{subset-obliviousness}.
We will use subset-obliviousness to bound the AAR of $\rnaPack$.

\Cref{sec:rna:simple-pack} shows how $\simplePack$
can be broken up into $\round$, $\complexPack$ and $\unround$
and used with the R\&A framework.

\subsection{Fractional structured packing}
\label{sec:rna:frac-pack}

Let $(\Itild, D)$ be an output of $\round(I)$ and let $\Xtild$
be an arbitrary subset of $\Itild$.
Our analysis of $\rnaPack$ is based around a concept called
\emph{fractional structured packing} of $\Xtild$.
Note that the notion of fractional structured packing only appears in the
analysis of $\rnaPack$. It is not needed to describe the algorithm.

We first define what it means to slice an item.
From a geometric perspective, slicing an item perpendicular to the $k\Th$ dimension
means cutting the item into 2 parts using a hyperplane perpendicular to the $k\Th$ axis.
E.g., for $d_g=2$, if $k=1$ for item $i$, then we slice $i$ using a vertical cut,
and if $k=2$, we slice $i$ using a horizontal cut.
The vector dimensions get split proportionately across the slices.

\begin{definition}[Slicing an item]
Let $i$ be a \geomvecdim{$d_g$}{$d_v$} item.
Slicing $i$ perpendicular to geometric dimension $k$
with proportionality $\alpha$ (where $0 < \alpha < 1$)
is the operation of replacing $i$ by two items $i_1$ and $i_2$ such that:
(i) $\forall j \neq k, \ell_j(i) = \ell_j(i_1) = \ell_j(i_2)$,
(ii) $\ell_k(i_1) = \alpha \ell_k(i)$ and $\ell_k(i_2) = (1-\alpha)\ell_k(i)$,
(iii) $\forall j \in [d_v], v_j(i_1) = \alpha v_j(i) \wedge v_j(i_2) = (1-\alpha) v_j(i)$.
\end{definition}

\begin{definition}[Fractional packing]
Let $\Itild$ be \geomvecdim{$d_g$}{$d_v$} items, where for each item $i \in \Itild$,
we are given a set $X(i)$ of axes perpendicular to which we can repeatedly slice $i$
($X(i)$ can be empty, which would mean that the item $i$ cannot be sliced).
If we slice items as per their given axes and then pack the slices into bins,
then the resulting packing is called a \emph{fractional} bin packing.
\end{definition}

\begin{figure}[htb]
\centering
\begin{tikzpicture}[
myarrow/.style = {->,>={Stealth},semithick},
mybrace/.style = {decoration={amplitude=3pt,brace,mirror,raise=1pt},semithick,decorate},
every node/.style = {scale=0.8},
scale=0.8
]
\draw (0,0) rectangle +(3,3);
\draw[fill={black!25}] (0,0) rectangle +(1.2,3);
\draw[fill={black!10}] (3,0) rectangle +(-1.5,1.2);
\draw[fill={black!10}] (3,1.2) rectangle +(-1.5,1.2);
\draw[mybrace] (0,0) -- node[below=1pt] {0.4} +(1.2,0);
\draw[mybrace] (1.5,0) -- node[below=1pt] {0.5} +(1.5,0);
\draw[mybrace] (3,0) -- node[right=2pt] {0.4} +(0,1.2);
\draw[mybrace] (3,1.2) -- node[right=2pt] {0.4} +(0,1.2);

\draw[fill={black!25}] (-8,0) rectangle +(1.2,3);
\path (-8,0) -- node[below=0pt] {0.4} +(1.2,0);
\path (-8,0) -- node[left=0pt] {1} +(0,3);
\node at (-6.2,1.5) {+};
\draw[fill={black!10}] (-5,1) rectangle +(3,1.2);
\path (-5,1) -- node[left=0pt] {0.4} +(0,1.2);
\draw[dashed] (-3.5,2.5) -- (-3.5,0.5);
\node[rotate=90,transform shape] at (-3.5,0.3) {\large\ding{34}};
\draw[mybrace] (-5,1) -- node[below=1pt] {0.5} +(1.5,0);
\draw[mybrace] (-3.5,1) -- node[below=1pt] {0.5} +(1.5,0);
\draw[myarrow] (-1.5,1.5) -- (-0.5,1.5);
\end{tikzpicture}

\caption{Example of a fractional packing of two items into a bin.}
\label{fig:frac-pack}
\end{figure}

We will now review a common approach used in the design of
approximation algorithms for packing problems,
which we will use to define \emph{fractional structured packing}.

Suppose we wanted to use a brute-force algorithm for bin packing.
Such an algorithm would enumerate all possible packings of the items into bins,
and pick the packing that required the minimum number of bins.
Of course, such an approach wouldn't work,
since the number of possible packings is exponential.
So, instead of enumerating all possible packings,
we will only consider a small subset of packings.
We will do this by carefully choosing a set of constraints,
and we will only consider packings that satisfy those constraints.
We call such packings \emph{structured}.
By carefully choosing the constraints, we hope that the
optimal structured packing would be a good approximation to the optimal packing,
and that the optimal structured packing would be easier to find than the optimal packing.

This idea of using structured packings isn't always enough by itself.
However, this idea has been successfully combined with the idea of fractional packing.
For example, Jansen and Pr\"adel~\cite{jansen2016new} showed that
given any packing of a 2D GBP instance into $m$ bins,
we can slice (a carefully chosen subset of) the items and then repack the items
into $(1.5+\eps)m + O(1)$ bins such that the resulting packing is \emph{container-based}.
\emph{Container-based} roughly means that in each bin,
items are packed into rectangular regions called containers,
and containers' heights and widths belong to a fixed set of $O(1)$ values.
In their algorithm, they find an almost-optimal fractional container-based packing
of the input, and show how to convert such a fractional packing into
a non-fractional packing without increasing the number of bins by too much.

This approach of using fractional packing together with structured packing
has been used in many approximation algorithms for bin packing
~\cite{jansen2016new,bansal2005tale} and knapsack~\cite{galvez2017approximating}.
Each of these algorithms uses a different definition of \emph{structured}.
For example, according to Jansen and Pr\"adel's algorithm,
a packing is structured iff it is container-based.
The common idea in these algorithms is to first show that the optimal
\emph{fractional structured packing} is a good approximation to the optimal packing
and then find a non-fractional packing that is roughly as good as the
optimal fractional structured packing.
We would like to formalize this idea and see if we can show that
such algorithms are subset-oblivious.

Formally, a fractional \config{} of $\Itild$ is a packing of slices of
some items from $\Itild$ into a bin.
Let $\Scal$ be a set of fractional \config{}s of $\Itild$.
Intuitively, $\Scal$ is the set of all structured fractional configurations.
For a set $\Xtild \subseteq \Itild$, a fractional bin packing of $\Xtild$
is said to be \emph{structured} (with respect to $\Scal$) iff
the \config{} of each bin in the packing belongs to $\Scal$.

We will assume that $\Scal$ is \emph{downward-closed}.
Intuitively, downward-closed means that if we remove some items from a structured packing,
the packing will remain structured. Formally,
for two fractional \config{}s $C_1$ and $C_2$, we say that $C_1$ is a sub\config{} of $C_2$
iff we can remove some items (or slices thereof) from $C_2$ to get $C_1$.
We say that $\Scal$ is downward-closed iff
for each $C \in \Scal$, all sub\config{}s of $C$ also lie in $\Scal$.
This assumption is necessary in our analysis of $\rnaPack$,
but this is a very mild assumption, since all published examples of structured packing
that we have come across are downward closed.

The R\&A framework of Bansal and Khan~\cite{bansal2014binpacking},
only worked with bin packing algorithms that used `container-based'
as their definition of structured packing.
Our R\&A framework, on the other hand, gives algorithm designers the freedom to define
the notion of structured packing (i.e., deciding on $\Scal$) in any way they want,
as long as it satisfies the downward closure property.
Typically, the choice of which definition of structured packing to use will depend
on the ease of proving Conditions C2 and C3
(which we describe in \cref{sec:rna:round,sec:rna:complex-pack}) for that definition.

Define $\fsopt(\Xtild)$ as the number of bins used in the optimal
fractional structured packing of $\Xtild \subseteq \Itild$.
To analyze the AAR of $\rnaPack$, we will bound the number of bins used to pack
the residual instance $S$ in terms of $\fsopt(\Stild)$,
and then we will bound $\fsopt(\Stild)$ in terms of $\opt(I)$.

\subsection{Properties of \texorpdfstring{$\round$}{round}}
\label{sec:rna:round}

\begin{definition}[Density vector]
The density vector of a \geomvecdim{$d_g$}{$d_v$} item $i$ is the vector
$v_{\Span} \defeq \left[v_0(i)/{\Span(i)}, {v_1(i)}/{\Span(i)}, \ldots, {v_{d_v}(i)}/{\Span(i)}\right]$.
Recall that $v_0(i) \defeq \vol(i)$
and note that $\|v_{\Span}\|_{\infty} = 1$.
\end{definition}

The subroutine $\round(I)$ returns a set of pairs of the form $(\Itild, D)$.
\textbf{Condition C1} is defined as the following constraints over each pair $(\Itild, D)$:
\begin{itemize}
\item {\bf C1.1.} {\em Small discard:} $D \subseteq I$ and $\Span(D) \le \eps\Span(I)$.
\item {\bf C1.2.} {\em Bijection from $I-D$ to $\Itild$:}
    Each item in $\Itild$ is obtained by modifying an item in $I-D$.
    Let $\pi$ be the bijection from $I-D$ to $\Itild$ that captures this relation.
\item {\bf C1.3.} {\em Homogeneity properties:}
    $\round$ partitions items in $\Itild$ into a constant number of classes:
    $\Ktild_1, \Ktild_2, \ldots, \Ktild_q$. These classes should satisfy the following properties,
    which we call \emph{homogeneity} properties:
    \begin{itemize}
    \item All items in a class have the same density vector.
    \item For each class $\Ktild_j$, we decide the set $X$ of axes perpendicular to which we can
        slice items in $\Ktild_j$. If items in a class $\Ktild_j$ are not allowed to be
        sliced perpendicular to dimension $k$, then all items in that class have
        the same length along dimension $k$.
        (For example, if $d_g=2$ and vertical cuts are forbidden, then all items
        have the same width.)
    \end{itemize}
\item {\bf C1.4.} {\em Bounded expansion:}
    Let $C$ be any \config{} of $I$ and $\Ktild$ be any one of the classes of $\Itild$.
    Let $\Ctild \defeq \{\pi(i): i \in C - D\}$. Then we need to prove that
    $\Span(\Ktild \cap \Ctild) \le c_{\max}$ for some constant $c_{\max}$.
    Let us call this result `bounded expansion lemma'.
\end{itemize}

Intuitively, the homogeneity properties allow us to replace (a slice of) an item
in a fractional packing by slices of other items of the same class.
Thus, while trying to get a fractional packing, we can focus on the item classes,
which are constant in number, instead of focusing on the $n$ items.
Intuitively, the bounded expansion lemma ensures that we do not round up items too much.

\textbf{Condition C2} (also called \emph{structural theorem}):
For some constant $\rho > 0$ and for some $(\Itild, D) \in \round(I)$,
$\fsopt(\Itild) \le \rho\opt(I) + O(1)$.

Intuitively, the structural theorem says that allowing slicing as per $\round$
and imposing a structure on the packing
does not increase the minimum number of bins by too much.
The AAR of $\rnaPack$ increases with $\rho$, so we want $\rho$ to be small.

\subsection{\texorpdfstring{$\complexPack$}{complex-pack}}
\label{sec:rna:complex-pack}

\textbf{Condition C3}:
For some constant $\alpha > 0$ and for any $(\Itild, D) \in \round(I)$
and any $\Xtild \subseteq \Itild$, $\complexPack(\Xtild)$ packs $\Xtild$ into at most
$\alpha\fsopt(\Xtild) + O(1)$ bins.

Intuitively, condition C3 says that we can find a packing that is
close to the optimal fractional structured packing.
The AAR of $\rnaPack$ increases with $\alpha$, so we want $\alpha$ to be small.

\subsection{\texorpdfstring{$\unround$}{unround}}
\label{sec:rna:unround}

\textbf{Condition C4}:
For some constant $\gamma > 0$, if $\complexPack(\Stild)$ outputs a packing
of $\Stild$ into $m$ bins, then $\unround$ modifies that packing
to get a packing of $S-D$ into $\gamma m + O(1)$ bins.

Intuitively, condition C4 says that unrounding does not increase the number of bins by too much.
The AAR of $\rnaPack$ increases with $\gamma$, so a small $\gamma$ is desirable.
If $\round$ only increases the dimensions of items,
then unrounding is trivial and $\gamma = 1$.

\subsection{AAR of R\&A}
\label{sec:rna:aar}

Recall that $\simplePack$ is a $2b(d_v+1)$-approximation algorithm for
\geomvec{$d_g$}{$d_v$} BP (see \cref{sec:span-pack}), where
$b \defeq 3$ when $d_g=2$, $b \defeq 9$ when $d_g = 3$,
and $b \defeq 4^{d_g}+2^{d_g}-1$ when $d_g > 3$.

\begin{restatable}{lemma}{rthmFoptConc}
\label{thm:fopt-conc}
%Let $\epsLP \in (0, 1)$ be a constant.
Let $\Stild$ be as computed by $\rnaPack(I, \beta, \eps)$. Then\\
$\fsopt(\Stild) \le \fsopt(\Itild)/\beta + 2b\mu\epsLP\opt(I) + O(1/\epsLP^2)$
with high probability.
\end{restatable}

\Cref{thm:fopt-conc} (proved in \cref{sec:rna-extra})
is the key ingredient in the analysis of R\&A.
Our proof of \cref{thm:fopt-conc} is inspired by the analysis in \cite{khan-thesis}.
We prove it by analyzing the \emph{fractional structured} \config{} LP of $\Itild$.
By the homogeneity property (C1.3), the number of constraints in this LP is a constant.
So by rank lemma (\cref{thm:rank-lemma-corr}) and downward closure
property (see \cref{thm:fcip-eq-fopt} in Appendix), we can show
$\fsopt$ to be approximately equal to the optimal solution to the LP.
We then harness the randomness of $\Stild$ and bounded expansion
property (C1.4) to use the
independent bounded difference inequality \cite{mcdiarmid1989method}
to compare the optimal LP objectives of $\Stild$ and $\Itild$.

\begin{theorem}
\label{thm:rna-pack}
The number of bins used by $\rnaPack(I, \beta, \eps)$ is at most
\[ \left( (\ln\beta)\mu + \frac{\gamma\alpha\rho}{\beta}
    + 2b(d_v+1+\gamma\alpha\mu)\eps \right) \opt(I) + O(1/\eps^2). \]
\end{theorem}
\begin{proof}
Let $J_{\LP}$ be the set of bins packed in the {\em randomized rounding of
configuration LP} step
(see line \ref{alg-line:rna-pack:rround-pack} in \cref{algo:rna-pack}
in \cref{sec:rna-extra}),
$J_D$ be the set of bins used to pack the discarded items $D \cap S$,
$J$ be the set of bins used to pack the rest of the items $S - D$,
and $\Jtild$ be the set of bins used by $\complexPack$ to pack items in $\Stild$.

Then $|J_{\LP}| \le T = \smallceil{(\ln\beta)\smallnorm{\xhat}_1} \le (\ln\beta)\mu\opt(I) + O(1)$.

Now, we have $|J_D| \le b\ceil{2\Span(D)} \le 2b\eps\Span(I) + b \le 2b(d_v+1)\eps\opt(I) + b$.
The first inequality follows from the property of $\simplePack$, the second follows from
C1.1 (Small Discard) and the last follows from \cref{thm:span-lb-opt}.

\begin{align*}
\text{Finally, } |J| &\le \gamma|\Jtild| + O(1)  \tag{property of $\unround$ (C4)}
\\ &\le \gamma\alpha\fsopt(\Stild) + O(1)  \tag{property of $\complexPack$ (C3)}
\\ &\le \gamma\alpha\left( {\fsopt(\Itild)}/{\beta} + 2b\mu\eps\opt(I) \right) + O(1/\eps^2)
    \tag{by \cref{thm:fopt-conc}}
\\ &\le \gamma\alpha\left(\rho/\beta + 2b\mu\eps\right)\opt(I) + O(1/\eps^2).
\end{align*}
Here, the last inequality follows from the structural theorem (C2),
which says that $\exists (\Itild, D) \in \round(I)$ such that
$\fsopt(\Itild) \le \rho\opt(I) + O(1)$.
Hence, the total number of bins is at most
\begin{align*}
& |J_{\LP}| + |J_D| + |J|
\le \left( (\ln\beta)\mu + \frac{\gamma\alpha\rho}{\beta}
    + 2b(d_v+1+\gamma\alpha\mu)\eps \right)\opt(I) + O(1/\eps^2).
\qedhere \end{align*}
\end{proof}

%When $\beta = 1$, we get $\alpha\gamma\rho$ as the asymptotic approximation factor.

The AAR of $\rnaPack(I)$ is roughly $\mu\ln\beta + \gamma\alpha\rho/\beta$.
This is minimized for $\beta = \gamma\alpha\rho/\mu$ and the minimum value is
$\mu\left(1 + \ln\left(\alpha\gamma\rho/\mu\right)\right)$.
As we require $\beta \ge 1$, we get this AAR only when $\gamma\alpha\rho \ge \mu$.
If $\mu \ge \gamma\alpha\rho$, the optimal $\beta$ is 1 and the AAR is roughly $\gamma\alpha\rho$.

\subsection{Example: \texorpdfstring{$\simplePack$}{simple-pack}}
\label{sec:rna:simple-pack}

We will show how to use $\simplePack$ with the R\&A framework.
Recall that $\simplePack$ is a $2b(d_v+1)$-approximation algorithm for \geomvec{$d_g$}{$d_v$} BP
(see \cref{sec:span-pack}), where
$b \defeq 3$ when $d_g=2$, $b \defeq 9$ when $d_g = 3$,
and $b \defeq 4^{d_g}+2^{d_g}-1$ when $d_g > 3$.
Using the R\&A framework on $\simplePack$ will
improve its AAR from $2b(d_v+1)$ to $b(1+\ln(2(d_v+1))) + O(\eps)$.
To do this, we need to show how to implement $\cLPsolve$, $\round$, $\complexPack$ and $\unround$.

\begin{enumerate}
\item $\cLPsolve(I)$:
Using the knapsack algorithm of \cref{sec:simple-gvks} and the LP algorithm of \cite{eku-pst},
we get a $b(1+\eps)$-approximate solution to $\cLP(I)$. Therefore, $\mu = b(1+\eps)$.

\item $\round(I)$:
returns just one pair: $(\Itild, \{\})$, where $\Itild \defeq \{\pi(i): i \in I\}$
and $\pi(i)$ is an item having height ($d_g\Th$ geometric dimension) equal to $\Span(i)$,
all other geometric dimensions equal to 1, and all vector dimensions equal to $\Span(i)$.
There is just one class in $\Itild$ and all items are allowed to be sliced
perpendicular to the height, so the homogeneity properties are satisfied.
Also, $c_{\max} = d_v+1$ by \cref{thm:span-lb-opt}.

\item \textbf{Structural theorem}:
We take structured packing to be the set of all possible packings
(i.e., $\Scal$ is the set of all possible fractional configurations).
Then $\fsopt(\Itild) = \ceil{\Span(I)} \le (d_v+1)\opt(I)$. Therefore, $\rho = d_v+1$.

\item $\complexPack(\Stild)$:
We can treat $\Stild$ as the classical bin packing instance $\{\Span(i): i \in S\}$
and pack it using Next-Fit. Therefore, by \cref{thm:next-fit}, we get
$|\complexPack(\Stild)| \le \ceil{2\Span(S)} \le 2\ceil{\Span(S)} = 2\fsopt(\Stild)$.
Therefore, $\alpha = 2$.

\item $\unround(\Jtild)$:
For each bin in $\Jtild$, we can pack the corresponding unrounded items into $b$ bins
(using Steinberg's algorithm or \cite{diedrich2008approximation} or $\fhk[4]$).
Therefore, $\gamma = b$.
\end{enumerate}

Therefore, we get an AAR of
$\mu(1+\ln(\gamma\alpha\rho/\mu)) + O(\eps) \approx b(1+\ln(2(d_v+1))) + O(\eps)$.

For $d_g = 2$, we can slightly improve the AAR by using the
$(2+\eps)$-approximation algorithm of \cite{aco-gvks} for \geomvec{2}{$d_v$} KS.
This gives us an AAR of $2(1 + \ln(3(d_v+1))) + O(\eps)$.
This is better than the AAR of $\betterSimplePack$ for $d_v \ge 3$.

The above example is presented only to illustrate an easy use of the R\&A framework.
It doesn't exploit the full power of the R\&A framework.
The algorithm $\cbPack$, which we outline in \cref{chap:gv-rbbp},
uses more sophisticated subroutines $\round$, $\complexPack$ and $\unround$,
and uses a more intricate definition of fractional structured packing
to get an even better AAR of $2(1+\ln(\frac{d+4}{2})) + \eps$
(improves to $2(1+\ln(19/12))+\eps$ $\approx 2.919+\eps$ for $d=1$).
