\section{\DFF{} Transform}
\label{sec:dff-trn}

In this section, we prove \cref{thm:dff-pack}.

\begin{lemma}
\label{thm:dff-pack-1}
Let $I$ be a set of $d$D items that can be packed into a bin. Let $g$ be a \dff{}.
Let $q \in [d]$. For $i \in I$, define $g(i)$ to be the item $\ihat$ for which
$\ell_j(\ihat) \defeq \ell_j(i)$ when $j \neq q$ and $\ell_q(\ihat) \defeq g(\ell_q(i))$.
Then the items $\{g(i): i \in I\}$ can be packed into a $d$D bin (without rotating the items).
\end{lemma}
Bansal, Caprara and Sviridenko \cite{rna} give a brief proof sketch for $d=2$,
based on which we provide a full proof below.
\begin{proof}
Any $d$D cuboid can be represented as the Cartesian product of
$d$ closed intervals on the real line. Let the bin be $[0, 1]^d$.
Any item $i \in I$ can be written as $\prod_{j=1}^d [v_j(i), v_j(i) + \ell_j(i)]$.
Here $v_j(i)$ is called the \emph{position} of item $i$ in dimension $j$.
Since each item $i$ lies completely inside the bin, $0 \le v_j(i) < v_j(i) + \ell_j(i) \le 1$.
Two cuboids $A$ and $B$ are said to overlap if their intersection has positive volume.
Since $I$ is a valid packing, no two items overlap.

Assume \wLoG{} that $q = d$.
Let $\proj(i)$ be the projection of item $i$ onto the hyperplane
perpendicular to the $d\Th$ dimension.
This hyperplane can be thought of as the \emph{base} of the bin.

We will now show that for each item $i$, we can change $\ell_d(i)$ to $g(\ell_d(i))$
and change $v_d(i)$ so that the items continue to fit in the bin.
But to define what the new value of $v_d(i)$ would be,
we need to first introduce some notation.

For two items $i_1$ and $i_2$, we say that $i_1 \prec i_2$
($i_1$ is a predecessor of $i_2$)
iff $v_d(i_1) < v_d(i_2)$ and $\proj(i_1)$ overlaps $\proj(i_2)$.
Call a sequence $[i_0, i_1, \ldots, i_{m-1}]$ of items a \emph{chain}
iff $i_{m-1} \prec i_{m-2} \prec \ldots \prec i_0$. $i_0$ is called the head of this chain.
The \emph{augmented height} of a chain $S$ is defined to be
$\sum_{i \in S} g(\ell_d(i))$.
For each item $i$, we wish to find the chain headed at $i$
with the maximum augmented height.

For an item $i$, define
\[ \level(i) \defeq \begin{cases} 0 & \textrm{if } i \textrm{ has no predecessors}
\\ {\displaystyle 1 + \max_{i' \prec i} \level(i')} & \textrm{otherwise} \end{cases}. \]
Since $\prec$ is anti-symmetric, $\level$ is well-defined.
Define $\pi$ and $u$ as
\begin{align*}
u(i) &\defeq g(\ell_d(i)) + \begin{cases} 0 & \textrm{if } \level(i) = 0
    \\ u(\pi(i)) & \textrm{otherwise}\end{cases}
& \pi(i) &\defeq \begin{cases} \Null & \textrm{if } \level(i) = 0
\\ {\displaystyle \argmax_{i' \prec i} u(i')} & \textrm{otherwise} \end{cases}.
\end{align*}
In the definition of $\pi$, ties can be broken arbitrarily for $\argmax$.
$i' \prec i$ implies $\level(i') < \level(i)$, so $\level(\pi(i)) < \level(i)$.
This ensures that the definitions of $\pi$ and $u$ are not mutually circular.

We can prove, by inducting on $\level(i)$, that
$\Pi(i) \defeq [i, \pi(i), \pi(\pi(i)), \ldots]$ is the chain headed at $i$ with
the maximum augmented height, and that the augmented height of $\Pi(i)$ is $u(i)$.

\begin{transformation}
\label{trn:ld-and-vd}
For each item $i \in I$, change $\ell_d(i)$ to $g(\ell_d(i))$
and change $v_d(i)$ to $v_d'(i) \defeq u(i) - g(\ell_d(i))$.
\end{transformation}

We need to prove that \cref{trn:ld-and-vd} produces a valid packing,
i.e. items don't overlap and all items lie completely inside the bin $[0, 1]^d$.

Let $i_1$ and $i_2$ be any two items. We will prove that they don't overlap in the new packing.
If $\proj(i_1)$ and $\proj(i_2)$ don't overlap, then $i_1$ and $i_2$ don't overlap
and we are done, so assume $\proj(i_1)$ and $\proj(i_2)$ overlap.
Assume \wLoG{} that $i_1 \prec i_2$. Then $\level(i_2) \ge 1$ and
\[ v_d'(i_2) = u(i_2) - g(\ell_d(i_2)) = \max_{i' \prec i_2} u(i')
\ge u(i_1) = v_d'(i_1) + g(\ell_d(i_1)). \]
Therefore, $i_1$ and $i_2$ don't overlap in the new packing.

After \cref{trn:ld-and-vd}, item $i$ lies completely inside the bin
iff $v_d'(i) + g(\ell_d(i)) = u(i) \le 1$. Let $i_0 \defeq i$ and
$\Pi(i) = [i_0, i_1, i_2, \ldots, i_{m-1}]$.
Then $u(i) = \sum_{j=0}^{m-1} g(\ell_d(i_j))$ and for all $j \in [m-1], i_j \prec i_{j-1}$.
Since $i_j$ and $i_{j-1}$ don't overlap in the original packing,
but $\proj(i_j)$ and $\proj(i_{j-1})$ overlap,
we get $v_d(i_{j-1}) \ge v_d(i_j) + \ell_d(i_j)$. Therefore,
\begin{align*}
\sum_{j=0}^{m-1} \ell_d(i_j)
&\le \ell_d(i) + \sum_{j=1}^{m-1} (v_d(i_{j-1}) - v_d(i_j))
\tag{since $v_d(i_{j-1}) \ge v_d(i_j) + \ell_d(i_j)$}
\\ &= \ell_d(i) + v_d(i) - v_d(i_{m-1})
\le 1.
\tag{$\because$ in the original packing, $i$ lies in the bin}
\end{align*}
Since $g$ is a \dff{} and $\sum_{j=0}^{m-1} \ell_d(i_j) \le 1$,
we get $u(i) = \sum_{j=0}^{m-1} g(\ell_d(i_j)) \le 1$.
Therefore, the packing obtained by \cref{trn:ld-and-vd} is valid.
So $\{g(i): i \in I\}$ can be packed into a bin.
\end{proof}

\rthmDffPack*
\begin{proof}[Proof]
Apply \cref{thm:dff-pack-1} multiple times, with $q$ ranging from $1$ to $d$.
\end{proof}
