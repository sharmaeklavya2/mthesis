\section{Important Ideas from the \texorpdfstring{$\hdhk$}{HDHk} Algorithm}
\label{sec:hdhk-prelims}

In this section, we will describe some important ideas behind the $\hdhk$ algorithm
for $d$BP by Caprara~\cite{caprara2008}.
These ideas are the building blocks for our algorithms for $d$MCBP.

\subsection{\DFF{}s}

Fekete and Schepers~\cite{fekete2004} present a useful approach for
obtaining lower bounds on the optimal solution to bin packing problems.
Their approach is based on \emph{\dff{}s}.

\begin{definition}
$g: [0, 1] \mapsto [0, 1]$ is a weighting function iff
for all $m \in \mathbb{Z}_{>0}$ and $x \in [0, 1]^m$,
\[ \sum_{i=1}^m x_i \le 1 \implies \sum_{i=1}^m g(x_i) \le 1 \]
(Weighting functions are also called \emph{dual feasible functions} (DFFs)).
\end{definition}

\begin{restatable}{theorem}{rthmDffPack}
\label{thm:dff-pack}
Let $I$ be a set of $d$D items that can be packed into a bin.
Let $g_1, g_2, \ldots, g_d$ be \dff{}s.
For $i \in I$, define $g(i)$ as the item whose length is $g_j(\ell_j(i))$ in the $j\Th$ dimension.
Then $\{g(i): i \in I\}$ can be packed into a $d$D bin (without rotating the items).
\end{restatable}
\Cref{thm:dff-pack} is proved in \cref{sec:dff-trn}.
%\cite{caprara2008} refers to Lemma 2 in \cite{fekete2004},
%but that just mentions a special case, and that too without proof.

\subsection{The Harmonic Function}
\label{sec:hdhk-prelims:harmonic}

To obtain a lower-bound on $\optdbp(I)$ using \cref{thm:dff-pack},
Caprara~\cite{caprara2008} defined a function $f_k$.
For an integer constant $k \ge 3$, $f_k: [0, 1] \mapsto [0, 1]$ is defined as
\[ f_k(x) \defeq \begin{cases}
\frac{1}{q} & x \in \left(\left.\frac{1}{q+1}, \frac{1}{q}\right]\right. \textrm{ for } q \in [k-1]
\\ \frac{k}{k-2}x & x \le \frac{1}{k} \end{cases} \]
$f_k$ was originally defined and studied by Lee and Lee~\cite{leelee} for their
online algorithm for 1BP, except that they used $k/(k-1)$ instead of $k/(k-2)$.
Define $\type_k: [0, 1] \mapsto [k]$ as
\[ \type_k(x) \defeq \begin{cases}
q & x \in \left(\left.\frac{1}{q+1}, \frac{1}{q}\right]\right. \textrm{ for } q \in [k-1]
\\ k & x \le \frac{1}{k} \end{cases} \]

Define $T_k$ to be the smallest positive constant such that
$H_k(x) \defeq f_k(x)/T_k$ is a \dff{}. We call $H_k$ the \emph{harmonic \dff}.
We can efficiently compute $T_k$ as a function of $k$ using ideas from \cite{leelee,eku-harmonic}.
\Cref{table:tk-values} lists the values of $T_k$ for the first few $k$.
It can also be proven that $T_k$ is a decreasing function of $k$
and $T_{\infty} \defeq \lim_{k \to \infty} T_k \approx 1.6910302$.

\begin{table}[!ht]
\centering
\begin{tabular}{|c|c|c|c|c|c|c|}
\hline
$k$ & 3 & 4 & 5 & 6 & 7 & $\infty$ \\
\hline
$T_k$ & 3 & 2 & $11/6 = 1.8\overline{3}$ & $7/4 = 1.75$
& $26/15 = 1.7\overline{3}$ & $\approx 1.6910302$ \\
\hline
\end{tabular}
\caption{Values of $T_k$.}
\label{table:tk-values}
\end{table}

For a $d$D cuboid $i$, define $f_k(i)$ to be the cuboid
whose length is $f_k(\ell_j(i))$ in the $j\Th$ dimension.
For a set $I$ of $d$D cuboids, let $f_k(I) \defeq \{f_k(i): i \in I\}$.
Similarly define $H_k(i)$ and $H_k(I)$.
Define $\type(i)$ to be a $d$-dimensional vector whose $j\Th$ component is $\type_k(\ell_j(i))$.
Note that there can be at most $k^d$ different values of $\type(i)$.
Sometimes, for the sake of convenience, we may express $\type(i)$ as an integer in $[k^d]$.

\begin{theorem}
\label{thm:fvol-bp}
For a set of $I$ of $d$D items, $\vol(f_k(I)) \le T_k^d \optdbp(I)$.
\end{theorem}
\begin{proof}
Let $m \defeq \optdbp(I)$. Let $J_j$ be the items in the
$j\Th$ bin in the optimal bin packing of $I$.
By \cref{thm:dff-pack} and because $H_k$ is a \dff{},
% \thmdep{thm:h-dff}{thm:fvol-bp}
$H_k(J_j)$ fits in a bin. Therefore,
\[ \vol(f_k(I))
= \sum_{j=1}^m T_k^d \vol(H_k(J_j))
\le \sum_{j=1}^m T_k^d
= T_k^d \optdbp(I)
\qedhere \]
\end{proof}

\subsection{The \texorpdfstring{$\hdhkunit$}{HDH-unit-pack} Subroutine}
\label{sec:hdhk-prelims:hdhkunit}

From the $\hdhk$ algorithm by Caprara~\cite{caprara2008},
we extracted out a useful subroutine, which we call $\hdhkunit$,
that satisfies the following useful property:
\begin{property}
\label{prop:hdhkunit}
The algorithm $\hdhkunit^{[t]}(I)$ takes a \emph{sequence} $I$ of $d$D items such that
all items have type $t$ and $\vol(f_k(I-\{\last(I)\})) < 1$
(here $\last(I)$ is the last item in sequence $I$).
It returns a packing of $I$ into a single $d$D bin
in $O(n\log n)$ time, where $n \defeq |I|$.
\end{property}

The design of $\hdhkunit$ and its correctness can be inferred from
Lemma 4.1 in \cite{caprara2008}.
We use $\hdhkunit$ as a black-box subroutine in our algorithms,
i.e., we only rely on \cref{prop:hdhkunit};
we don't need to know anything else about $\hdhkunit$.
Nevertheless, for the sake of completeness, in \cref{sec:hdhkunit},
we give a complete description of $\hdhkunit$ and prove its correctness.
