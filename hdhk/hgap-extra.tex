\section{Details of the \texorpdfstring{$\hgapk$}{HGaP} Algorithm}
\label{sec:hgap-extra}

This section gives details of the subroutines used by $\hgapk$.
It also proves the theorems claimed in \cref{sec:hgap}.

\subsection{Preliminaries}

\begin{definition}
\label{defn:pred}
Let $I_1$ and $I_2$ be sets of 1D items.
Then $I_1$ is defined to be a predecessor of $I_2$ (denoted as $I_1 \preceq I_2$)
iff there exists a one-to-one mapping $\pi: I_1 \mapsto I_2$ such that
$\forall i \in I_1, i \le \pi(i)$.
\end{definition}
\begin{observation}
\label{obs:pred-pack}
Let $I_1 \preceq I_2$ where $\pi$ is the corresponding mapping.
Then we can obtain a packing of $I_1$ from a packing of $I_2$,
by packing each item $i \in I_1$ in the place of $\pi(i)$.
Hence, $\opt(I_1) \le \opt(I_2)$.
\end{observation}

\begin{definition}[Canonical shelving]
\label{defn:hgap:can-shelv}
Let $I$ be a set of rectangles.
Order the items in $I$ in non-increasing order of height
(break ties arbitrarily but deterministically)
and greedily pack them into tight shelves,
slicing items using vertical cuts if necessary.
The set of shelves thus obtained is called the \emph{canonical shelving} of $I$,
and is denoted by $\canShelv(I)$.
(The canonical shelving is unique because ties are broken deterministically.)
\end{definition}

See \cref{fig:can-shelv-example} for an example of canonical shelving.

Suppose a set $I$ of rectangular items is packed into a set $J$ of shelves.
Then we can interpret $J$ as a 1BP instance where
the height of each shelf is the size of the corresponding 1D item.

\begin{figure}[htb]
\centering
\begin{tikzpicture}[
item/.style = {draw,fill={black!15}},
bin/.style={draw,thick},
myarrow/.style={->,>={Stealth},thick},
mybrace/.style = {decoration={brace,mirror,raise=1pt,amplitude=5pt},semithick,decorate},
]
\definecolor{myblue}{HTML}{4768CC}
\definecolor{myred}{HTML}{CC4747}
\definecolor{mycyan}{HTML}{47CCCC}
\definecolor{mygreen}{HTML}{8ACC47}
\definecolor{myyellow}{HTML}{CCAB47}
\begin{scope}
\path[item,fill=myblue] (0,0) rectangle +(1.2,1.5) node[text=white,pos=0.5] {1};
\node[anchor=north] at (0.6,0) {0.3};
\path[item,fill=myred] (1.4,0) rectangle +(1.6,1.3) node[text=white,pos=0.5] {2};
\node[anchor=north] at (2.2,0) {0.4};
\path[item,fill=mycyan] (3.2,0) rectangle +(1.6,1.2) node[pos=0.5] {3};
\node[anchor=north] at (4,0) {0.4};
\path[item,fill=myyellow] (5,0) rectangle +(2,1) node[pos=0.5] {4};
\node[anchor=north] at (6,0) {0.5};
\path[item,fill=mygreen] (0,3) rectangle +(3.6,0.8) node[pos=0.5] {5};
\node[anchor=north] at (1.8,3) {0.9};
\path[item] (3.8,3) rectangle +(1,0.6) node[pos=0.5] {6};
\node[anchor=north] at (4.3,3) {0.25};
\end{scope}
\draw[myarrow] (7.5,2.1) -- (9,2.1);
\begin{scope}[xshift=10cm]
\begin{scope}
\path[item,fill=myblue] (0.0,0) rectangle +(1.2,1.5) node[text=white,pos=0.5] {1};
\path[item,fill=myred] (1.2,0) rectangle +(1.6,1.3) node[text=white,pos=0.5] {2};
\path[item,fill=mycyan] (2.8,0) rectangle +(1.2,1.2) node[pos=0.5] {3};
\path[bin] (0,0) rectangle +(4,1.5);
\draw[mybrace] (2.8,0) -- node[below=5pt] {0.3} (4,0);
\end{scope}
\begin{scope}[yshift=1.9cm]
\path[item,fill=mycyan] (0,0) rectangle +(0.4,1.2) node[pos=0.5] {3};
\path[item,fill=myyellow] (0.4,0) rectangle +(2,1) node[pos=0.5] {4};
\path[item,fill=mygreen] (2.4,0) rectangle +(1.6,0.8) node[pos=0.5] {5};
\path[bin] (0,0) rectangle +(4,1.2);
\end{scope}
\begin{scope}[yshift=3.5cm]
\path[item,fill=mygreen] (0,0) rectangle +(2,0.8) node[pos=0.5] {5};
\path[item] (2,0) rectangle +(1,0.6) node[pos=0.5] {6};
\path[bin] (0,0) rectangle +(4,0.8);
\draw[mybrace] (2,0.8) -- node[above=5pt] {0.5} (0,0.8);
\end{scope}
\end{scope}
\end{tikzpicture}

\caption[Six items and their canonical shelving into three tight shelves of width 1.]%
{Six items and their canonical shelving into three tight shelves of width 1.
The items are numbered by decreasing order of height.
Each item has its width mentioned below it.
Item 3 was sliced into two items of widths 0.3 and 0.1.
Item 5 was sliced into two items of widths 0.4 and 0.5.}
\label{fig:can-shelv-example}
\end{figure}

We will now prove that the canonical shelving is optimal,
i.e., any shelf-based bin packing of items can be obtained
by first computing the canonical shelving and then
packing the shelves into bins like a 1BP instance.

\begin{lemma}
\label{thm:hgap:can-shelv-pred}
Let $I$ be a set of rectangles packed inside shelves $J$.
Let $J^* \defeq \canShelv(I)$. Then $J^* \preceq J$.
\end{lemma}
\begin{proof}
We say that a shelf is full if the total width of items in a shelf is 1.
Arrange the shelves $J$ in non-increasing order of height,
and arrange the items $I$ in non-increasing order of height.
Then try to pack $I$ into $J$ using the following greedy algorithm:
For each item $i$, pack the largest possible slice of $i$ into
the first non-full shelf and pack the remaining slice (if any) in the next shelf.
If this greedy algorithm succeeds, then within each shelf of $J$,
there is a shelf of $J^*$, so $J^* \preceq J$.
We will now prove that this greedy algorithm always succeeds.

For the sake of proof by contradiction, assume that the greedy algorithm failed, i.e.,
for an item (or slice) $i$ there was a non-full shelf $S$ but $h(i) > h(S)$.
Let $I'$ be the items (and slices) packed before $i$ and $J'$ be the shelves before $S$.
Therefore, $w(I') = |J'|$.

All items in $I'$ have height at least $h(i)$,
so all shelves in $J'$ have height at least $h(i)$.
All shelves after $J'$ have height less than $h(i)$.
Therefore, $J'$ is exactly the set of shelves of height at least $h(i)$.

In the packing $P$, $I' \cup \{i\}$ can only be packed into shelves of height
at least $h(i)$, so $w(I') + w(i) \le |J'|$. But this contradicts $w(I') = |J'|$.
Therefore, the greedy algorithm cannot fail.
\end{proof}

\begin{lemma}
\label{thm:nos-sum}
Consider the inequality $x_1 + x_2 + \ldots + x_n \le s$,
where for each $j \in [n]$, $x_j \in \mathbb{Z}_{\ge 0}$.
Let $N$ be the number of solutions to this inequality.
Then $N = \binom{s+n}{n} \le (s+1)^n$.
\end{lemma}
\begin{proof}
The proof of $N = \binom{s+n}{n}$ is a standard result in combinatorics.

To prove $N \le (s+1)^n$, note that we can choose each
$x_j \in \{0, 1, \ldots, s\}$ independently.
\end{proof}

\subsection{Structural Theorem}
\label{sec:hgap:struct}

Let $I$ be a set of $d$D items.
Let $\Ihat \defeq \round(I)$.
Let $\delta \in (0, 1)$ be a constant.
Let $\Ihat_L \defeq \{i \in \Ihat: h(i) > \delta\}$ and $\Ihat_S \defeq \Ihat - \Ihat_L$.
Let $J \defeq \canShelv(\Ihat_L)$. Let $m \defeq |J|$, i.e., $J$ contains $m$ shelves.
We can interpret $\Ihat_S$ as a single sliceable 1D item of size $a(\Ihat_S)$.

We will show the existence of a structured $\delta$-fractional packing of $\Ihat$
into at most $T_k^{d-1}(1+\delta)\optdbp(I) + \ceildeltsq + 1 + \delta$ bins.
This would prove \cref{thm:hgap:struct}.

\begin{definition}[Linear grouping~\cite{bp-aptas}]
\label{defn:hgap:lingroup}
Arrange the 1D items $J$ in non-increasing order of size and number them from 1 to $m$.
Let $q \defeq \floor{\delta\size(J)} + 1$.
Let $J_1$ be the first $q$ items, $J_2$ be the next $q$ items, and so on.
$J_j$ is called the $j\Th$ \emph{linear group} of $J$.
This gives us $t \defeq \ceil{m/q}$ linear groups.
Note that the last group, $J_t$, may have less than $q$ items.

Let $h_j$ be the size of the first item in $J_j$. Let $h_{t+1} \defeq 0$.
For $j \in [t-1]$, let $J_j\lo$ be the items obtained by
decreasing the height of items in $J_j$ to $h_{j+1}$.
For $j \in [t]$, let $J_j\hi$ be the items obtained by
increasing the height of items in $J_j$ to $h_j$.

Let ${J\lo \defeq \bigcup_{j=1}^{t-1} J_j\lo}$
and ${J\hi \defeq \bigcup_{j=1}^t J_j\hi}$.
We call $J\lo$ a down-rounding of $J$ and $J\hi$ an up-rounding of $J$.
\end{definition}

\begin{lemma}
\label{thm:hgap:n-pivots}
%\thmdep{defn:hgap:lingroup}{thm:hgap:n-pivots}
$t \le \ceildeltsq$.
\end{lemma}
\begin{proof}
Since each shelf in $J$ has height more than $\delta$, $\size(J) > |J|\delta$.
\[ t \defeq \ceil{\frac{|J|}{\floor{\delta\size(J)}+1}}
\le \ceil{\frac{\size(J)/\delta}{\delta\size(J)}}
= \ceil{\frac{1}{\delta^2}} \qedhere \]
\end{proof}

\begin{lemma}
\label{thm:hgap:pred-chain}
%\thmdep{defn:hgap:lingroup}{thm:hgap:pred-chain}
$J\lo \preceq J \preceq J\hi \preceq J\lo \cup J_1\hi$.
\end{lemma}
\begin{proof}
It is trivial to see that $J\lo \preceq J \preceq J\hi$.
For $j \in [t-1]$, all (1D) items in both $J_j\lo$ and $J_{j+1}\hi$ have height $h_{j+1}$,
and $|J_{j+1}| \le q = |J_j|$. Therefore, $J_{j+1}\hi \preceq J_j\lo$ and hence
\[ J\hi = J_1\hi \cup \bigcup_{j=1}^{t-1} J_{j+1}\hi
\preceq J_1\hi \cup \bigcup_{j=1}^{t-1} J_j\lo = J_1\hi \cup J\lo \qedhere \]
\end{proof}

\begin{lemma}
\label{thm:hgap:can-shelv-size}
$\size(J) < 1 + a(\Ihat_L)$.
\end{lemma}
\begin{proof}
In the canonical shelving of $\Ihat_L$, let $S_j$ be the $j\Th$ shelf.
Let $h(S_j)$ be the height of $S_j$.
Let $a(S_j)$ be the total area of the items in $S_j$.
Since the shelves are tight, items in $S_j$ have height at least $h(S_{j+1})$.
So, $a(S_j) \ge h(S_{j+1})$ and
\[ \size(J) = \sum_{j=1}^{|J|} h(S_j) \le 1 + \sum_{j=1}^{|J|-1} h(S_{j+1})
\le 1 + \sum_{j=1}^{|J|-1} a(S_j) < 1 + a(\Ihat_L) \qedhere \]
\end{proof}

\begin{lemma}
\label{thm:hgap:sopt-le-optlo}
%\thmdep{defn:hgap:lingroup}{thm:hgap:sopt-le-optlo}
$\sopt_{\delta}(\Ihat) < \opt(J\lo \cup \Ihat_S) + \delta a(\Ihat_L) + (1 + \delta)$.
\end{lemma}
\begin{proof}
By the definition of $\canShelv$, $\Ihat_L$ can be packed into $J$.
By \thmdepcref{thm:hgap:pred-chain}{}, $J \preceq J\hi$, so $\Ihat_L$ can be packed into $J\hi$.
By \thmdepcref{thm:hgap:n-pivots}{}, the number of distinct sizes in $J\hi$
is at most $\ceildeltsq$.
So, the optimal 1D bin packing of $J\hi \cup \Ihat_S$ will
give us a structured $\delta$-fractional bin packing of $\Ihat$.
Hence, $\sopt_{\delta}(\Ihat) \le \opt(J\hi \cup \Ihat_S)$.

By \thmdepcref{thm:hgap:pred-chain,obs:pred-pack}{} we get
\[ \opt(J\hi \cup \Ihat_S) \le \opt(J\lo \cup J_1\hi \cup \Ihat_S)
\le \opt(J\lo \cup \Ihat_S) + \opt(J_1\hi) \]
By \thmdepcref{thm:hgap:can-shelv-size}{},
\[ \opt(J_1\hi) \le |J_1\hi| \le q \le 1 + \delta\size(J)
< 1 + \delta(1 + a(\Ihat_L)) \qedhere \]
\end{proof}

\subsubsection{LP for Packing \texorpdfstring{$J\lo \cup \Ihat_S$}{J\^{}lo + I\^{}\_S}}

We will formulate an integer linear program for bin packing $J\lo \cup \Ihat_S$.

Let $C \in \mathbb{Z}_{\ge 0}^{t-1}$ such that $h_C \defeq \sum_{j=1}^{t-1} C_jh_{j+1} \le 1$.
Then $C$ is called a \config{}.
$C$ represents a set of 1D items that can be packed into a bin
and where $C_j$ items are from $J_j\lo$.
Let $\Ccal$ be the set of all \config{}s.
We can pack at most $\ceil{1/\delta}-1$ items into a bin because $h_t > \delta$.
By \cref{thm:nos-sum}, we get
$|\Ccal| \le \binom{\ceil{1/\delta}-1+t-1}{t-1} \le \ceildeltsq^{1/\delta}$.

Let $x_C$ be the number of bins packed according to \config{} $C$.
Bin packing $J\lo \cup \Ihat_S$ is equivalent to finding the optimal integer solution
to the following linear program, which we denote as $\LP(\Ihat)$.
\[ \begin{array}{*3{>{\displaystyle}l}}
\min_{x \in \mathbb{R}^{|\Ccal|}} & \sum_{C \in \Ccal} x_C
\\[1.5em] \textrm{where }
    & \sum_{C \in \Ccal} C_j x_C \ge q & \forall j \in [t-1]
\\[1.5em] & \sum_{C \in \Ccal} (1-h_C)x_C \ge a(\Ihat_S)
\\[1em] & x_C \ge 0 & \forall C \in \Ccal
\end{array} \]
Here the first set of constraints say that for each $j \in [t-1]$,
all of the $q \defeq \floor{\delta\size(J)}+1$ shelves $J\lo_j$
should be covered by the configurations in $x$.
The second constraint says that we should be able to pack $a(\Ihat_S)$
into the non-shelf space in the bins.
\begin{lemma}
\label{thm:hgap:opt-to-lp}
$\opt(J\lo \cup \Ihat_S) \le \opt(\LP(\Ihat)) + t$.
\end{lemma}
\begin{proof}
Let $x^*$ be an optimal extreme-point solution to $\LP(\Ihat)$.
By rank-lemma, $x^*$ has at most $t$ non-zero entries.
Let $\xhat$ be a vector where $\xhat_C \defeq \ceil{x_C^*}$.
Then $\xhat$ is an integral solution to $\LP(\Ihat)$ and
$\sum_C \xhat_C < t + \sum_C x_C^* = \opt(\LP(\Ihat)) + t$.
\end{proof}
%
The dual of $\LP(\Ihat)$, denoted by $\DLP(\Ihat)$, is
\begin{align*}
\max_{y \in \mathbb{R}^{t-1}, z \in \mathbb{R}}
    & a(\Ihat_S)z + q \sum_{j=1}^{t-1} y_j
\\ \textrm{ where } & \sum_{j=1}^{t-1} C_j y_j + (1-h_C)z \le 1 \quad \forall C \in \Ccal
\\ & z \ge 0 \textrm{ and } y_j \ge 0 \quad \forall j \in [t-1]
\end{align*}

\subsubsection{Weighting Function for a Feasible Solution to
\texorpdfstring{$\DLP(\Ihat)$}{DLP(I\^{})}}

We will now see how to obtain a monotonic weighting function
$\eta: [0, 1] \mapsto [0, 1]$ from a feasible solution to $\DLP(\Ihat)$.
To do this, we adapt techniques from Caprara's analysis of $\hdhk$~\cite{caprara2008}.
Such a weighting function will help us upper-bound $\opt(\LP(\Ihat))$
in terms of $\optdbp(I)$.

We first describe a transformation that helps us convert any feasible
solution of $\DLP(\Ihat)$ to a feasible solution that is \emph{monotonic}.
We then show how to obtain a weighting function from this monotonic solution.

\begin{transformation}
\label{trn:hgap:half-mono}
Let $(y, z)$ be a feasible solution to $\DLP(\Ihat)$. Let $s \in [t-1]$.
Define $y_t \defeq 0$ and $h_{t+1} \defeq 0$.
Then change $y_s$ to $\max(y_s, y_{s+1} + (h_{s+1} - h_{s+2})z)$.
\end{transformation}
\begin{lemma}
\label{thm:hgap:half-mono-feas}
Let $(y, z)$ be a feasible solution to $\DLP(\Ihat)$.
Let $(\yhat, z)$ be the new solution obtained by applying \cref{trn:hgap:half-mono}
with parameter $s \in [t-1]$. Then $(\yhat, z)$ is feasible for $\DLP(\Ihat)$.
\end{lemma}
\begin{proof}
For a \config{} $C$, let $f(C, y, z) \defeq C^Ty + (1-h_C)z$,
where $C^Ty \defeq \sum_{j=1}^{t-1} C_jy_j$.
Since $(y, z)$ is feasible for $\DLP(\Ihat)$, $f(C, y, z) \le 1$.
As per \cref{trn:hgap:half-mono},
\[ \yhat_j \defeq \begin{cases} \max(y_s, y_{s+1} + (h_{s+1} - h_{s+2})z) & j = s
\\ y_j & j \neq s \end{cases} \]
If $y_s \ge y_{s+1} + (h_{s+1} - h_{s+2})z$, then $\yhat = y$,
so $(\yhat, z)$ would be feasible for $\DLP(\Ihat)$.
So now assume that $y_s < y_{s+1} + (h_{s+1} - h_{s+2})z$.

Let $C$ be a \config{}. Define $C_t \defeq 0$. Let
\[ \Chat_j \defeq \begin{cases} 0 & j = s \\ C_s + C_{s+1} & j = s+1
\\ C_j & \textrm{otherwise} \end{cases} \]
Then, $C^T\yhat - \Chat^Ty
= C_s\yhat_s + C_{s+1}\yhat_{s+1} - \Chat_sy_s - \Chat_{s+1}y_{s+1}
= C_s(h_{s+1} - h_{s+2}) z$.

Also, $h_{\Chat} - h_C
= \Chat_sh_{s+1} + \Chat_{s+1}h_{s+2} - C_sh_{s+1} - C_{s+1}h_{s+2}
= - C_s(h_{s+1} - h_{s+2})$.

Since $h_{\Chat} \le h_C \le 1$, $\Chat$ is a \config{}.
\begin{align*}
f(C, \yhat, z) &= C^T\yhat + (1-h_C)z
\\ &= (\Chat^Ty + C_s(h_{s+1} - h_{s+2})z)
    + (1 - h_{\Chat} - C_s(h_{s+1} - h_{s+2}))z
\\ &= f(\Chat, y, z) \le 1
\end{align*}
Therefore, $(\yhat, z)$ is feasible for $\DLP(\Ihat)$.
\end{proof}

\begin{definition}
Let $(y, z)$ be a feasible solution to $\DLP(\Ihat)$. Let
\[ \yhat_j \defeq \begin{cases}
\max(y_{t-1}, zh_t) & j = t-1
\\ \max(y_j, \yhat_{j+1} + (h_{j+1} - h_{j+2})z) & j < t-1
\end{cases} \]
Then $(\yhat, z)$ is called the monotonization of $(y, z)$.
\end{definition}
\begin{lemma}
\label{thm:hgap:mono-feas}
Let $(y, z)$ be a feasible solution to $\DLP(\Ihat)$.
Let $(\yhat, z)$ be the monotonization of $(y, z)$.
Then $(\yhat, z)$ is a feasible solution to $\DLP(\Ihat)$.
\end{lemma}
\begin{proof}
$(\yhat, z)$ can be obtained by multiple applications of \cref{trn:hgap:half-mono}:
first with $s = t-1$, then with $s = t-2$, and so on till $s = 1$.
Then by \thmdepcref{thm:hgap:half-mono-feas}{}, $(\yhat, z)$ is feasible for $\DLP(\Ihat)$.
\end{proof}

Let $(y^*, z^*)$ be an optimal solution to $\DLP(\Ihat)$.
Let $(\yhat, z^*)$ be the monotonization of $(y^*, z^*)$.
Then define the function $\eta: [0, 1] \mapsto [0, 1]$ as
\[ \eta(x) \defeq \begin{cases}
\yhat_1 & \textrm{if } x \in [h_2, 1]
\\ \yhat_j & \textrm{if } x \in [h_{j+1}, h_j), \textrm{ for } 2 \le j \le t-1
\\ xz^* & \textrm{if } x < h_t
\end{cases} \]

\begin{lemma}
\label{thm:hgap:eta-dff}
$\eta$ is a monotonic \dff{}.
\end{lemma}
\begin{proof}
$\eta$ is monotonic by the definition of monotonization.

Let $X \subseteq (0, 1]$ be a finite set such that $\Sum(X) \le 1$.
Let $X_0 \defeq X \cap [0, h_t)$,
let $X_1 \defeq X \cap [h_2, 1]$
and for $2 \le j \le t-1$, let $X_j \defeq X \cap [h_{j+1}, h_j)$.
Let $C \in \mathbb{Z}^{t-1}_{\ge 0}$ such that $C_j \defeq |X_j|$.
Let $h_C \defeq \sum_{j=1}^{t-1} C_jh_{j+1}$.
\begin{align*}
1 \ge \Sum(X) &= \Sum(X_0) + \sum_{j=1}^{t-1} \Sum(X_j)
\\ &\ge \Sum(X_0) + \sum_{j=1}^{t-1} C_jh_{j+1}
\tag{for $j \ge 1$, each element in $X_j$ is at least $h_{j+1}$}
\\ &= \Sum(X_0) + h_C
\end{align*}
Since $h_C \le 1 - \Sum(X_0) \le 1$, $C$ is a \config{}. Therefore,
\begin{align*}
\sum_{x \in X} \eta(x)
&= \sum_{j=0}^{t-1} \sum_{x \in X_j} \eta(x)
= z^*\Sum(X_0) + \sum_{j=1}^{t-1} C_j\yhat_j
\tag{by definition of $\eta$}
\\ &\le (1-h_C)z^* + C^T\yhat
\tag{$h_C \le 1 - \Sum(X_0)$}
\\ &\le 1
\tag{$C$ is a \config{} and $(\yhat, z^*)$ is feasible for $\DLP(\Ihat)$
    by \thmdepcref{thm:hgap:mono-feas}{}}
\end{align*}
\end{proof}

\begin{lemma}
\label{thm:hgap:p-vs-opt}
For $i \in I$, let $p(i) \defeq \eta(h(i))w(i)$.
Then $\opt(\LP(\Ihat)) \le p(I) \le T_k^{d-1}\optdbp(I)$.
\end{lemma}
\begin{proof}
Let $(y^*, z^*)$ be an optimal solution to $\DLP(\Ihat)$.
Let $(\yhat, z^*)$ be the monotonization of $(y^*, z^*)$.

In the canonical shelving of $I$, suppose a rectangular item $i$ (or a slice thereof)
lies in shelf $S$ where $S \in J_j$.
Then $h(i) \in [h_{j+1}, h_j]$, where $h_{t+1} \defeq 0$.
This is because shelves in $J \defeq \canShelv(\Ihat)$ are tight.
If $j = 1$, then $\eta(h(i)) = \yhat_1 \ge y^*_1$.
If $2 \le j \le t-1$, then $\eta(h(i)) \in \{\yhat_{j-1}, \yhat_j\} \ge \yhat_j \ge y^*_j$.
We are guaranteed that for $j \in [t-1]$, and each shelf $S \in J_j$, $w(S) = 1$.
\begin{align*}
p(I) &= \sum_{j=1}^t \sum_{S \in J_j} \sum_{i \in S} \eta(h(i))w(i)
    + \sum_{i \in \Ihat_S} \eta(h(i))w(i)
    \tag{by definition of $p$}
\\ &\ge \sum_{j=1}^{t-1} \sum_{S \in J_j} \sum_{i \in S} y^*_j w(i)
    + \sum_{i \in \Ihat_S} (h(i)z^*)w(i)
\tag{by definition of $\eta$}
\\ &= \sum_{j=1}^{t-1} y^*_j q + a(\Ihat_S)z^*
\tag{since $w(J_j) = q$ for $j \in [t-1]$}
\\ &= \opt(\DLP(\Ihat))
\tag{$(y^*, z^*)$ is optimal for $\DLP(\Ihat)$}
\end{align*}
By strong duality of linear programs, $\opt(\LP(\Ihat)) = \opt(\DLP(\Ihat)) \le p(I)$.

Since $\eta$ and $H_k$ are \dff{}s (by \thmdepcref{thm:hgap:eta-dff}{}),
% \thmdep{thm:h-dff}{}
we get that $p(I) \le T_k^{d-1}\optdbp(I)$ by \thmdepcref{thm:dff-pack}{}.
\end{proof}

\rthmHgapStruct*
\begin{proof}
\begin{align*}
& a(\Ihat_L) \le a(\Ihat)
= \sum_{i \in I} \left(\ell_d(i)\prod_{j=1}^{d-1} f_k(\ell_j(i))\right)
\le T_k^{d-1}\optdbp(I)
\tag{by \thmdepcref{thm:dff-pack}{thm:hgap:struct}}
\end{align*}
\begin{align*}
\sopt_{\delta}(\Ihat) &< \opt(J\lo \cup \Ihat_S) + \delta a(\Ihat_L) + (1+\delta)
\tag{by \thmdepcref{thm:hgap:sopt-le-optlo}{thm:hgap:struct}}
\\ &\le \opt(\LP(\Ihat)) + \ceil{\frac{1}{\delta^2}} + \delta T_k^{d-1}\optdbp(I) + (1+\delta)
\tag{by \thmdepcref{thm:hgap:opt-to-lp,thm:hgap:n-pivots}{thm:hgap:struct}}
\\ &\le T_k^{d-1}(1+\delta)\optdbp(I) + \ceil{\frac{1}{\delta^2}} + 1 + \delta
\tag{by \thmdepcref{thm:hgap:p-vs-opt}{thm:hgap:struct}}
\end{align*}
\end{proof}

\subsection{Guessing Shelves and Bins}
\label{sec:hgap:guess-shelves}

We want $\guessShelves(\Icalhat, \delta)$ to return all possible packings
of empty shelves into at most $n \defeq |\Icalhat|$ bins such that
each packing is structured for $(\flatten(\Icalhat), \delta)$.

Let $H = \{h(i): i \in \flatten(\Icalhat)\}$. Let $N \defeq |\flatten(\Icalhat)|$.
$\guessShelves(\Icalhat, \delta)$ starts by picking the distinct heights of shelves
by iterating over all subsets of $H$ of size at most $\ceildeltsq$.
The number of such subsets is at most $N^{\ceildeltsq}+1$.
Let $\Htild \defeq \{h_1, h_2, \ldots, h_t\}$ be one such guess, where $t \le \ceildeltsq$.
\WLoG, assume $h_1 > h_2 > \ldots > h_t > \delta$.

Next, $\guessShelves$ needs to decide the number of shelves of each height
and a packing of those shelves into bins.
Let $C \in \mathbb{Z}_{\ge 0}^t$ such that $h_C \defeq \sum_{j=1}^{t-1} C_jh_j \le 1$.
Then $C$ is called a \config{}.
$C$ represents a set of shelves that can be packed into a bin
and where $C_j$ shelves have height $h_j$.
Let $\Ccal$ be the set of all \config{}s.
We can pack at most $\ceil{1/\delta}-1$ items into a bin because $h_t > \delta$.
By \thmdepcref{thm:nos-sum}{thm:hgap:guess-shelves}, we get
\[ |\Ccal| \le \binom{\ceil{1/\delta}-1+t}{t}
\le \binom{\ceil{1/\delta}-1+\ceildeltsq}{\ceil{1/\delta}-1}
\le \left(\ceil{\frac{1}{\delta^2}}+1\right)^{1/\delta}. \]
There can be at most $n$ bins, and $\guessShelves$ has to decide the \config{} of each bin.
By \thmdepcref{thm:nos-sum}{thm:hgap:guess-shelves},
the number of ways of doing this is at most
$\binom{|\Ccal|+n}{|\Ccal|} \le (n+1)^{|\Ccal|}$.
Therefore, $\guessShelves$ computes all \config{}s and then
iterates over all $\binom{|\Ccal|+n}{|\Ccal|}$ combinations of these configs.

This completes the description of $\guessShelves$ and proves
\cref{thm:hgap:guess-shelves}.

\subsection{\texorpdfstring{$\chooseAndPack$}{choose-and-pack}}
\label{sec:hgap:choose-and-pack}

$\chooseAndPack(\Icalhat, P, \delta)$ takes as input a set $\Icalhat$ of 2D itemsets,
a packing $P$ of empty shelves into bins and constant $\delta \in (0, 1)$.
It tries to pack $\Icalhat$ into $P$ and one additional shelf.
Before we design $\chooseAndPack$, let us see how to handle a special case,
i.e., where $\Icalhat$ is \emph{simple}.
\begin{definition}
A set $\Icalhat$ of 2D itemsets is \emph{$\delta$-simple} iff
the width of each $\delta$-large item in $\flatten(\Icalhat)$
is a multiple of $1/|\Icalhat|$.
\end{definition}

Let $P$ be a bin packing of empty shelves.
Let $h_1 > h_2 > \ldots > h_t$ be the distinct heights of
the shelves in $P$, where $h_t > \delta$.
We will use dynamic programming to either pack a simple instance $\Icalhat$
into $P$ or claim that no assortment of $\Icalhat$ can be packed into $P$.
Call this algorithm $\simpleChooseAndPack(\Icalhat, P, \delta)$.

Let $\Icalhat \defeq \{I_1, I_2, \ldots, I_n\}$.
For $j \in \{0, 1, \ldots, n\}$, define $\Icalhat_j \defeq \{I_1, I_2, \ldots, I_j\}$,
i.e., $\Icalhat_j$ contains the first $j$ itemsets from $\Icalhat$.
Let $\vec{u} \defeq [u_1, u_2, \ldots, u_t] \in \{0, 1, \ldots, n^2\}^t$ be a vector.
Let $\Phi(j, \vec{u})$ be the set of all assortments of $\Icalhat_j$ that can be
packed into $t$ shelves, where the $r\Th$ shelf has height $h_r$ and width $u_r/n$.
For a set $K$ of items, define $\smallArea(K)$ as
the total area of $\delta$-small items in $K$.
Define $g(j, \vec{u}) \defeq \min_{K \in \Phi(j, \vec{u})} \smallArea(K)$.
If $\Phi(j, \vec{u}) = \emptyset$, then we let $g(j, \vec{u}) = \infty$.

We will show how to compute $g(j, \vec{u})$ for all $j \in \{0, 1, \ldots, n\}$
and all $\vec{u} \in \{0, 1, \ldots, n^2\}^t$ using dynamic programming.
Let there be $n_r$ shelves in $P$ having height $h_r$.
Then for $j = n$ and $u_r = n_r n$, $\Icalhat$ can be packed into $P$ iff
$g(j, \vec{u})$ is at most the area of non-shelf space in $P$.

Note that in any solution $K$ corresponding to $g(j, \vec{u})$,
we can assume without loss of generality that the item $i$ from $K \cap I_j$
is placed in the smallest shelves possible.
This is because we can always swap $i$ with the slices of items in those shelves.
This observation gives us the following recurrence relation for $g(j, \vec{u})$:
\begin{equation}
\label{eqn:g-rec}
g(j, \vec{u}) = \begin{cases}
\infty & \textrm{ if } u_j < 0 \textrm{ for some } j \in [t]
\\ 0 & \textrm{ if } n = 0 \textrm{ and } u_j \ge 0 \textrm{ for all } j \in [t]
\\ \min_{i \in I_j} \left(\begin{array}{ll}\smallArea(\{i\})
        \\ + \, g(j-1, \reduce(\vec{u}, i))\end{array} \right)
    & \textrm{ if } n > 0 \textrm{ and } u_j \ge 0 \textrm{ for all } j \in [t]
\end{cases}
\end{equation}
Here $\reduce(\vec{u}, i)$ is a vector obtained as follows:
If $i$ is $\delta$-small, then $\reduce(\vec{u}, i) \defeq \vec{u}$.
Otherwise, initialize $x$ to $w(i)$.
Let $p_i$ be the largest integer $r$ such that $h(i) \le h_r$.
For $r$ varying from $p_i$ to 2, subtract $\min(x, u_j)$ from $x$ and $u_j$.
Then subtract $x$ from $u_1$. The new value of $\vec{u}$ is defined to be
the output of $\reduce(\vec{u}, i)$.

The recurrence relation allows us to compute $g(j, \vec{u})$
for all $j$ and $\vec{u}$ using dynamic programming
in time $O(Nn^{2t})$ time, where $N \defeq |\flatten(\Icalhat)|$.
With a bit more work, we can also compute the corresponding assortment $K$, if one exists.
Therefore, $\simpleChooseAndPack(\Icalhat, P, \delta)$ computes a packing of $\Icalhat$
into $P$ if one exists, or returns \Null{} if no assortment of $\Icalhat$ can be packed into $P$.

Now we will look at the case where $\Icalhat$ is not $\delta$-simple.
Let $\Icalhat'$ be the instance obtained by rounding up the width
of each $\delta$-large item in $\Icalhat$ to a multiple of $1/n$, where $n \defeq |\Icalhat|$.
Let $\Pbar$ be the bin packing obtained by adding another bin to $P$
containing a single shelf of height $h_1$.
$\chooseAndPack(\Icalhat, P, \delta)$ computes $\Icalhat'$ and $\Pbar$,
and then returns the output of $\simpleChooseAndPack(\Icalhat', \Pbar, \delta)$.

\rthmCAPCorrect*
\begin{proof}
Follows from the definition of $\simpleChooseAndPack$.
\end{proof}

\rthmCAPNotNull*
\begin{proof}
Let $\Khat'$ be the items obtained by rounding up the width of each item in $\Khat$
to a multiple of $1/n$. Then $\Khat'$ is an assortment of $\Icalhat'$.
We will show that $\Khat'$ fits into $\Pbar$,
so $\simpleChooseAndPack(\Icalhat', \Pbar, \delta)$ will not output \Null.

Slice each item $i \in \Khat'$ into two pieces using a vertical cut such that
one piece has width equal to the original width of $i$ in $\Khat$,
and the other piece has width less than $1/n$.
This splits $\Khat'$ into sets $\Khat$ and $T$.
$T$ contains at most $n$ items, each of width less than $1/n$.
Therefore, we can pack $\Khat$ into $P$ and we can pack $T$ into
the newly-created shelf of height $h_1$.
Therefore, $\Khat'$ can be packed into $\Pbar$,
so $\simpleChooseAndPack(\Icalhat', \Pbar, \delta)$ won't output \Null.
\end{proof}

\rthmCAPTime*
\begin{proof}
The running time of $\chooseAndPack(\Icalhat, P, \delta)$
is dominated by computing $g(j, \vec{u})$ for all $j$ and $\vec{u}$,
which takes $O(Nn^{2t})$ time.
Since $P$ is structured for $(\Icalhat, \delta)$, the number of distinct shelves
in $P$, which is $t$, is at most $\ceildeltsq$.
\end{proof}

\subsection{\texorpdfstring{$\inflate$}{inflate}}
\label{sec:hgap:inflate}

Let $I$ be a set of $d$D items. Let $P$ be a shelf-based $\delta$-fractional
bin packing of $\Ihat \defeq \round(I)$ into $m$ bins.
Let there be $t$ distinct heights of shelves in $P$: $h_1 > h_2 > \ldots > h_t > \delta$.
We want to design an algorithm $\inflate(P)$ that returns a packing
of $I$ into approximately $|P|$ bins.

Define $\Ihat_L \defeq \{i \in \Ihat: h(i) > \delta\}$ and $\Ihat_S \defeq \Ihat - \Ihat_L$.
Let there be $Q$ distinct base types in $I$ (so $Q \le k^{d-1}$).

\subsubsection{Separating Base Types}
\label{sec:hgap:sep-btype}

We will now impose an additional constraint over $P$:
items in each shelf must have the same $\btype$.
This will be helpful later, when we will try to compute a packing of $d$D items $I$.

Separating base types of $\Ihat_S$ is easy, since we can slice them
in both directions. An analogy is to think of a mixture of
multiple immiscible liquids of different densities settling into equilibrium.
%See \cref{fig:split-base-small} for an example.

\begin{comment}
\begin{figure}[!ht]
\centering
\input{img/split-base-small.tikz}
\caption{Separating base types of $\Khat_S$ in a bin.
This example has 3 different base types in $\Khat_S$.}
\label{fig:split-base-small}
\end{figure}
\end{comment}

Let there be $n_j$ shelves of height $h_j$.
Let $\Ihat_j$ be the items packed into shelves of height $h_j$.
Therefore, $w(\Ihat_j) \le n_j$.
Let $\Ihat_{j,q} \subseteq \Ihat_j$ be the items of base type $q \in [Q]$.

For each $q$, pack $\Ihat_{j,q}$ into $\smallceil{w(\Ihat_{j,q})}$ shelves of height $h_j$
(slicing items if needed). For these newly-created shelves, define the $\btype$
of the shelf to be the $\btype$ of the items in it.
Let the number of newly-created shelves of height $h_j$ be $n_j'$. Then
\[ n_j' = \sum_{q=1}^Q \smallceil{w(\Ihat_{j,q})}
< \sum_{q=1}^Q w(\Ihat_{j,q}) + Q
\le n_j + Q \implies n_j' \le n_j + Q - 1 \]
$n_j$ of these shelves can be packed into existing bins in place of the old shelves.
The remaining $n_j' - n_j \le Q-1$ shelves can be packed on the base of new bins.

Therefore, by using at most $t(Q-1)$ new bins,
we can ensure that for every shelf,
all items in that shelf have the same $\btype$.
These new bins don't contain any items from $\Ihat_S$.
Call this new bin packing $P'$.

\subsubsection{Forbidding Horizontal Slicing}
\label{sec:hgap:forbid-hslice}

We will now use $P'$ to compute a shelf-based bin packing $P''$ of $\Ihat$ where
items in $\Ihat$ can be sliced using vertical cuts only.

Let $\Ihat_{q,S}$ be the items in $\Ihat_S$ of base type $q$.
Pack items $\Ihat_{q,S}$ into shelves using $\canShelvHyp$.
Suppose $\canShelv$ used $m_q$ shelves to pack $\Ihat_{q,S}$.
For $j \in [m_q]$, let $h_{q,j}$ be the height of the $j\Th$ shelf.
Let $H_q \defeq \sum_{j=1}^{m_q} h_{q,j}$ and $H \defeq \sum_{q=1}^Q H_q$.
Since for $j \in [m_q-1]$, all items in the $j\Th$ shelf have height at least $h_{q,j+1}$,
\[ a(\Ihat_{q,S}) > \sum_{j=1}^{m_q-1} h_{q,j+1} \ge H_q - h_{q,1} \ge H_q - \delta. \]
Therefore, $H < a(\Ihat_S) + Q\delta$.
Let $\Jhat_S$ be the set of these newly-created shelves.

Use Next-Fit to pack $\Jhat_S$ into the space used by $\Ihat_S$ in $P'$.
$\Ihat_S$ uses at most $m$ bins in $P'$ (recall that $m \defeq |P|$).
A height of less than $\delta$ will remain unpacked in each of those bins.
The total height occupied by $\Ihat_S$ in $P'$ is $a(\Ihat_S)$.
Therefore, Next-Fit will pack a height of more than $a(\Ihat_S) - \delta m$.

Some shelves in $\Jhat_S$ may still be unpacked.
Their total height will be less than
$H - (a(\Ihat_S) - \delta m) < \delta(Q + m)$.
We will pack these shelves into new bins using Next-Fit.
The number of new bins used is at most $\ceil{\delta(Q + m)/(1-\delta)}$.
Call this bin packing $P''$. The number of bins in $P''$
is at most $m' \defeq m + t(Q-1) + \ceil{\delta(Q+m)/(1-\delta)}$.

\subsubsection{Shelf-Based \texorpdfstring{$d$}{d}D packing}
\label{sec:hgap:2d-to-dd}

We will now show how to convert the packing $P''$ of $\Ihat$ that uses $m'$ bins
into a packing of $I$ that uses $m'$ $d$D bins.

First, we repack the items into the shelves.
For each $q \in [Q]$, let $\Jhat_q$ be the set of shelves in $P''$ of $\btype$ $q$.
Let $\Ihat^{[q]}$ be the items packed into $\Jhat_q$.
Compute $\Jhat^*_q \defeq \canShelvHyp(\Ihat^{[q]})$ and pack the shelves
$\Jhat^*_q$ into $\Jhat_q$. This is possible by
\thmdepcref{thm:hgap:can-shelv-pred}{thm:hgap:inflate}.

This repacking gives us an ordering of shelves in $\Jhat_q$.
Number the shelves from 1 onwards.
All items have at most 2 slices. If an item has 2 slices, and one slice is packed
into shelf number $p$, then the other slice is packed into shelf number $p+1$.
The slice in shelf $p$ is called the leading slice.
Every shelf has at most one leading slice.

Let $S_j$ be the $j\Th$ shelf of $\Jhat_q$.
Let $R_j$ be the set of unsliced items in $S_j$
and the item whose leading slice is in $S_j$.
Order the items in $R_j$ arbitrarily, except that the sliced item, if any, should be last.
Then $w(R_j - \last(R_j)) < 1$.
So, we can use $\hdhkunitHyp^{[q]}(R_j)$ to pack $R_j$ into a $(d-1)$D bin.
This $(d-1)$D bin gives us a $d$D shelf whose height is the same as that of $S_j$.
On repeating this process for all shelves in $\Jhat_q$ and for all $q \in [Q]$,
we get a packing of $I$ into shelves.
Since each $d$D shelf corresponds to a shelf in $P''$ of the same height,
we can pack these $d$D shelves into bins in the same way as $P''$.
This gives us a bin packing of $I$ into $m'$ bins.

\subsubsection{The Algorithm}

\Cref{sec:hgap:sep-btype,sec:hgap:forbid-hslice,sec:hgap:2d-to-dd}
describe how to convert a shelf-based $\delta$-fractional packing $P$
of $\Ihat$ having $t$ distinct shelf heights into a shelf-based $d$D bin packing of $I$.
We call this conversion algorithm $\inflate$.

It is easy to see that the time taken by $\inflate$ is $O(|I|\log|I|)$.

If $P$ has $m$ bins, then the number of bins in $\inflate(P)$ is at most
\[ m + t(Q-1) + \ceil{\frac{\delta(Q+m)}{1-\delta}}
< \frac{m}{1-\delta} + t(Q-1) + 1 + \frac{\delta Q}{1-\delta} \]
This proves \cref{thm:hgap:inflate}.

\subsection{Improving Running Time}
\label{sec:hgap:improve-time}

For simplicity of presentation, we left out some opportunities for improving the running
time of $\hgapk$. Here we briefly describe a way of speeding up $\hgapk$
which reduces its running time from $O(N^{1+\ceildeltsq}n^{R+2\ceildeltsq})$ to
$O(N^{1+\ceildeltsq}n^{2\ceildeltsq})$.
Here $N \defeq |\flatten(\Icalhat)|$, $n \defeq |\Icalhat|$, $\delta \defeq \eps/(2+\eps)$
and $R \defeq \binom{\ceildeltsq + \ceil{1/\delta}-1}{\ceil{1/\delta}-1}
\le (1+\ceildeltsq)^{1/\delta}$.

In $\guessShelves$, we guess two things simultaneously:
(i) the number and heights of shelves
(ii) the packing of the shelves into bins.
This allows us to guess the optimal structured $\delta$-fractional packing.
But we don't need that; an approximate structured packing would do.

Therefore, we only guess the number and heights of shelves.
We guess at most $N^{\ceildeltsq}+1$ distinct heights of shelves,
and by \cref{thm:nos-sum}, we guess at most $(n+1)^{\ceildeltsq}$ vectors
of shelf-height frequencies.
Then we can use Lueker and De La Vega's $O(n\log n)$-time APTAS for
1BP~\cite{bp-aptas} to pack the shelves into bins.

Also, once we guess the distinct heights of shelves,
we don't need to run $\chooseAndPack$ afresh for every packing of empty shelves.
We can reuse the dynamic programming table.

The running time is, therefore,
\[ O\left(N^{\ceildeltsq}\left(n^{\ceildeltsq} n\log n + Nn^{2\ceildeltsq}\right)\right)
= O(N^{1+\ceildeltsq}n^{2\ceildeltsq}) \]
