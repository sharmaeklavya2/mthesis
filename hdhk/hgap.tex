\section{Better Algorithm for \texorpdfstring{$d$}{d}MCBP (\texorpdfstring{$\hgapk$}{HGaP})}
\label{sec:hgap}

Here we will describe a $T_k^{d-1}(1+\eps)$-\asymAppx{} algorithm for $d$MCBP that is based on
$\hdhk$ and Lueker and de la Vega's APTAS for 1BP~\cite{bp-aptas}.
We call our algorithm \emph{Harmonic Guess-and-Pack} ($\hgapk$).
This improves upon $\fhk$ that has AAR $T_k^d$.

\begin{comment}
In \cite{caprara2008}, Caprara showed that we can get a $T_k^{d-1}(1+\eps)$-approx
algorithm for $d$BP by using $\hdhksp$ to pack the items into shelves
and then packing the shelves into bins using the APTAS for 1BP.
$\hgapk$ works the other way around, i.e., we first guess
the heights of the shelves and the way they are packed into bins,
and then we compute an assortment of the input that can be packed into the shelves.
\end{comment}

\begin{definition}
For a $d$D item $i$, let $h(i) \defeq \ell_d(i)$,
$w(i) \defeq \prod_{j=1}^{d-1} f_k(\ell_j(i))$ and $a(i) \defeq w(i)h(i)$.
Let $\round(i)$ be a rectangle of height $h(i)$ and width $w(i)$.
For a set $X$ of $d$D items, define $w(X) \defeq \sum_{i \in X} w(i)$
and $\round(X) \defeq \{\round(i): i \in X\}$.
%For a set $\Icalhat$ of $d$D itemsets, define $round(\Ical) \defeq \{\round(I): I \in \Ical\}$.
\end{definition}

For any $\eps > 0$, the algorithm $\hgapk(\Ical, \eps)$ returns
a bin packing of $\Ical$, where $\Ical$ is a set of $d$D itemsets.
$\hgapk$ first converts $\Ical$ to a set $\Icalhat$ of 2D itemsets.
It then computes $P_{\best}$, which is a \emph{structured} bin packing of $\Icalhat$
(we formally define \emph{structured} later).
Finally, it uses the algorithm $\inflate$ to convert $P_{\best}$
into a bin packing of the $d$D itemsets $\Ical$, where
$|\inflate(P_{\best})|$ is very close to $|P_{\best}|$.
See \cref{algo:hgap} for a more precise description.
This approach of converting items to 2D, packing them,
and then converting back to $d$D is very useful,
because most of our analysis is about how to compute a structured 2D packing, and
a packing of 2D items is easier to visualize and reason about than a packing of $d$D items.

\begin{algorithm}[!ht]
\caption{$\hgapk(\Ical, \eps)$: Returns a bin packing of $d$D itemsets $\Ical$,
where $\eps \in (0, 1)$.}
\label{algo:hgap}
\begin{algorithmic}[1]
\State Let $\delta \defeq \eps/(2+\eps)$.
\State \label{alg-line:hgap:round}$\Icalhat = \{\round(I): I \in \Ical\}$
\State Initialize $P_{\best}$ to \Null.
\For{$P \in \guessShelves(\Icalhat, \delta)$}
    \State $\Pbar = \chooseAndPack(\Icalhat, P, \delta)$
    \If{$\Pbar$ is not \Null{} and ($P_{\best}$ is \Null{} or $|\Pbar| \le |P_{\best}|$)}
        \State $P_{\best} = \Pbar$
    \EndIf
\EndFor
\State \Return $\inflate(P_{\best})$
\end{algorithmic}
\end{algorithm}

A bin packing is said to be \emph{shelf-based} if items are packed into \emph{shelves}
and the shelves are packed into bins, where a shelf is a rectangle of width 1.
See \cref{fig:shelf-based-2} for an example.
A structured bin packing is a shelf-based bin packing
where the heights of the shelves satisfy some additional properties
(we describe these properties later).
The algorithm $\guessShelves$ repeatedly guesses the number and heights of shelves
and computes a structured packing $P$ of those shelves into bins.
Then for each packing $P$, the algorithm $\chooseAndPack(\Icalhat, P, \delta)$ \emph{tries to}
pack an assortment of $\Icalhat$ into the shelves in $P$ plus maybe one additional shelf.
If $\chooseAndPack$ succeeds, call the resulting bin packing $\Pbar$.
Otherwise, $\chooseAndPack$ returns \Null.
$P_{\best}$ is the value of $\Pbar$ with the minimum number of bins
across all guesses by $\guessShelves$.

\begin{figure}[!ht]
\centering
\begin{tikzpicture}[scale=0.8,
shelf-line/.style = {draw={textColor!30!bgColor}},
item/.style = {fill={textColor!10!bgColor}}
]
%\fill[textColor!8!bgColor] (0,3.5) -- (4,4);
\draw[shelf-line] (0,1.5) -- (4,1.5);
\draw[shelf-line] (0,2.7) -- (4,2.7);
\draw[shelf-line] (0,3.5) -- (4,3.5);
\draw (0,0) rectangle (4,4);
\draw[item]
    (0.0,0) rectangle +(1.3,1.5)
    (1.3,0) rectangle +(1.6,1.3)
    (2.9,0) rectangle +(1.1,1.2);
\draw[item]
    (0.0,1.5) rectangle +(0.4,1.2)
    (0.4,1.5) rectangle +(2,1)
    (2.4,1.5) rectangle +(1.6,0.8);
\draw[item]
    (0,2.7) rectangle +(2,0.8)
    (2,2.7) rectangle +(1,0.6);
\end{tikzpicture}

\caption{An example of shelf-based packing with 3 shelves.}
\label{fig:shelf-based-2}
\end{figure}

We prove that $\hgapk$ is $T_k^{d-1}(1+\eps)$-\asymAppx{} by showing that
for some $P^* \in \guessShelves(\Icalhat, \delta)$,
we have $|P^*| \lessapprox T_k^{d-1}\opt(\Ical)(1+\eps)$
and $\chooseAndPack(\Icalhat, P^*, \delta)$ is not \Null.

We will now precisely define \emph{structured} bin packing
and state the main theorems on $\hgapk$.

\subsection{Structured Packing}

\begin{definition}[Slicing]
Slicing a 1D item $i$ is the operation of replacing it by items $i_1$ and $i_2$
such that $\size(i_1) + \size(i_2) = \size(i)$.

Slicing a rectangle $i$ using a vertical cut is the operation of replacing $i$
by two rectangles $i_1$ and $i_2$ where
$h(i) = h(i_1) = h(i_2)$ and $w(i) = w(i_1) + w(i_2)$.
Slicing $i$ using a horizontal cut is the operation of replacing $i$
by two rectangles $i_1$ and $i_2$ where
$w(i) = w(i_1) = w(i_2)$ and $h(i) = h(i_1) + h(i_2)$.
\end{definition}

\begin{definition}[Shelf-based $\delta$-fractional packing]
\label{defn:hgap:shelf-based-packing}
Let $\delta \in (0, 1)$ be a constant. Let $K$ be a set of rectangular items.
Items in $K_L \defeq \{i \in K: h(i) > \delta\}$ are said to be `$\delta$-large'
and items in $K_S \defeq K - K_L$ are said to be `$\delta$-small'.
A $\delta$-fractional bin packing of $K$ is defined to be a packing of $K$ into bins where
items in $K_L$ can be sliced (recursively) using vertical cuts only,
and items in $K_S$ can be sliced (recursively) using both horizontal and vertical cuts.

A \emph{shelf} is a rectangle of width 1 into which we can pack items
such that the bottom edge of each item in the shelf touches
the bottom edge of the shelf. A shelf can itself be packed into a bin.
A $\delta$-fractional bin packing of $K$ is said to be \emph{shelf-based} iff
(all slices of) all items in $K_L$ are packed into shelves,
the shelves are packed into the bins,
and items in $K_S$ are packed outside the shelves (and inside the bins).
Packing of items into a shelf $S$ is said to be \emph{tight} iff
the top edge of some item (or slice) in $S$ touches the top edge of $S$.
\end{definition}

\begin{definition}[Structured packing]
Let $K$ be a set of rectangles and
let $P$ be a packing of empty shelves into bins.
Let $H$ be the set of heights of shelves in $P$ (note that $H$ is not a multiset,
i.e., we only consider distinct heights of shelves).
Then $P$ is said to be \emph{structured} for $(K, \delta)$ iff
$|H| \le \ceildeltsq$ and each element in $H$ is
the height of some $\delta$-large item in $K$.

A shelf-based $\delta$-fractional packing of $K$ is said to be
\emph{structured} iff the shelves in the packing are structured for $(K, \delta)$.
Define $\sopt_{\delta}(K)$ to be the number of bins in the optimal structured
$\delta$-fractional packing of $K$.
\end{definition}

$\hgapk$ relies on the following key structural theorem.
We formally prove it in \cref{sec:hgap:struct} and give an outline of the proof here.
\begin{restatable}[Structural theorem]{theorem}{rthmHgapStruct}
\label{thm:hgap:struct}
Let $I$ be a set of $d$D items. Let $\delta \in (0, 1)$ be a constant. Then
$\sopt_{\delta}(\round(I)) < T_k^{d-1}(1+\delta)\optdbp(I) + \ceildeltsq + 1 + \delta$.
\end{restatable}
\begin{proof}[Proof outline]
Let $\Ihat \defeq \round(I)$. Let $\Ihat_L$ and $\Ihat_S$ be the
$\delta$-large and $\delta$-small items in $\Ihat$, respectively.

We give a simple greedy algorithm to pack $\Ihat_L$ into shelves.
Let $J$ be the shelves output by this algorithm.
We can treat $J$ as a 1BP instance,
and $\Ihat_S$ as a sliceable 1D item of size $a(\Ihat_S)$.
We prove that an optimal 1D bin packing of $J \cup \Ihat_S$ gives us
an optimal shelf-based $\delta$-fractional packing of $\Ihat$.

We use linear grouping by Lueker and Vega~\cite{bp-aptas}.
We partition $J$ into linear groups of size $\floor{\delta\size(J)} + 1$ each.
Let $h_j$ be the height of the first 1D item in the $j\Th$ group.
Let $J\hi$ be the 1BP instance obtained by rounding up the height of each item
in the $j\Th$ group to $h_j$ for all $j$.
Then $J\hi$ contains at most $\ceildeltsq$ distinct sizes,
so the optimal packing of $J\hi \cup \Ihat_S$ gives us a
structured $\delta$-fractional packing of $\Ihat$.
Therefore, $\sopt_{\delta}(\Ihat) \le \opt(J\hi \cup \Ihat_S)$.
Let $J\lo$ be the 1BP instance obtained by rounding down the height of each item
in the $j\Th$ group to $h_{j+1}$ for all $j$.
We prove that $J\lo$ contains at most $\ceildeltsq-1$ distinct sizes and that
$\opt(J\hi \cup \Ihat_S) < \opt(J\lo \cup \Ihat_S) + \delta a(\Ihat_L) + (1 + \delta)$.

We model packing $J\lo \cup \Ihat_S$ as a linear program, denoted by $\LP(\Ihat)$,
that has at most $\ceildeltsq^{1/\delta}$ variables and $\ceildeltsq$ non-trivial constraints.
The optimum extreme point solution to $\LP(\Ihat)$, therefore,
has at most $\ceildeltsq$ positive entries,
so $\opt(J\lo \cup \Ihat_S) \le \opt(\LP(\Ihat)) + \ceildeltsq$.

We use techniques from Caprara~\cite{caprara2008} to obtain a
monotonic \dff{} $\eta$ from the optimal solution to the dual of $\LP(\Ihat)$.
For each item $i \in I$, we define $p(i) \defeq w(i)\eta(h(i))$
and prove that $p(I) \ge \opt(\LP(\Ihat))$.
By \cref{thm:dff-pack}, we get that $p(I) \le T_k^{d-1}\optdbp(I)$
and $a(\Ihat_L) \le T_k^{d-1}\optdbp(I)$.
Combining the above facts gives us an upper-bound on $\sopt_{\delta}(\Ihat)$
in terms of $\optdbp(I)$.
\end{proof}

\subsection{Subroutines}

\subsubsection{\texorpdfstring{$\guessShelves$}{guess-shelves}}

The algorithm $\guessShelves(\Icalhat, \delta)$ takes a set $\Icalhat$
of 2D itemsets and a constant $\delta \in (0, 1)$ as input.
We will design $\guessShelves$ so that it satisfies the following theorem.
\begin{theorem}
\label{thm:hgap:guess-shelves}
$\guessShelves(\Icalhat, \delta)$ returns all possible packings of empty shelves into at most
$|\Icalhat|$ bins such that each packing is structured for $(\flatten(\Icalhat), \delta)$.
$\guessShelves(\Icalhat, \delta)$ returns at most $T \defeq (N^{\ceildeltsq}+1)(n+1)^R$
packings, where $N \defeq |\flatten(\Icalhat)|$, $n \defeq |\Icalhat|$,
and $R \defeq \binom{\ceildeltsq + \ceil{1/\delta}-1}{\ceil{1/\delta}-1}
\le (1+\ceildeltsq)^{1/\delta}$. Its running time is $O(T)$.
\end{theorem}
$\guessShelves$ works by first guessing at most $\ceildeltsq$ distinct heights of shelves.
It then enumerates all configurations, i.e.,
different ways in which shelves can be packed into a bin.
It then guesses the configurations in a bin packing of the shelves.
$\guessShelves$ can be easily implemented using standard techniques.
For the sake of completeness, we give a more precise description of $\guessShelves$
and prove \cref{thm:hgap:guess-shelves} in \cref{sec:hgap:guess-shelves}.

\subsubsection{\texorpdfstring{$\chooseAndPack$}{choose-and-pack}}

$\chooseAndPack(\Icalhat, P, \delta)$ takes as input a set $\Icalhat$ of 2D itemsets,
a constant $\delta \in (0, 1)$, and a bin packing $P$ of empty shelves that is
structured for $(\flatten(\Icalhat), \delta)$.
It tries to pack an assortment of $\Icalhat$ into the shelves in $P$.

$\chooseAndPack$ works by rounding up the width of all
$\delta$-large items in $\Icalhat$ to a multiple of $1/n$.
This would increase the number of shelves required by 1, so it adds another empty shelf.
It then uses dynamic programming to pack an assortment into the shelves,
such that the area of the chosen $\delta$-small items is minimum.
This is done by maintaining a dynamic programming table that keeps track of
the number of itemsets considered so far and
the remaining space in shelves of each type.
If it is not possible to pack the items into the shelves,
then $\chooseAndPack$ outputs \Null.
In \cref{sec:hgap:choose-and-pack}, we give the details of this algorithm
and formally prove the following theorems:

\begin{restatable}{theorem}{rthmCAPNotNull}
\label{thm:hgap:cap-not-null}
If there exists an assortment $\Khat$ of $\Icalhat$ having
a structured $\delta$-fractional bin packing $P$,
then $\chooseAndPack(\Icalhat, P, \delta)$ does not output \Null.
\end{restatable}
\begin{restatable}{theorem}{rthmCAPCorrect}
\label{thm:hgap:cap-correct}
If the output of $\chooseAndPack(\Icalhat, P, \delta)$ is not \Null,
then the output $\Pbar$ is a shelf-based $\delta$-fractional packing of some assortment
of $\Icalhat$ such that $|\Pbar| \le |P| + 1$
and the distinct shelf heights in $\Pbar$ are the same as that in $P$.
\end{restatable}
\begin{restatable}{theorem}{rthmCAPTime}
\label{thm:hgap:cap-time}
$\chooseAndPack(\Icalhat, P, \delta)$ runs in $O(Nn^{2\ceildeltsq})$ time.
Here $N \defeq |\flatten(\Icalhat)|$, $n \defeq |\Icalhat|$.
\end{restatable}

\subsubsection{\texorpdfstring{$\inflate$}{inflate}}

For a set $I$ of $d$D items, $\inflate$ is an algorithm that
converts a shelf-based packing of $\round(I)$ into a packing of $I$
having roughly the same number of bins.

For a $d$D item $i$, $\btype(i)$ (called \emph{base type}) is defined to be
a $(d-1)$-dimensional vector whose $j\Th$ component is $\type_k(\ell_j(i))$.
Roughly, $\inflate(P)$ works as follows:
It first slightly modifies the packing $P$ so that
items of different base types are in different shelves and
$\delta$-small items are no longer sliced using horizontal cuts.
Then it converts each 2D shelf to a $d$D shelf of the same height using $\hdhkunitHyp$
(a $d$D shelf is a cuboid where the first $d-1$ dimensions are equal to 1).

In \cref{sec:hgap:inflate}, we formally describe $\inflate$ and prove the following theorem about it.
\begin{theorem}
\label{thm:hgap:inflate}
Let $I$ be a set of $d$D items having $Q$ distinct base types.
Let $P$ be a shelf-based $\delta$-fractional packing of $\round(I)$
where shelves have $t$ distinct heights.
Then $\inflate(P)$ returns a packing of $I$ into less than
$|P|/(1-\delta) + t(Q-1) + 1 + \delta Q/(1-\delta)$ bins
in $O(n\log n)$ time, where $n \defeq |I|$.
\end{theorem}

Now that we have mentioned the guarantees of all the subroutines used by $\hgapk$,
we can prove the correctness and running time of $\hgapk$.

\subsection{Correctness and Running Time of \texorpdfstring{$\hgapk$}{HGaP}}

\begin{theorem}
\label{thm:hgap:hgap}
The number of bins used by $\hgapk(\Ical, \eps)$ to pack $\Ical$ is less than
\[ T_k^{d-1}(1+\eps)\optdmcbp(\Ical)
    + \ceil{\left(\frac{2}{\eps}+1\right)^2}\left(Q + \frac{\eps}{2}\right)
    + 3 + (Q+3)\frac{\eps}{2}. \]
Here $Q$ is the number of distinct base types in $\flatten(\Ical)$.
\end{theorem}
\begin{proof}
Let $K^*$ be the assortment in an optimal bin packing of $\Ical$.
Let $\Khat^* = \round(K^*)$.
Let $P^*$ be the optimal structured $\delta$-fractional bin packing of $\Khat^*$.
Then $|P^*| = \sopt_{\delta}(\Khat^*)$ by the definition of $\sopt$.
By \thmdepcref{thm:hgap:guess-shelves}{}, $P^* \in \guessShelves(\Icalhat, \delta)$.
Let $\Pbar^* = \chooseAndPack(\Icalhat, P^*, \delta)$.
By \thmdepcref{thm:hgap:cap-not-null}{}, $\Pbar^*$ is not \Null{}.
By \thmdepcref{thm:hgap:cap-correct}{}, $P_{\best}$ is
structured for $(\flatten(\Icalhat), \delta)$
and $|P_{\best}| \le |\Pbar^*| \le \sopt_{\delta}(\Khat^*) + 1$.

By \thmdepcref{thm:hgap:inflate}{}, we get that
\[ |\inflate(P_{\best})| < \frac{\sopt_{\delta}(\Khat^*)}{1-\delta}
    + \ceil{\frac{1}{\delta^2}}(Q-1) + 1 + \frac{\delta Q + 1}{1-\delta}. \]
By \thmdepcref{thm:hgap:struct}{} (structural theorem)
and using $\optdbp(K^*) = \optdmcbp(\Ical)$, we get
\[ \sopt_{\delta}(\Khat^*) < T_k^{d-1}(1+\delta)\optdmcbp(\Ical) + \ceildeltsq + 1 + \delta. \]
Therefore, $|\inflate(P_{\best})|$ is less than
\begin{align*}
&T_k^{d-1}\frac{1+\delta}{1-\delta}\optdmcbp(\Ical)
    + \ceil{\frac{1}{\delta^2}}\left(Q + \frac{\delta}{1-\delta}\right)
    + 3 + \frac{\delta(3+Q)}{1-\delta}
\\ &= T_k^{d-1}(1+\eps)\optdmcbp(\Ical)
    + \ceil{\left(\frac{2}{\eps}+1\right)^2}\left(Q + \frac{\eps}{2}\right)
    + 3 + (Q+3)\frac{\eps}{2}.
\qedhere \end{align*}
\end{proof}
\begin{theorem}
\label{thm:hgap:hgap-time}
$\hgapk(\Ical, \eps)$ runs in time $O(N^{1+\ceildeltsq}n^{R+2\ceildeltsq})$, where
$N \defeq |\flatten(\Icalhat)|$, $n \defeq |\Icalhat|$, $\delta \defeq \eps/(2+\eps)$
and $R \defeq \binom{\ceildeltsq + \ceil{1/\delta}-1}{\ceil{1/\delta}-1}
\le (1+\ceildeltsq)^{1/\delta}$.
\end{theorem}
\begin{proof}
Follows from \thmdepcref{thm:hgap:guess-shelves,thm:hgap:cap-time,thm:hgap:inflate}{}.
\end{proof}
\Cref{sec:hgap:improve-time} gives hints on improving the running time of $\hgapk$.
