In this chapter, we focus on the $d$-dimensional multiple-choice geometric
bin packing problem ($d$MCBP).
This problem generalizes both the non-rotational and rotational versions of $d$D GBP.
See \cref{sec:intro:hdhk} for a detailed introduction to this problem
and its significance.

In $d$MCBP, we are given a set $\Ical = \{I_1, I_2, \ldots, I_n\}$ as input,
where for each $j$, $I_j$ is a set of items, henceforth called an {\em itemset}.
We have to pick exactly one item from each itemset and pack those items
into the minimum number of bins.

\subsection*{Preliminaries}

For convenience, in this chapter only, we will denote non-rotational $d$D GBP by $d$BP
and non-rotational $d$D strip packing by $d$SP.
In $d$BP and $d$MCBP, we assume \wLoG{} that bins are $d$D cubes of side length 1.

We now give an equivalent definition of $d$MCBP for notational convenience.
Let $\Ical$ be a set of itemsets.
Let $K$ be a set of items that contains exactly one item from each itemset in $\Ical$.
Formally, for each itemset $I \in \Ical$, $|K \cap I| = 1$.
Then $K$ is called an \emph{assortment} of $\Ical$.
Let $\assortSet(\Ical)$ denote the set of all assortments of $\Ical$.
%
In $d$MCBP, given an input instance $\Ical$,
we have to select an assortment $K \in \assortSet(\Ical)$ and output a bin packing of $K$,
such that the number of bins used is minimized. Therefore,
$\optdmcbp(\Ical) = \min_{K \in \assortSet(\Ical)} \optdbp(K)$.

Define $\flatten(\Ical)$ as the union of all itemsets in $\Ical$.
Then $n \defeq |\Ical|$ is the number of itemsets in $\Ical$
and $N \defeq |\flatten(\Ical)|$ is the total number of items across all itemsets of $\Ical$.

\subsection*{Overview of the Chapter}

\begin{itemize}
\item In \cref{sec:hdhk-prelims}, we describe ideas from $\hdhk$~\cite{caprara2008}
    that help us devise harmonic-based algorithms for $d$MCBP.
    For two of these ideas, $\hdhkunit$ and \dff{}s,
    we give more details in \cref{sec:hdhkunit,sec:dff-trn}, respectively.
\item In \cref{sec:fhk}, we show a simple $O(N + n\log n)$-time algorithm for $d$MCBP,
    called $\fhk$, having an AAR of $T_k^d$, where $n$ is the number of itemsets
    and $N$ is total number of items across all the $n$ itemsets.
\item We present an algorithm for $d$MCBP, called $\hgapk$, having an AAR of
    $T_k^{d-1}(1+\eps)$ and having a running time of $N^{O(1/\eps^2)}n^{(1/\eps)^{O(1/\eps)}}$.
    We give an overview of $\hgapk$ in \cref{sec:hgap}
    and give its details in \cref{sec:hgap-extra}.
\item In \cref{sec:hdhk-sp}, we define the $d$D multiple-choice strip packing problem ($d$MCSP)
    and extend Caprara's $\hdhk$ algorithm~\cite{caprara2008} to $d$MCSP.
    The algorithm has an AAR of $T_k^{d-1}$ and runs in time $O(N + n\log n)$.
\item In \cref{sec:hdhks}, we define the $d$D multiple-choice geometric knapsack problem ($d$MCKS),
    and for any $0 < \eps < 1$, we show an $O(N\log N + Nn/\eps)$-time algorithm
    that is $3^d(1+\eps)$-approximate.
\item Caprara~\cite{caprara2008} showed that no shelf-based algorithm for 2BP or 2SP
    can get an AAR better than $T_{\infty} \approx 1.69103$,
    and his $\hdhk$ algorithm achieves an AAR of $T_k^{d-1}$ for $d$BP and $d$SP.
    In \cref{sec:hard-example}, we extend that result to show that no shelf-based algorithm
    for $d$BP or $d$SP can get an AAR better than $T_{\infty}^{d-1}$.
\end{itemize}
