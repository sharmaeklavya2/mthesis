\chapter{Introduction}
\label{chap:intro}

\section{Classical Bin Packing Problem}

In the classical bin packing problem, we are given a set $I$ of items.
Each item $i \in I$ has a size $s(i) \in (0, 1]$ associated with it.
Our goal is to partition $I$ into the minimum number of bins,
such that the sum of sizes of items in each bin is at most 1.
The classical bin packing problem and its generalizations
have diverse applications in computer science and operations research,
like packing trucks with a given weight limit,
allocating jobs to servers,
allocating memory in computers~\cite{handbook-of-combinopt-bp},
or assigning commercials to station breaks in television programming,

Classical bin packing is known to be NP-hard.
In fact, deciding whether a set of items can be packed into two bins is NP-complete,
by a simple reduction from the partition problem.
Hence, we look at approximation algorithms.
Let $\opt(I)$ be the minimum number of bins required to pack a set $I$ of items.
An algorithm $\Acal$ is said to be $\alpha$-approximate iff
$\Acal$ requires at most $\alpha\opt(I)$ bins to pack $I$.

Since it is NP-complete to decide whether a set of items can be packed into two bins,
it is NP-hard to obtain a polynomial-time algorithm for bin packing
with approximation ratio less than $3/2$.
(Using the results of D\'osa and Sgall, we can prove that the First-Fit Decreasing algorithm
is $3/2$-approximate~\cite{dosa2013first,dosa2007tight}.)
However, such a reasoning doesn't rule out the existence of an algorithm
that uses $\opt(I) + 1$ bins.
Therefore, we turn our attention to \emph{asymptotic approximation algorithms}.
\begin{definition}
A bin packing algorithm $\Acal$ is said to be $\alpha$-asymptotic-approximate iff
$\Acal$ requires at most $\alpha\opt(I) + \beta$ bins
for some value $\beta \in o(\opt(I))$ (usually, $\beta$ is a constant).
$\alpha$ is called the asymptotic approximation ratio (AAR) of $\Acal$.
\end{definition}

An Asymptotic Polynomial-Time Approximation Scheme (APTAS) is an algorithm
that accepts a parameter $\eps > 0$ and gives an AAR of $1+\eps$.
The running time for such an algorithm usually increases as $\eps$ decreases.
Lueker and Vega~\cite{bp-aptas} gave an APTAS for classical bin packing.

In the online version of bin packing, the items arrive one-by-one,
and for each item, we have to immediately and irrevocably pack it into a bin.
In the online version, the (asymptotic) approximation ratio is also called the
(asymptotic) \emph{competitive ratio}.
(The non-online version is called the offline version, i.e., where we can read the
whole input before we start packing.)

\subsection{Prior Work}

The Next-Fit algorithm~\cite{johnson-thesis} is one of the simplest algorithms
for the online classical bin packing problem.
In this algorithm, we start with an empty bin, and designate it as the \emph{open bin}.
We repeatedly pack items into the open bin till we
come across an item that doesn't fit in the open bin.
We then \emph{close} that bin and \emph{open} a new bin and resume.
Let $s(I)$ denote the sum of sizes of all items in $I$.
It is easy to prove that Next-Fit uses at most $\ceil{2s(I)}$ bins.
Since $s(I) \le \opt(I)$, we get that Next-Fit is 2-approximate.

Lee and Lee~\cite{leelee} gave an algorithm for online classical bin packing,
called the $\operatorname{Harmonic}_k$ algorithm.
This algorithm takes as input a set $I$ of items and an integer parameter $k \ge 2$.
The number of bins used by $\operatorname{Harmonic}_k$ to pack $I$ is less than
$T_k\opt(I) + k$, where $T_k$ is a decreasing function of $k$
and $T_{\infty} \defeq \lim_{k \to \infty} T_k \approx 1.6910302$.

Many other algorithms have been devised for the online classical bin packing problem.
See \cref{table:online-1bp} for examples and~\cite{handbook-of-combinopt-bp} for a detailed survey.
The best algorithm we are aware of is the Advanced Harmonic algorithm by
Balogh, B\'ek\'esi, D\'osa, Epstein and Levin~\cite{balogh2018}
that has an AAR of 1.57829.

\begin{table}[ht]
\centering
\caption{Online Algorithms for bin packing.}
\begin{tabular}{ll}
\toprule Algorithm & Approximation guarantee
\\ \midrule Next-Fit~\cite{johnson-thesis}
    & $\le 2\opt(I)$
\\[\defaultaddspace] First-Fit~\cite{dosa2013first}
    & $\le \floor{1.7\opt(I)}$
%\\[\defaultaddspace] \hline Harmonic$_k$~\cite{leelee}
    %& $< T_k\opt(I) + k - H(k-1)$
\\[\defaultaddspace] Harmonic$_k$~\cite{leelee}
    & $< T_k\opt(I) + k$\quad $(T_\infty \approx 1.69103)$
\\[\defaultaddspace] Advanced Harmonic~\cite{balogh2018}
    & $\le 1.57829\opt(I) + O(1)$
\\ \bottomrule
\end{tabular}
\label{table:online-1bp}
\end{table}

Balogh, B\'ek\'esi and Galambos gave a lower bound of $248/161 \approx 1.54037$
on the asymptotic competitive ratio~\cite{balogh2012new}.
This bound was recently improved to 1.54278~\cite{balogh2021new},
which is the best-known lower bound we are aware of.
See~\cite{handbook-of-combinopt-bp} for a detailed survey on lower bounds.

For the offline version of classical bin packing,
First-Fit Decreasing (FFD) is a popular algorithm.
Johnson~\cite{johnson-thesis} showed that the number of bins used by
FFD is at most $(11/9)\opt(I) + 4$.
The additive constant $4$ was improved by D\'osa
to $2/3$, which is tight~\cite{dosa2007tight}.

Lueker and Vega gave the first APTAS for classical bin packing~\cite{bp-aptas}.
This was later improved upon by Karmarkar and Karp to an algorithm
that uses at most $\opt(I) + O(\log^2\opt(I))$ bins~\cite{karmarkar-karp}.
Rothvoss gave an algorithm that uses at most
$\opt(I) + O(\log\opt(I)\log\log\opt(I))$ bins~\cite{rothvoss2013}.
Hoberg and Rothvoss gave an algorithm that uses at most
$\opt(I) + O(\log\opt(I))$ bins~\cite{HobergR17}
(see \cref{table:offline-1bp}).

\begin{table}[ht]
\centering
\caption{Offline Algorithms for bin packing.}
\begin{tabular}{ll}
\toprule Algorithm & Approximation guarantee
\\ \midrule First-Fit Decreasing (FFD)~\cite{dosa2007tight}
    & $\le \frac{11}{9}\opt(I) + \frac{2}{3}$
\\[\defaultaddspace] Lueker, Vega~\cite{bp-aptas}
    & $\le (1+\eps)\opt(I) + O(1/\eps^2)$
\\[\defaultaddspace] Karmarkar, Karp~\cite{karmarkar-karp}
    & $\le \opt(I) + O(\log^2\opt(I))$
\\[\defaultaddspace] Rothvoss~\cite{rothvoss2013}
    & $\le \opt(I) + O(\log\opt(I)\log\log\opt(I))$
\\[\defaultaddspace] Hoberg, Rothvoss~\cite{HobergR17}
    & $\le \opt(I) + O(\log\opt(I))$
\\ \bottomrule
\end{tabular}
\label{table:offline-1bp}
\end{table}

On the hardness side, to the best of our knowledge,
an $\opt(I) + 1$ algorithm hasn't yet been proven to not exist.
