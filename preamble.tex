% use packages
\usepackage{amssymb,amsmath,amsthm}
\usepackage{cleveref}
\usepackage{algorithm,algpseudocode}
\usepackage{enumerate}
\usepackage{mathrsfs}
\usepackage{xcolor}
\usepackage{tikz}
\usepackage{datetime}
\usetikzlibrary{shapes,arrows, trees}

% Numbered environments
\newtheorem{remarks}{Remarks}[chapter] %\label{rmk:}
\newtheorem{example}{Example}[chapter] %\label{exp:}
\newtheorem{theorem}{Theorem}[chapter]
\newtheorem{lemma}{Lemma}[chapter]
\newtheorem{corollary}{Corollary}[chapter]
\newtheorem{discussion}{Discussion}[chapter] %\label{dis:}
\newtheorem{definition}{Definition}[chapter]
\newtheorem{proposition}{Proposition}[chapter]
\newtheorem{mechanism}{Mechanism}[chapter]
\newtheorem{question}{Question}[chapter]
\newtheorem{observation}{Observation}[chapter]
\newtheorem{fact}{Fact}[chapter]

\crefname{observation}{observation}{observations}
\crefname{algorithm}{algorithm}{algorithms}
\crefname{align}{equation}{equations}
\crefname{eqnarray}{equation}{equations}

% shorthands
\newcommand*{\Th}{^{\textrm{th}}}
\let\eps\varepsilon

% math
\newcommand*{\floor}[1]{\left\lfloor #1 \right\rfloor}
\newcommand*{\ceil}[1]{\left\lceil #1 \right\rceil}
\newcommand*{\abs}[1]{\left\lvert #1 \right\rvert}
\newcommand*{\norm}[1]{\left\lVert #1 \right\rVert}
\DeclareMathOperator*{\E}{E}
\DeclareMathOperator*{\Var}{Var}
\DeclareMathOperator*{\argmin}{argmin}
\DeclareMathOperator*{\argmax}{argmax}
\DeclareMathOperator{\opt}{opt}

% Miscellaneous
\allowdisplaybreaks[1]
\newdateformat{monthyeardate}{\monthname[\THEMONTH], \THEYEAR}
\newcommand{\blankpage}{\newpage\thispagestyle{empty}\mbox{}\newpage}
