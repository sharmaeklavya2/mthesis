\chapter{Notation and Preliminaries}
\label{chap:preliminaries}

\section{Notation}

\subsection{General}

For an integer $n \ge 0$, define $[n] \defeq \{1, 2, \ldots, n\}$.
Define $\poly(n)$ as the set of polynomial and sub-polynomial functions of $n$.
For a set $X$, define $|X|$ as the cardinality of the set
and define $\Sum(X) \defeq \sum_{x \in X} x$.

For a vector $\xvec \in \mathbb{R}^d$, we denote the $i\Th$ entry by $\xvec_i$ or $\xvec[i]$.
Define $\Sum(\xvec) \defeq \sum_{i=1}^d \xvec_i$.
Define the support of $\xvec$ as the set of indices of $\xvec$ with non-zero value,
i.e., $\support(\xvec) \defeq \{i \in [d]: \xvec_i \neq 0\}$.
The $\ell_p$-norm of $\xvec$, denoted as $\norm{\xvec}_p$, is defined as
\[ \norm{\xvec}_p \defeq \left( \sum_{i=1}^d \abs{\xvec_i}^p \right)^{1/p}. \]
Note that if all entries of $\xvec$ are non-negative, then $\Sum(\xvec) = \norm{\xvec}_1$.

For a matrix $A$, we define the entry at the $i\Th$ row and $j\Th$ column
by $A_{i,j}$ or $A[i, j]$.

\subsection{Items}
\label{sec:preliminaries:items}

Let $i$ be a $d$D cuboidal item
(this means that $i$ is the cross-product of $d$ intervals on the real line).
Denote the length of $i$ along the $j\Th$ dimension by $\ell_j(i)$.
The first $d-1$ dimensions of $i$ are called \emph{base dimensions}
and the $d\Th$ dimension of $i$ is called \emph{height}, denoted by $h(i) \defeq \ell_d(i)$.
The volume of $i$ is defined as
\[ \vol(i) \defeq \prod_{i=1}^d \ell_j(i). \]
When $d=2$, we say that $i$ is a rectangular item.
Define the width of $i$ as $w(i) \defeq \ell_1(i)$
and the area of $i$ as $a(i) \defeq \vol(i) = w(i)h(i)$.
For 1D items (i.e., items in a classical bin packing instance),
$\vol$ is also called $\size$.

For a set $I$ of items, let $f: I \mapsto \mathbb{R}$ be a function
mapping each item to a real number.
Then for a set $X \subseteq I$, define $f(X) \defeq \sum_{i \in X} f(i)$
(unless stated otherwise).
For example, $\vol(X) \defeq \sum_{i \in X} \vol(i)$.

\subsection{Packing of Items}

Let $I$ be a set of items. Hence, $|I|$ is the number of items in $I$.
Let $P$ be a packing of items into bins. Define $|P|$ as the number of bins used by $P$.
Let $\Acal$ be an algorithm for bin packing.
Define $\Acal(I)$ as the packing of $I$ output by $\Acal$.
Hence, the number of bins used by $\Acal$ to pack $I$ is $|\Acal(I)|$.

Let $I$ be a bin packing instance.
Define $\opt(I)$ as the minimum number of bins needed to pack $I$.
Sometimes, we may be considering multiple versions of bin packing,
like 2D GBP with rotations (denoted as 2BPR)
and 2D GBP without rotations (denoted as 2BPNR).
Hence, to disambiguate, we will use the problem name in subscript,
e.g., $\opt_{\mathrm{2BPR}}(I)$ and $\opt_{\mathrm{2BPNR}}(I)$.
We will sometimes look at other kinds of packing problems,
like strip packing (SP) and knapsack (KS).
If the problem is clear from context, we will use $\opt(I)$
to denote the objective value of the optimal solution to $I$.
Otherwise, to disambiguate, we will use $\opt_{\mathrm{BP}}(I)$,
$\opt_{\mathrm{SP}}(I)$, $\opt_{\mathrm{KS}}(I)$, etc.

\section{Approximation Algorithms}

\subsection{Minimization Problems}
\label{sec:appx:min}

In a minimization problem, there is a set of feasible solutions associated with each input,
and each feasible solution has a cost associated with it.
Our task is to output a feasible solution of minimum cost for the given input.
For an input $I$, let $\opt(I)$ denote the cost of the minimum-cost feasible solution.
The bin packing problem is an example of a minimization problem.

Let $\Acal$ be an algorithm for the minimization problem.
Define $\Acal(I)$ as the output of $\Acal$ on input $I$.
Algorithm $\Acal$ is said to be $\alpha$-approximate iff
for each input $I$, the cost of $\Acal(I)$ is at most $\alpha\opt(I)$.
The minimum value $\alpha$ for which $\Acal$ is $\alpha$-approximate
is called the \emph{approximation ratio} of $\Acal$.
Note that $\alpha$ is always at least 1.

The algorithm $\Acal$ is called a PTAS (Polynomial-Time Approximation Scheme)
iff it takes a constant $\eps > 0$ as a parameter, is $(1+\eps)$-approximate,
and runs in time polynomial in $n$.
Note that $\eps$ is a constant, so $n^{1/\eps}$ is polynomial in $n$.
A PTAS is called an FPTAS iff it is a PTAS and runs in time polynomial
in both $n$ and $1/\eps$.

\subsection{Maximization Problems}
\label{sec:appx:max}

In a maximization problem, there is a set of feasible solutions associated with each input,
and each feasible solution has a score.
Our task is to output a feasible solution of maximum score for the given input.
For an input $I$, let $\opt(I)$ denote the score of the maximum-score feasible solution.
The knapsack problem is an example of a maximization problem.

Let $\Acal$ be an algorithm for the maximization problem.
Define $\Acal(I)$ as the output of $\Acal$ on input $I$.
Algorithm $\Acal$ is said to be $\alpha$-approximate iff
for each input $I$, the score of $\Acal(I)$ is at least $\opt(I)/\alpha$.
The minimum value $\alpha$ for which $\Acal$ is $\alpha$-approximate
is called the \emph{approximation ratio} of $\Acal$.
Note that $\alpha$ is always at least 1.

The algorithm $\Acal$ is called a PTAS (Polynomial-Time Approximation Scheme)
iff it takes a constant $\eps > 0$ as a parameter, is $(1+\eps)$-approximate,
and runs in time polynomial in $n$.
A PTAS is called an FPTAS iff it is a PTAS and runs in time polynomial
in both $n$ and $1/\eps$.

(Some authors use a slightly different definition of $\alpha$-approximate.
By this definition, algorithm $\Acal$ is $\alpha$-approximate iff
for each input $I$, the score of $\Acal(I)$ is at least $\alpha\opt(I)$.
Note that $\alpha$ is always at most 1 by this definition.)

\section{Simple Packing Algorithms}
\label{sec:preliminaries:simple-algos}

\begin{lemma}
\label{thm:next-fit}
Let $I$ be a classical bin packing instance. Then
$|\operatorname{Next-Fit}(I)| < 2\size(I) + 1$.
Equivalently, $|\operatorname{Next-Fit}(I)| \le \ceil{2\size(I)}$.
\end{lemma}
\begin{proof}
Assume $I$ is non-empty (otherwise, the lemma is trivially true).
Let $m \defeq |\operatorname{Next-Fit}(I)|$.
Let $J_j$ be the items packed into the $j\Th$ bin by Next-Fit.
For $j \le m-1$, by the definition of Next-Fit,
the first item in the $(j+1)\Th$ bin didn't fit into the $j\Th$ bin.
Therefore, $\size(J_j) + \size(J_{j+1}) > 1$. This gives us
\begin{align*}
& \size(I)
= \frac{\size(J_1) + \size(J_m)}{2} + \sum_{j=1}^{m-1} \frac{\size(J_j) + \size(J_{j+1})}{2}
> \sum_{j=1}^{m-1} \frac{1}{2} = \frac{m-1}{2}
\\ &\implies m < 2\size(I) + 1 \iff m \le \ceil{2\size(I)}.  \qedhere
\end{align*}
\end{proof}

\begin{lemma}
\label{thm:next-fit-small}
Let $I$ be a classical bin packing instance, where each item has size at most $\eps$. Then
$|\operatorname{Next-Fit}(I)| \le \ceil{\size(I)/(1-\eps)}$.
\end{lemma}
\begin{proof}
Assume $I$ is non-empty (otherwise, the lemma is trivially true).
Let $m \defeq |\operatorname{Next-Fit}(I)|$.
Let $J_j$ be the items packed into the $j\Th$ bin by Next-Fit.
For $j \le m-1$, we have $\size(J_j) > 1-\eps$.
Therefore,
\begin{align*}
& \size(I) > \sum_{j=1}^{m-1} \size(J_j) \ge (m-1)(1-\eps)
\\ &\implies m < 1 + \frac{\size(I)}{1-\eps}
\quad\iff m \le \ceil{\frac{\size(I)}{1-\eps}}.
\qedhere \end{align*}
\end{proof}

\begin{lemma}[NFDH for strip packing~\cite{coffman1980performance}]
\label{thm:nfdh-strip}
Let $I$ be a set of rectangular items of width at most 1.
Then $I$ can be packed (without rotation) into a rectangular bin of width 1
and height less than $2a(I) + max_{i \in I} h(i)$
using the Next-Fit Decreasing Height (NFDH) algorithm.
\end{lemma}
\begin{proof}
Let there be $p$ shelves output by NFDH.
Let $S_j$ be the items in the $j\Th$ shelf.
Let $h_j$ be the height of the $j\Th$ shelf.
Let $H$ be the sum of heights of all the shelves.

For $j \le p-1$, all items in the $j\Th$ shelf have height at least $h_{j+1}$.
The combined width of the items in $S_j$ and the first item in $S_{j+1}$ is more than 1.
Therefore, $a(S_j) + a(S_{j+1}) > h_{j+1}$. This gives us
\begin{align*}
& a(I) = \frac{a(S_1) + a(S_p)}{2} + \sum_{j=1}^{p-1} \frac{a(S_j) + a(S_{j+1})}{2}
> \sum_{j=1}^{p-1} \frac{h_{j+1}}{2} = \frac{H-h_1}{2}
\\ &\implies H < 2a(I) + h_1 = 2a(I) + \max_{i \in I} h(i).
\qedhere \end{align*}
\end{proof}

\begin{lemma}[NFDH for small items~\cite{coffman1980performance}]
\label{thm:nfdh-small}
Let $I$ be a set of rectangular items where each item has width at most $\delta_W$
and height at most $\delta_H$. Let there be a rectangular bin of width $W$ and height $H$.
If $a(I) \le (W - \delta_W)(H - \delta_H)$, then the Next-Fit Decreasing Height (NFDH)
algorithm can pack $I$ into the bin (without rotating the items).
\end{lemma}
\begin{proof}
NFDH packs the items into shelves of width $W$.
Let the number of shelves be $p$.
Let $h_j$ be the height of the $j\Th$ shelf.
Let $S_j$ be the items in the $j\Th$ shelf.
Let $\Htild$ be the total height of the shelves.
We need to prove that $\Htild \le H$.

For $j \le p-1$, all items in the $j\Th$ shelf have height at least $h_{j+1}$,
and the total width of the items in $S_j$ is more than $W-\delta_W$,
because otherwise we could have fit another item into the $j\Th$ shelf.
Therefore, $a(S_j) > (W-\delta_W)h_{j+1}$ for $j \le p-1$. This gives us
\begin{align*}
& (W - \delta_W)(H - \delta_H) \ge a(I) > \sum_{j=1}^{p-1} a(S_j)
\ge (W-\delta_W)\sum_{j=1}^{p-1} h_{j+1}
= (W-\delta_W)(\Htild - h_1)
\\ &\implies H - \delta_H > \Htild - h_1
\quad\implies \Htild < H - (\delta_H - h_1) \le H.
\qedhere \end{align*}
\end{proof}

\begin{lemma}[NFDH for bin packing]
\label{thm:nfdh-bin}
Let $I$ be a set of rectangular items of width and height at most 1.
Then the number of square bins of side length 1 that
Next-Fit Decreasing Height (NFDH) uses to pack $I$ is less than $4a(I) + 3$.
\end{lemma}
\begin{proof}
The bin packing version of NFDH first packs $I$ into shelves
and then packs the shelves into bins using Next-Fit.
Let $H$ be the sum of heights of all the shelves.

By \cref{thm:nfdh-strip}, we get $H < 2a(I) + 1$.
By \cref{thm:next-fit}, the number of bins needed is less than $2H + 1$.
Therefore, the number of bins needed is less than $4a(I) + 3$.
\end{proof}

\section{Configuration Linear Program}
\label{sec:config-lp}

Let $I$ be a bin packing instance.
We will now describe a linear program (LP) associated with $I$,
called the \emph{configuration LP}.
Configuration LPs are defined for a large class of bin packing problems,
like classical bin packing, geometric bin packing and vector bin packing.

For a set $I$ of items, define a configuration of $I$ as a non-empty subset of $I$
that can fit into a bin, and a way of packing those items into a bin.
An equivalent definition of bin packing is that we need to find
a small number of configurations such that each item belongs to
at least one of those configurations.

Let $C$ be a configuration. We say that $i \in C$ iff
item $i$ belongs to the set of items corresponding to $C$.
We can denote each configuration as a vector $C \in \{0, 1\}^n$,
where $C_i = 1$ iff $i \in C$.
Let $\Ccal$ be the set of all possible configurations.

We can express bin packing as the problem of finding integer solutions
to the following linear program, called the configuration LP,
where $x_C \in \{0, 1\}$ denotes the number of bins that have configuration $C$:
\[ \begin{array}{*3{>{\displaystyle}l}}
\min_{x \in \mathbb{R}^{|\Ccal|}} & \multicolumn{2}{>{\displaystyle}l}{\sum_{C \in \Ccal} x_C}
\\[1.5em] \textrm{where } & \sum_{C \ni i} x_C \ge 1 & \forall i \in [n]
\\[1em] \textrm{and } & x_C \ge 0 & \forall C \in \Ccal
\end{array} \]

Define the configuration matrix $A \in \{0, 1\}^{n \times |\Ccal|}$
as a matrix where $A[i, C]$ is 1 if configuration $C$ contains item $i$ and 0 otherwise.
We can also write the configuration LP as
\[ \min_{x \in \mathbb{R}^{|\mathcal{C}|}} \Sum(x)
\quad\textrm{where}\quad Ax \ge \mathbf{1} \quad\textrm{and}\quad x \ge 0. \]

Linear programs are a useful tool in the area of approximation algorithms.
Many optimization problems can be expressed as integer programs.
\emph{Rounding-based} algorithms first solve the LP relaxation of these integer programs,
and then \emph{round} the relaxed solution to get an approximate solution to the original problem
\cite{daa:det-lp-round,daa:rand-lp-round,iterative-methods}.
This approach has also been successfully applied to bin packing:
the Round-and-Approx framework~\cite{rna,bansal2014binpacking}
applies randomized rounding to a solution to the configuration LP.

We would like to solve the configuration LP in polynomial time.
However, the number of configurations can be up to $2^n-1$,
which is exponential in the input size.
One may wonder what it would even mean to solve the LP in polynomial time,
since even writing down a feasible solution to the LP can take exponential time!
The key observation (which we will soon show) is that
there exists an optimal solution to the configuration LP
whose support contains at most $n$ configurations.
Therefore, we can restrict ourselves to solutions of $\poly(n)$-sized support
without loss of generality. Such solutions can be written down in polynomial time.

\begin{definition}[Extreme point; Definition 1.2.1 in \cite{iterative-methods}]
Let $P \subseteq \mathbb{R}^d$ and $x \in P$.
$x$ is called an extreme point of $P$ iff there is no $y \neq 0$
such that $x + y \in P$ and $x - y \in P$.
\end{definition}
\begin{lemma}
If a linear program is bounded (i.e., it attains the optimal objective value at some point),
then it contains an optimal solution that is an extreme point of the set of all feasible solutions.
\end{lemma}
\begin{lemma}[Rank Lemma; Lemma 2.1.4 in \cite{iterative-methods}]
Let $P \defeq \{x: Ax \ge b \textrm{ and } x \ge 0\}$. Let $\xhat$ be an extreme point of $P$.
Let $S \defeq \{a_i: (A\xhat)_i = b_i\}$, where $a_i$ is the $i\Th$ row of $A$.
Then $|\support(\xhat)|$ is equal to the maximal number of linearly independent vectors in $S$.
\end{lemma}
\begin{corollary}
\label{thm:rank-lemma-corr}
Let $P \defeq \{x: Ax \ge b \textrm{ and } x \ge 0\}$ and let $\xhat$ be an extreme point of $P$.
Then $|\support(\xhat)|$ is at most the number of inequalities in $Ax \ge b$.
\end{corollary}
By \cref{thm:rank-lemma-corr}, we get that there exist extreme point optimal solutions
to the configuration LP, and these solutions have at most $n$ configurations in their support.

\subsection{Solving the Configuration LP}

Karmarkar and Karp~\cite{karmarkar-karp} gave an FPTAS for solving the configuration LP
of any classical bin packing instance.
They do this by slightly modifying the Ellipsoid algorithm of
Gr\"otschel, Lov\'asz and Schrijver~\cite{gls-ellipsoid}.

The algorithm of Plotkin, Shmoys and Tardos~\cite{plotkin1995fast} can be used to
obtain a PTAS for the configuration LP of any variant of bin packing,
given a PTAS for the corresponding knapsack problem.
Bansal, Caprara, Jansen, Pr\"adel and Sviridenko~\cite{bansal2009structural}
showed how to use this to solve the configuration LP for 2D GBP
(both the rotational and non-rotational versions).
