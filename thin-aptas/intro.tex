For a constant $\delta > 0$, a rectangle is said to be $\delta$-thin
iff either its width is at most $\delta$ or its height is at most $\delta$.
In this chapter, we give something like an APTAS for packing thin rectangles into bins.
Formally, we give an algorithm for 2D GBP, called $\thinCPack$
(named after `thin compartmental packing'),
that accepts a parameter $\eps$, and we show that for every constant $\eps \in (0, 1)$,
there exists a constant $\delta \in (0, \eps)$ such that $\thinCPack$ has an AAR of $1+\eps$
when all items in the input are $\delta$-thin rectangles.

This contrasts the fact that an APTAS doesn't exist for 2D GBP~\cite{bansal2006bin}.
This indicates that to improve upon algorithms for 2D GBP,
we should focus on big rectangles, i.e., rectangles whose
width and height are both more than a constant $\delta$.

\subsection*{Overview of the Algorithm}

$\thinCPack$ takes a set $I$ of items as input and has the following outline:
\begin{enumerate}
\item Invoke the subroutine $\round(I)$. This returns a pair $(\Itild, D)$,
    where $D \subseteq I$, $D$ has a very small area,
    and $\Itild$ is obtained by increasing the width or height of each item in $I - D$.
    The set $\Itild$ has special properties that help us easily pack it.
\item Compute the optimal \emph{compartmental fractional} bin packing of $\Itild$.
    (We will define \emph{compartmental} and \emph{fractional} later).
\item Use this packing of $\Itild$ to obtain a packing of $I - D$
    that uses slightly more number of bins.
\item Pack $D$ into bins using the Next-Fit Decreasing Height algorithm~%
\cite{coffman1980performance}.
\end{enumerate}

Let $\opt(I)$ be the minimum number of bins needed to pack $I$.
To bound the AAR of $\thinCPack$, we will prove a structural theorem,
i.e., we will prove that the optimal fractional compartmental packing of $\Itild$
uses close to $\opt(I)$ bins.
