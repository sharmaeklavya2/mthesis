\section{Packing Rounded Items}
\label{sec:thin-bp:algo}

Let $I$ be a set of $\delta$-\thin{} items. In this section,
we give an algorithm for computing a near-optimal packing of $I$, called $\thinCPack$.
Roughly, $\thinCPack$ first computes $(\Itild, \Imed) \defeq \round(I)$.
It then computes the optimal fractional compartmental packing of $\Itild$
by first guessing a packing of empty compartments into bins
and then fractionally packing the wide and tall items into the compartments.
It then converts the fractional packing of $\Itild$ to a non-fractional packing of $I$
with only a tiny increase in the number of bins.
See \cref{fig:thincpack} for a visual overview of $\thinCPack$.

\begin{figure}[htb]
\hbadness=10000
\begin{subfigure}[t]{0.3\textwidth}
\centering
\tikzset{compartment/.style={draw,thick,fill={textColor!10!bgColor}},
container/.style={},item/.style={}}
\ifcsname pGameL\endcsname\else\newlength{\myu}\fi
\setlength{\myu}{0.8cm}
\tikzset{pic1/.pic={
    \path[item]
        (0.00\myu, 0\myu) rectangle +(0.09\myu, 0.6\myu)
        (0.09\myu, 0\myu) rectangle +(0.10\myu, 0.6\myu)
        (0.19\myu, 0\myu) rectangle +(0.08\myu, 0.6\myu)
        (0.27\myu, 0\myu) rectangle +(0.08\myu, 0.6\myu);
    \path[item]
        (0\myu, 0.6\myu) rectangle +(0.10\myu, 0.4\myu)
        (0.10\myu, 0.6\myu) rectangle +(0.07\myu, 0.4\myu)
        (0.17\myu, 0.6\myu) rectangle +(0.07\myu, 0.4\myu)
        (0.24\myu, 0.6\myu) rectangle +(0.08\myu, 0.4\myu)
        (0.32\myu, 0.6\myu) rectangle +(0.07\myu, 0.4\myu);
    \path[item]
        (0.40\myu, 0\myu) rectangle +(0.11\myu, 0.5\myu)
        (0.50\myu, 0\myu) rectangle +(0.09\myu, 0.5\myu)
        (0.59\myu, 0\myu) rectangle +(0.09\myu, 0.5\myu);
    \path[item]
        (0.40\myu, 0.5\myu) rectangle +(0.08\myu, 0.5\myu)
        (0.48\myu, 0.5\myu) rectangle +(0.09\myu, 0.5\myu)
        (0.57\myu, 0.5\myu) rectangle +(0.08\myu, 0.5\myu);
    \path[item]
        (0.70\myu, 0\myu) rectangle +(0.1\myu, 0.4\myu)
        (0.80\myu, 0\myu) rectangle +(0.1\myu, 0.4\myu)
        (0.90\myu, 0\myu) rectangle +(0.1\myu, 0.4\myu);
    \path[item]
        (0.70\myu, 0.4\myu) rectangle +(0.07\myu, 0.3\myu)
        (0.77\myu, 0.4\myu) rectangle +(0.07\myu, 0.3\myu)
        (0.84\myu, 0.4\myu) rectangle +(0.07\myu, 0.3\myu)
        (0.91\myu, 0.4\myu) rectangle +(0.07\myu, 0.3\myu);
    \path[item]
        (0.70\myu, 0.7\myu) rectangle +(0.08\myu, 0.3\myu)
        (0.78\myu, 0.7\myu) rectangle +(0.08\myu, 0.3\myu)
        (0.86\myu, 0.7\myu) rectangle +(0.09\myu, 0.3\myu);
    \path[container]
        (0\myu, 0\myu) rectangle +(0.4\myu, 0.6\myu)
        (0\myu, 0.6\myu) rectangle +(0.4\myu, 0.4\myu)
        (0.4\myu, 0\myu) rectangle +(0.3\myu, 0.5\myu)
        (0.4\myu, 0.5\myu) rectangle +(0.3\myu, 0.5\myu)
        (0.7\myu, 0\myu) rectangle +(0.3\myu, 0.4\myu)
        (0.7\myu, 0.4\myu) rectangle +(0.3\myu, 0.3\myu)
        (0.7\myu, 0.7\myu) rectangle +(0.3\myu, 0.3\myu);
    \path[compartment] (0\myu, 0\myu) rectangle (1\myu, 1\myu);
}}
\tikzset{pic2/.pic={
    \path[item]
        (0.0\myu, 0\myu) rectangle +(0.1\myu, 1\myu)
        (0.1\myu, 0\myu) rectangle +(0.1\myu, 1\myu)
        (0.2\myu, 0\myu) rectangle +(0.1\myu, 1\myu)
        (0.3\myu, 0\myu) rectangle +(0.1\myu, 1\myu)
        (0.4\myu, 0\myu) rectangle +(0.1\myu, 1\myu);
    \path[item]
        (0.50\myu, 0\myu) rectangle +(0.08\myu, 0.5\myu)
        (0.58\myu, 0\myu) rectangle +(0.08\myu, 0.5\myu)
        (0.66\myu, 0\myu) rectangle +(0.08\myu, 0.5\myu)
        (0.74\myu, 0\myu) rectangle +(0.08\myu, 0.5\myu)
        (0.82\myu, 0\myu) rectangle +(0.08\myu, 0.5\myu)
        (0.90\myu, 0\myu) rectangle +(0.08\myu, 0.5\myu);
    \path[item]
        (0.50\myu, 0.5\myu) rectangle +(0.09\myu, 0.5\myu)
        (0.59\myu, 0.5\myu) rectangle +(0.09\myu, 0.5\myu)
        (0.68\myu, 0.5\myu) rectangle +(0.09\myu, 0.5\myu)
        (0.77\myu, 0.5\myu) rectangle +(0.09\myu, 0.5\myu)
        (0.86\myu, 0.5\myu) rectangle +(0.09\myu, 0.5\myu);
    \path[container]
        (0\myu, 0\myu) rectangle +(0.5\myu, 1\myu)
        (0.5\myu, 0\myu) rectangle +(0.5\myu, 0.5\myu)
        (0.5\myu, 0.5\myu) rectangle +(0.5\myu, 0.5\myu);
    \path[compartment] (0\myu, 0\myu) rectangle (1\myu, 1\myu);
}}
\tikzset{pic3/.pic={
    \path[item]
        (0.0\myu, 0\myu) rectangle +(0.1\myu, 0.4\myu)
        (0.1\myu, 0\myu) rectangle +(0.1\myu, 0.4\myu)
        (0.2\myu, 0\myu) rectangle +(0.1\myu, 0.4\myu)
        (0.3\myu, 0\myu) rectangle +(0.1\myu, 0.4\myu);
    \path[item]
        (0.00\myu, 0.4\myu) rectangle +(0.09\myu, 0.3\myu)
        (0.09\myu, 0.4\myu) rectangle +(0.09\myu, 0.3\myu)
        (0.18\myu, 0.4\myu) rectangle +(0.09\myu, 0.3\myu)
        (0.27\myu, 0.4\myu) rectangle +(0.09\myu, 0.3\myu);
    \path[item]
        (0.00\myu, 0.7\myu) rectangle +(0.08\myu, 0.3\myu)
        (0.08\myu, 0.7\myu) rectangle +(0.08\myu, 0.3\myu)
        (0.16\myu, 0.7\myu) rectangle +(0.08\myu, 0.3\myu)
        (0.24\myu, 0.7\myu) rectangle +(0.08\myu, 0.3\myu)
        (0.32\myu, 0.7\myu) rectangle +(0.08\myu, 0.3\myu);
    \path[item]
        (0.4\myu, 0\myu) rectangle +(0.1\myu, 0.5\myu)
        (0.5\myu, 0\myu) rectangle +(0.1\myu, 0.5\myu)
        (0.6\myu, 0\myu) rectangle +(0.1\myu, 0.5\myu)
        (0.7\myu, 0\myu) rectangle +(0.1\myu, 0.5\myu)
        (0.8\myu, 0\myu) rectangle +(0.1\myu, 0.5\myu)
        (0.9\myu, 0\myu) rectangle +(0.1\myu, 0.5\myu);
    \path[item]
        (0.40\myu, 0.5\myu) rectangle +(0.07\myu, 0.5\myu)
        (0.47\myu, 0.5\myu) rectangle +(0.07\myu, 0.5\myu)
        (0.54\myu, 0.5\myu) rectangle +(0.09\myu, 0.5\myu)
        (0.63\myu, 0.5\myu) rectangle +(0.09\myu, 0.5\myu)
        (0.72\myu, 0.5\myu) rectangle +(0.08\myu, 0.5\myu)
        (0.80\myu, 0.5\myu) rectangle +(0.08\myu, 0.5\myu)
        (0.88\myu, 0.5\myu) rectangle +(0.08\myu, 0.5\myu);
    \path[container]
        (0\myu, 0\myu) rectangle +(0.4\myu, 0.4\myu)
        (0\myu, 0.4\myu) rectangle +(0.4\myu, 0.3\myu)
        (0\myu, 0.7\myu) rectangle +(0.4\myu, 0.3\myu)
        (0.4\myu, 0\myu) rectangle +(0.6\myu, 0.5\myu)
        (0.4\myu, 0.5\myu) rectangle +(0.6\myu, 0.5\myu);
    \path[compartment] (0\myu, 0\myu) rectangle (1\myu, 1\myu);
}}
\tikzset{pic4/.pic={
    \path[item] foreach \h/\y in {0.15/0.00,0.14/0.15,0.13/0.29,0.12/0.42,0.11/0.54,
            0.10/0.65,0.09/0.75,0.08/0.84,0.07/0.92}{
         foreach \hdiff/\w/\x in {0.000/0.15/0.00,0.001/0.14/0.15,0.002/0.13/0.29,0.003/0.12/0.42,
                0.004/0.11/0.54,0.005/0.10/0.65,0.006/0.09/0.75,0.007/0.08/0.84,0.008/0.07/0.92}{
            (\x\myu, \y\myu) rectangle +(\w\myu, \h\myu-\hdiff\myu)
    }};
}}
\begin{tikzpicture}
\pic at (2\myu, 0\myu) {pic4};

\pic[xscale=-3,rotate=90] at (2\myu, 2\myu) {pic1};
\pic[xscale=-3,rotate=90] at (1\myu, 4\myu) {pic2};

\pic[xscale=-2,rotate=90] at (1\myu, 1\myu) {pic3};
\pic[xscale=-2,rotate=90] at (2\myu, 3\myu) {pic1};
\pic[xscale=-2,rotate=90] at (0\myu, 0\myu) {pic2};

\pic[yscale=2] at (1\myu, 2\myu) {pic1};
\pic[yscale=2] at (3\myu, 0\myu) {pic2};
\pic[yscale=2] at (4\myu, 0\myu) {pic3};
\pic[yscale=2] at (4\myu, 3\myu) {pic1};
\pic[yscale=4] at (0\myu, 1\myu) {pic2};

\draw[ultra thick] (0\myu, 0\myu) rectangle (5\myu, 5\myu);
\end{tikzpicture}

\caption{Guess the packing of empty compartments in each bin (\cref{sec:enum-configs}).}
\end{subfigure}
\hfill
\begin{subfigure}[t]{0.3\textwidth}
\centering
\tikzset{compartment/.style={draw,thick},
container/.style={draw,fill={textColor!25!bgColor}},
item/.style={}}
\ifcsname pGameL\endcsname\else\newlength{\myu}\fi
\setlength{\myu}{0.8cm}
\tikzset{pic1/.pic={
    \path[item]
        (0.00\myu, 0\myu) rectangle +(0.09\myu, 0.6\myu)
        (0.09\myu, 0\myu) rectangle +(0.10\myu, 0.6\myu)
        (0.19\myu, 0\myu) rectangle +(0.08\myu, 0.6\myu)
        (0.27\myu, 0\myu) rectangle +(0.08\myu, 0.6\myu);
    \path[item]
        (0\myu, 0.6\myu) rectangle +(0.10\myu, 0.4\myu)
        (0.10\myu, 0.6\myu) rectangle +(0.07\myu, 0.4\myu)
        (0.17\myu, 0.6\myu) rectangle +(0.07\myu, 0.4\myu)
        (0.24\myu, 0.6\myu) rectangle +(0.08\myu, 0.4\myu)
        (0.32\myu, 0.6\myu) rectangle +(0.07\myu, 0.4\myu);
    \path[item]
        (0.40\myu, 0\myu) rectangle +(0.11\myu, 0.5\myu)
        (0.50\myu, 0\myu) rectangle +(0.09\myu, 0.5\myu)
        (0.59\myu, 0\myu) rectangle +(0.09\myu, 0.5\myu);
    \path[item]
        (0.40\myu, 0.5\myu) rectangle +(0.08\myu, 0.5\myu)
        (0.48\myu, 0.5\myu) rectangle +(0.09\myu, 0.5\myu)
        (0.57\myu, 0.5\myu) rectangle +(0.08\myu, 0.5\myu);
    \path[item]
        (0.70\myu, 0\myu) rectangle +(0.1\myu, 0.4\myu)
        (0.80\myu, 0\myu) rectangle +(0.1\myu, 0.4\myu)
        (0.90\myu, 0\myu) rectangle +(0.1\myu, 0.4\myu);
    \path[item]
        (0.70\myu, 0.4\myu) rectangle +(0.07\myu, 0.3\myu)
        (0.77\myu, 0.4\myu) rectangle +(0.07\myu, 0.3\myu)
        (0.84\myu, 0.4\myu) rectangle +(0.07\myu, 0.3\myu)
        (0.91\myu, 0.4\myu) rectangle +(0.07\myu, 0.3\myu);
    \path[item]
        (0.70\myu, 0.7\myu) rectangle +(0.08\myu, 0.3\myu)
        (0.78\myu, 0.7\myu) rectangle +(0.08\myu, 0.3\myu)
        (0.86\myu, 0.7\myu) rectangle +(0.09\myu, 0.3\myu);
    \path[container]
        (0\myu, 0\myu) rectangle +(0.4\myu, 0.6\myu)
        (0\myu, 0.6\myu) rectangle +(0.4\myu, 0.4\myu)
        (0.4\myu, 0\myu) rectangle +(0.3\myu, 0.5\myu)
        (0.4\myu, 0.5\myu) rectangle +(0.3\myu, 0.5\myu)
        (0.7\myu, 0\myu) rectangle +(0.3\myu, 0.4\myu)
        (0.7\myu, 0.4\myu) rectangle +(0.3\myu, 0.3\myu)
        (0.7\myu, 0.7\myu) rectangle +(0.3\myu, 0.3\myu);
    \path[compartment] (0\myu, 0\myu) rectangle (1\myu, 1\myu);
}}
\tikzset{pic2/.pic={
    \path[item]
        (0.0\myu, 0\myu) rectangle +(0.1\myu, 1\myu)
        (0.1\myu, 0\myu) rectangle +(0.1\myu, 1\myu)
        (0.2\myu, 0\myu) rectangle +(0.1\myu, 1\myu)
        (0.3\myu, 0\myu) rectangle +(0.1\myu, 1\myu)
        (0.4\myu, 0\myu) rectangle +(0.1\myu, 1\myu);
    \path[item]
        (0.50\myu, 0\myu) rectangle +(0.08\myu, 0.5\myu)
        (0.58\myu, 0\myu) rectangle +(0.08\myu, 0.5\myu)
        (0.66\myu, 0\myu) rectangle +(0.08\myu, 0.5\myu)
        (0.74\myu, 0\myu) rectangle +(0.08\myu, 0.5\myu)
        (0.82\myu, 0\myu) rectangle +(0.08\myu, 0.5\myu)
        (0.90\myu, 0\myu) rectangle +(0.08\myu, 0.5\myu);
    \path[item]
        (0.50\myu, 0.5\myu) rectangle +(0.09\myu, 0.5\myu)
        (0.59\myu, 0.5\myu) rectangle +(0.09\myu, 0.5\myu)
        (0.68\myu, 0.5\myu) rectangle +(0.09\myu, 0.5\myu)
        (0.77\myu, 0.5\myu) rectangle +(0.09\myu, 0.5\myu)
        (0.86\myu, 0.5\myu) rectangle +(0.09\myu, 0.5\myu);
    \path[container]
        (0\myu, 0\myu) rectangle +(0.5\myu, 1\myu)
        (0.5\myu, 0\myu) rectangle +(0.5\myu, 0.5\myu)
        (0.5\myu, 0.5\myu) rectangle +(0.5\myu, 0.5\myu);
    \path[compartment] (0\myu, 0\myu) rectangle (1\myu, 1\myu);
}}
\tikzset{pic3/.pic={
    \path[item]
        (0.0\myu, 0\myu) rectangle +(0.1\myu, 0.4\myu)
        (0.1\myu, 0\myu) rectangle +(0.1\myu, 0.4\myu)
        (0.2\myu, 0\myu) rectangle +(0.1\myu, 0.4\myu)
        (0.3\myu, 0\myu) rectangle +(0.1\myu, 0.4\myu);
    \path[item]
        (0.00\myu, 0.4\myu) rectangle +(0.09\myu, 0.3\myu)
        (0.09\myu, 0.4\myu) rectangle +(0.09\myu, 0.3\myu)
        (0.18\myu, 0.4\myu) rectangle +(0.09\myu, 0.3\myu)
        (0.27\myu, 0.4\myu) rectangle +(0.09\myu, 0.3\myu);
    \path[item]
        (0.00\myu, 0.7\myu) rectangle +(0.08\myu, 0.3\myu)
        (0.08\myu, 0.7\myu) rectangle +(0.08\myu, 0.3\myu)
        (0.16\myu, 0.7\myu) rectangle +(0.08\myu, 0.3\myu)
        (0.24\myu, 0.7\myu) rectangle +(0.08\myu, 0.3\myu)
        (0.32\myu, 0.7\myu) rectangle +(0.08\myu, 0.3\myu);
    \path[item]
        (0.4\myu, 0\myu) rectangle +(0.1\myu, 0.5\myu)
        (0.5\myu, 0\myu) rectangle +(0.1\myu, 0.5\myu)
        (0.6\myu, 0\myu) rectangle +(0.1\myu, 0.5\myu)
        (0.7\myu, 0\myu) rectangle +(0.1\myu, 0.5\myu)
        (0.8\myu, 0\myu) rectangle +(0.1\myu, 0.5\myu)
        (0.9\myu, 0\myu) rectangle +(0.1\myu, 0.5\myu);
    \path[item]
        (0.40\myu, 0.5\myu) rectangle +(0.07\myu, 0.5\myu)
        (0.47\myu, 0.5\myu) rectangle +(0.07\myu, 0.5\myu)
        (0.54\myu, 0.5\myu) rectangle +(0.09\myu, 0.5\myu)
        (0.63\myu, 0.5\myu) rectangle +(0.09\myu, 0.5\myu)
        (0.72\myu, 0.5\myu) rectangle +(0.08\myu, 0.5\myu)
        (0.80\myu, 0.5\myu) rectangle +(0.08\myu, 0.5\myu)
        (0.88\myu, 0.5\myu) rectangle +(0.08\myu, 0.5\myu);
    \path[container]
        (0\myu, 0\myu) rectangle +(0.4\myu, 0.4\myu)
        (0\myu, 0.4\myu) rectangle +(0.4\myu, 0.3\myu)
        (0\myu, 0.7\myu) rectangle +(0.4\myu, 0.3\myu)
        (0.4\myu, 0\myu) rectangle +(0.6\myu, 0.5\myu)
        (0.4\myu, 0.5\myu) rectangle +(0.6\myu, 0.5\myu);
    \path[compartment] (0\myu, 0\myu) rectangle (1\myu, 1\myu);
}}
\tikzset{pic4/.pic={
    \path[item] foreach \h/\y in {0.15/0.00,0.14/0.15,0.13/0.29,0.12/0.42,0.11/0.54,
            0.10/0.65,0.09/0.75,0.08/0.84,0.07/0.92}{
         foreach \hdiff/\w/\x in {0.000/0.15/0.00,0.001/0.14/0.15,0.002/0.13/0.29,0.003/0.12/0.42,
                0.004/0.11/0.54,0.005/0.10/0.65,0.006/0.09/0.75,0.007/0.08/0.84,0.008/0.07/0.92}{
            (\x\myu, \y\myu) rectangle +(\w\myu, \h\myu-\hdiff\myu)
    }};
}}
\begin{tikzpicture}
\pic at (2\myu, 0\myu) {pic4};

\pic[xscale=-3,rotate=90] at (2\myu, 2\myu) {pic1};
\pic[xscale=-3,rotate=90] at (1\myu, 4\myu) {pic2};

\pic[xscale=-2,rotate=90] at (1\myu, 1\myu) {pic3};
\pic[xscale=-2,rotate=90] at (2\myu, 3\myu) {pic1};
\pic[xscale=-2,rotate=90] at (0\myu, 0\myu) {pic2};

\pic[yscale=2] at (1\myu, 2\myu) {pic1};
\pic[yscale=2] at (3\myu, 0\myu) {pic2};
\pic[yscale=2] at (4\myu, 0\myu) {pic3};
\pic[yscale=2] at (4\myu, 3\myu) {pic1};
\pic[yscale=4] at (0\myu, 1\myu) {pic2};

\draw[ultra thick] (0\myu, 0\myu) rectangle (5\myu, 5\myu);
\end{tikzpicture}

\caption{Fractionally pack wide and tall items into compartments.
This partitions each compartment into \emph{containers} (\cref{sec:feas-lp}).}
\end{subfigure}
\hfill
\begin{subfigure}[t]{0.3\textwidth}
\centering
\tikzset{compartment/.style={draw,ultra thick},
container/.style={draw,thick},
item/.style={draw,very thin,fill={textColor!25!bgColor}}}
\ifcsname pGameL\endcsname\else\newlength{\myu}\fi
\setlength{\myu}{0.8cm}
\tikzset{pic1/.pic={
    \path[item]
        (0.00\myu, 0\myu) rectangle +(0.09\myu, 0.6\myu)
        (0.09\myu, 0\myu) rectangle +(0.10\myu, 0.6\myu)
        (0.19\myu, 0\myu) rectangle +(0.08\myu, 0.6\myu)
        (0.27\myu, 0\myu) rectangle +(0.08\myu, 0.6\myu);
    \path[item]
        (0\myu, 0.6\myu) rectangle +(0.10\myu, 0.4\myu)
        (0.10\myu, 0.6\myu) rectangle +(0.07\myu, 0.4\myu)
        (0.17\myu, 0.6\myu) rectangle +(0.07\myu, 0.4\myu)
        (0.24\myu, 0.6\myu) rectangle +(0.08\myu, 0.4\myu)
        (0.32\myu, 0.6\myu) rectangle +(0.07\myu, 0.4\myu);
    \path[item]
        (0.40\myu, 0\myu) rectangle +(0.11\myu, 0.5\myu)
        (0.50\myu, 0\myu) rectangle +(0.09\myu, 0.5\myu)
        (0.59\myu, 0\myu) rectangle +(0.09\myu, 0.5\myu);
    \path[item]
        (0.40\myu, 0.5\myu) rectangle +(0.08\myu, 0.5\myu)
        (0.48\myu, 0.5\myu) rectangle +(0.09\myu, 0.5\myu)
        (0.57\myu, 0.5\myu) rectangle +(0.08\myu, 0.5\myu);
    \path[item]
        (0.70\myu, 0\myu) rectangle +(0.1\myu, 0.4\myu)
        (0.80\myu, 0\myu) rectangle +(0.1\myu, 0.4\myu)
        (0.90\myu, 0\myu) rectangle +(0.1\myu, 0.4\myu);
    \path[item]
        (0.70\myu, 0.4\myu) rectangle +(0.07\myu, 0.3\myu)
        (0.77\myu, 0.4\myu) rectangle +(0.07\myu, 0.3\myu)
        (0.84\myu, 0.4\myu) rectangle +(0.07\myu, 0.3\myu)
        (0.91\myu, 0.4\myu) rectangle +(0.07\myu, 0.3\myu);
    \path[item]
        (0.70\myu, 0.7\myu) rectangle +(0.08\myu, 0.3\myu)
        (0.78\myu, 0.7\myu) rectangle +(0.08\myu, 0.3\myu)
        (0.86\myu, 0.7\myu) rectangle +(0.09\myu, 0.3\myu);
    \path[container]
        (0\myu, 0\myu) rectangle +(0.4\myu, 0.6\myu)
        (0\myu, 0.6\myu) rectangle +(0.4\myu, 0.4\myu)
        (0.4\myu, 0\myu) rectangle +(0.3\myu, 0.5\myu)
        (0.4\myu, 0.5\myu) rectangle +(0.3\myu, 0.5\myu)
        (0.7\myu, 0\myu) rectangle +(0.3\myu, 0.4\myu)
        (0.7\myu, 0.4\myu) rectangle +(0.3\myu, 0.3\myu)
        (0.7\myu, 0.7\myu) rectangle +(0.3\myu, 0.3\myu);
    \path[compartment] (0\myu, 0\myu) rectangle (1\myu, 1\myu);
}}
\tikzset{pic2/.pic={
    \path[item]
        (0.0\myu, 0\myu) rectangle +(0.1\myu, 1\myu)
        (0.1\myu, 0\myu) rectangle +(0.1\myu, 1\myu)
        (0.2\myu, 0\myu) rectangle +(0.1\myu, 1\myu)
        (0.3\myu, 0\myu) rectangle +(0.1\myu, 1\myu)
        (0.4\myu, 0\myu) rectangle +(0.1\myu, 1\myu);
    \path[item]
        (0.50\myu, 0\myu) rectangle +(0.08\myu, 0.5\myu)
        (0.58\myu, 0\myu) rectangle +(0.08\myu, 0.5\myu)
        (0.66\myu, 0\myu) rectangle +(0.08\myu, 0.5\myu)
        (0.74\myu, 0\myu) rectangle +(0.08\myu, 0.5\myu)
        (0.82\myu, 0\myu) rectangle +(0.08\myu, 0.5\myu)
        (0.90\myu, 0\myu) rectangle +(0.08\myu, 0.5\myu);
    \path[item]
        (0.50\myu, 0.5\myu) rectangle +(0.09\myu, 0.5\myu)
        (0.59\myu, 0.5\myu) rectangle +(0.09\myu, 0.5\myu)
        (0.68\myu, 0.5\myu) rectangle +(0.09\myu, 0.5\myu)
        (0.77\myu, 0.5\myu) rectangle +(0.09\myu, 0.5\myu)
        (0.86\myu, 0.5\myu) rectangle +(0.09\myu, 0.5\myu);
    \path[container]
        (0\myu, 0\myu) rectangle +(0.5\myu, 1\myu)
        (0.5\myu, 0\myu) rectangle +(0.5\myu, 0.5\myu)
        (0.5\myu, 0.5\myu) rectangle +(0.5\myu, 0.5\myu);
    \path[compartment] (0\myu, 0\myu) rectangle (1\myu, 1\myu);
}}
\tikzset{pic3/.pic={
    \path[item]
        (0.0\myu, 0\myu) rectangle +(0.1\myu, 0.4\myu)
        (0.1\myu, 0\myu) rectangle +(0.1\myu, 0.4\myu)
        (0.2\myu, 0\myu) rectangle +(0.1\myu, 0.4\myu)
        (0.3\myu, 0\myu) rectangle +(0.1\myu, 0.4\myu);
    \path[item]
        (0.00\myu, 0.4\myu) rectangle +(0.09\myu, 0.3\myu)
        (0.09\myu, 0.4\myu) rectangle +(0.09\myu, 0.3\myu)
        (0.18\myu, 0.4\myu) rectangle +(0.09\myu, 0.3\myu)
        (0.27\myu, 0.4\myu) rectangle +(0.09\myu, 0.3\myu);
    \path[item]
        (0.00\myu, 0.7\myu) rectangle +(0.08\myu, 0.3\myu)
        (0.08\myu, 0.7\myu) rectangle +(0.08\myu, 0.3\myu)
        (0.16\myu, 0.7\myu) rectangle +(0.08\myu, 0.3\myu)
        (0.24\myu, 0.7\myu) rectangle +(0.08\myu, 0.3\myu)
        (0.32\myu, 0.7\myu) rectangle +(0.08\myu, 0.3\myu);
    \path[item]
        (0.4\myu, 0\myu) rectangle +(0.1\myu, 0.5\myu)
        (0.5\myu, 0\myu) rectangle +(0.1\myu, 0.5\myu)
        (0.6\myu, 0\myu) rectangle +(0.1\myu, 0.5\myu)
        (0.7\myu, 0\myu) rectangle +(0.1\myu, 0.5\myu)
        (0.8\myu, 0\myu) rectangle +(0.1\myu, 0.5\myu)
        (0.9\myu, 0\myu) rectangle +(0.1\myu, 0.5\myu);
    \path[item]
        (0.40\myu, 0.5\myu) rectangle +(0.07\myu, 0.5\myu)
        (0.47\myu, 0.5\myu) rectangle +(0.07\myu, 0.5\myu)
        (0.54\myu, 0.5\myu) rectangle +(0.09\myu, 0.5\myu)
        (0.63\myu, 0.5\myu) rectangle +(0.09\myu, 0.5\myu)
        (0.72\myu, 0.5\myu) rectangle +(0.08\myu, 0.5\myu)
        (0.80\myu, 0.5\myu) rectangle +(0.08\myu, 0.5\myu)
        (0.88\myu, 0.5\myu) rectangle +(0.08\myu, 0.5\myu);
    \path[container]
        (0\myu, 0\myu) rectangle +(0.4\myu, 0.4\myu)
        (0\myu, 0.4\myu) rectangle +(0.4\myu, 0.3\myu)
        (0\myu, 0.7\myu) rectangle +(0.4\myu, 0.3\myu)
        (0.4\myu, 0\myu) rectangle +(0.6\myu, 0.5\myu)
        (0.4\myu, 0.5\myu) rectangle +(0.6\myu, 0.5\myu);
    \path[compartment] (0\myu, 0\myu) rectangle (1\myu, 1\myu);
}}
\tikzset{pic4/.pic={
    \path[item] foreach \h/\y in {0.15/0.00,0.14/0.15,0.13/0.29,0.12/0.42,0.11/0.54,
            0.10/0.65,0.09/0.75,0.08/0.84,0.07/0.92}{
         foreach \hdiff/\w/\x in {0.000/0.15/0.00,0.001/0.14/0.15,0.002/0.13/0.29,0.003/0.12/0.42,
                0.004/0.11/0.54,0.005/0.10/0.65,0.006/0.09/0.75,0.007/0.08/0.84,0.008/0.07/0.92}{
            (\x\myu, \y\myu) rectangle +(\w\myu, \h\myu-\hdiff\myu)
    }};
}}
\begin{tikzpicture}
\pic at (2\myu, 0\myu) {pic4};

\pic[xscale=-3,rotate=90] at (2\myu, 2\myu) {pic1};
\pic[xscale=-3,rotate=90] at (1\myu, 4\myu) {pic2};

\pic[xscale=-2,rotate=90] at (1\myu, 1\myu) {pic3};
\pic[xscale=-2,rotate=90] at (2\myu, 3\myu) {pic1};
\pic[xscale=-2,rotate=90] at (0\myu, 0\myu) {pic2};

\pic[yscale=2] at (1\myu, 2\myu) {pic1};
\pic[yscale=2] at (3\myu, 0\myu) {pic2};
\pic[yscale=2] at (4\myu, 0\myu) {pic3};
\pic[yscale=2] at (4\myu, 3\myu) {pic1};
\pic[yscale=4] at (0\myu, 1\myu) {pic2};

\draw[ultra thick] (0\myu, 0\myu) rectangle (5\myu, 5\myu);
\end{tikzpicture}

\caption{Pack the items non-fractionally (\cref{sec:greedy-cont}).}
\end{subfigure}
\caption{Major steps of $\thinCPack$ after $\round$ing $I$.}
\label{fig:thincpack}
\end{figure}

\subsection{Enumerating Packing of Compartments}
\label{sec:enum-configs}

We will compute the optimal fractional compartmental packing of $\Itild$ in two steps.
First, for each bin, we will guess the compartments in the bin.
Each such packing of compartments into bins is called a \emph{configuration}.
Then we will fractionally pack the items into the compartments.

There can be at most $\nW \defeq 3(1/\epsLarge-1)|\Tcal| + 1$ wide compartments in a bin.
Each wide compartment can have $(1/\epsCont)^2$ $y$-coordinates
of the top and bottom edges and at most $|\Tcal|^2/2$ $x$-coordinates
of the left and right edges, where $\epsCont \defeq \eps\epsLarge/6|\Tcal|$.
The rest of the space is for tall compartments.
Therefore, the number of configurations is at most
\[ \nC \defeq \left((1/\epsCont)^2|\Tcal|^2/2\right)^{\nW}
    \le \left(\frac{3|\Tcal|^2}{\eps\epsLarge}\right)^{6|\Tcal|/\epsLarge}
    \le \left(1+\frac{1}{\eps\epsLarge}\right)^{
        \left(1+\frac{1}{\eps\epsLarge}\right)^{2/\epsLarge + 1}}. \]
Since each configuration can have at most $n$ bins, the number of combinations
of configurations is at most $(n+1)^{\nC}$.

Therefore, we can iterate over all possible bin packings of empty compartments in $O(n^{\nC})$ time.
Let $\iterPackings(\Itild)$ be an algorithm for this, i.e.,
$\iterPackings(\Itild)$ outputs the set of all possible bin packings of empty compartments into
at least $\smallceil{a(\Itild)}$ and at most $n$ bins,
where $n$ is the number of items in $\Itild$.

\subsection{Packing Items Into Compartments}
\label{sec:feas-lp}

For each bin packing of empty compartments, we will try to
fractionally pack the items into the bins.
Formally, let $P$ be a packing of empty compartments into bins.
We will create a feasibility linear program, called $\FP(\Itild, P)$,
that is feasible iff wide and tall items in $\Itild$ can be packed
into the compartments in $P$.
If $\FP(\Itild, P)$ is feasible, then small items can also be
fractionally packed since $P$ contains at least $a(\Itild)$ bins.

Let $w_1', w_2', \ldots, w_p'$ be the distinct widths of wide compartments in $P$.
Let $U_j$ be the set of wide compartments in $P$ having width $w_j'$.
Let $h(U_j)$ be the sum of heights of the compartments in $U_j$.
By \cref{defn:thin-bp:compartmental}, we know that $p \le |\Tcal|^2/2$.
Let $w_1, w_2, \ldots, w_r$ be the distinct widths of items in $\Wtild$
(recall that $\Wtild$ is the set of wide items in $\Itild$).
Let $\Wtild_j$ be the items in $\Wtild$ having width $w_j$.
Let $h(\Wtild_j)$ be the sum of heights of all items in $\Wtild_j$.
By \cref{thm:thin-bp:lingroup-distinct}, we get $r \le 1/\eps\epsLarge$.

Let $C \defeq [C_0, C_1, \ldots, C_r]$ be a vector, where
$C_0 \in [p]$ and $C_j \in \mathbb{Z}_{\ge 0}$ for $j \in [r]$.
$C$ is called a \emph{wide configuration} iff
$w(C) \defeq \sum_{j=1}^r C_jw_j \le w_{C_0}'$.
Intuitively, a wide configuration $C$ represents a set of wide items that can be placed
side-by-side into a compartment of width $w_{C_0}'$.
Let $\Ccal$ be the set of all wide configurations.
Then $|\Ccal| \le p/\epsLarge^r$, which is a constant.
Let $\Ccal_j \defeq \{C \in \Ccal: C_0 = j\}$.

To pack $\Wtild$ into wide compartments,
we must determine the height of each configuration.
Let $x \in \mathbb{R}_{\ge 0}^{|\Ccal|}$ be a vector where
$x_C$ denotes the height of configuration $C$.
Then $\Wtild$ can be packed into wide compartments according to $x$ iff
$x$ is a feasible solution the following feasibility linear program,
named $\FP_W(\Itild, P)$:
\[ \begin{array}{*4{>{\displaystyle}l}}
\sum_{C \in \Ccal} C_jx_C &\ge h(\Wtild_j)
    & \forall j \in [r]
    & \qquad\textrm{($\Wtild_j$ should be covered)}
\\[1.75em] \sum_{C \in \Ccal \textrm{ and } C_0 = j} x_C &\le h(U_j)
    & \forall j \in [p]
    & \qquad\textrm{($\Ccal_j$ should fit in $U_j$)}
\\[1.75em] x_C &\ge 0 & \forall C \in \Ccal
\end{array} \]

Let $x^*$ be an extreme point solution to $\FP_W(\Itild, P)$
(if $\FP_W(\Itild, P)$ is feasible).
By Rank Lemma, at most $p+r$ entries of $x^*$ are non-zero.
Since the number of variables and constraints is constant,
$x^*$ can be computed in constant time.

Let $\Htild$ be the set of tall items in $\Itild$.
By \cref{thm:thin-bp:lingroup-distinct}, we get that items in $\Htild$
have at most $1/\eps\epsLarge$ distinct heights.
Let there be $q$ distinct heights of tall compartments in $P$.
By \cref{defn:thin-bp:compartmental}, we know that $q \le 1/\epsCont = 6|\Tcal|/\eps\epsLarge$.
We can similarly define \emph{tall configurations} and we can similarly define
a feasibility linear program for tall items, named $\FP_H(\Itild, P)$.
$\Htild$ can be packed into tall compartments in $P$ iff $\FP_H(\Itild, P)$ is feasible.
Let $y^*$ be an extreme point solution to $\FP_H(\Itild, P)$.
Then $y^*$ can be computed in constant time and
$y^*$ has at most $q + 1/\eps\epsLarge$ positive entries.

Therefore, $\Itild$ can be packed into $P$ iff the feasibility linear program
$\FP(\Itild, P) \defeq \FP_W(\Itild, P) \wedge \FP_H(\Itild, P)$ is feasible.

The solution $(x^*, y^*)$ shows us how to split each compartment into \emph{shelves},
where each shelf corresponds to a configuration $C$
and the shelf can be split into $C_j$ \emph{containers} of width $w_j$
and one container of width $w_{C_0}' - w(C)$.
Let there be $m$ bins in $P$. After splitting the configurations across compartments,
we get at most $p + q + 2/\eps\epsLarge + m(\nW + \nH)$ shelves.

\subsection{Converting a Fractional Packing to a Non-Fractional Packing}
\label{sec:greedy-cont}

Let there be $m$ bins in a packing $P$ of empty compartments into bins.
Suppose it is possible to pack $\Itild$ into $P$.
Let $x^*$ and $y^*$ be extreme-point solutions to
$\FP_W(\Itild, P)$ and $\FP_H(\Itild, P)$, respectively.
This gives us a fractional compartmental packing of $\Itild$ into $m$ bins.
We will now show how to convert this to a non-fractional compartmental packing
by removing some items of small total area.
Formally, we give an algorithm called $\greedyPack(\Itild, P, x^*, y^*)$.
It returns a pair $(Q, D)$, where $Q$ is a (non-fractional) compartmental
bin packing of items $\Itild - D$, where the compartments in the bin are as per $P$.
$D$ is called the set of discarded items, and we will prove that $a(D)$ is small.

For a configuration $C$ in a wide compartment, there is a container
of width $w_{C_0}' - w(C)$ available for packing small items.
Hence, there are $p + q + 2/\eps\epsLarge + m(\nW + \nH)$ containers available
inside compartments for packing small items.
By \cref{thm:empty-to-rects}, we can partition the space outside compartments into
at most $m(3(\nW + \nH) + 1)$ containers.
Therefore, the total number of containers available for packing small items is at most
\[ m_S \defeq (p + q + 2/\eps\epsLarge) + m(4(\nW + \nH) + 1)
\le \left(\frac{|\Tcal|^2}{2} + \frac{6|\Tcal|}{\eps\epsLarge}
    + \frac{2}{\eps\epsLarge}\right) + \frac{16|\Tcal|}{\epsLarge} m. \]

Greedily assign small items to small containers, i.e., keep assigning small items
to a container till the area of items assigned to it is at least
the area of the container, and then resume from the next container.
Each small item will get assigned to some container.
For each container $C$, pack the largest possible prefix of the assigned items using
the Next-Fit Decreasing Height (NFDH) algorithm.
By \cref{thm:nfdh-small}, the area of unpacked items would be
less than $\epsSmall + \delta + \epsSmall\delta$. Summing over all containers,
we get that the unpacked area is less than
$(\epsSmall + \delta + \epsSmall\delta)m_S \le 3\epsSmall m_S$.

For each $j$, greedily assign wide items from $\Wtild_j$ to containers of width $w_j$,
i.e., keep assigning items till the height of items exceeds the height of the container.
Each wide item will get assigned to some container.
Then discard the last item from each container.
For each shelf in a wide compartment having configuration $C$,
the total area of items we discard is at most $\delta w(C)$.
Similarly, we can discard tall items of area at most $\delta h(C)$
from each shelf in a tall compartment having configuration $C$.

Hence, across all configurations, we discard wide and tall items of area at most
\[ \delta((p + q + 2/\eps\epsLarge) + m(\nW + \nH))
\le \delta\left(\frac{|\Tcal|^2}{2} + \frac{6|\Tcal|}{\eps\epsLarge}
    + \frac{2}{\eps\epsLarge}\right) + \frac{4\delta|\Tcal|}{\epsLarge}m. \]

Therefore, for $(Q, D) \defeq \greedyPack(\Itild, P, x^*, y^*)$, we get
\begin{equation}
\label{eqn:greedy-discard-area}
a(D) < \frac{52|\Tcal|\epsSmall}{\epsLarge} m
    + 4\epsSmall\left(\frac{|\Tcal|^2}{2} + \frac{6|\Tcal|}{\eps\epsLarge}
        + \frac{2}{\eps\epsLarge}\right)
\end{equation}
where $m$ is the number of bins used by $P$.

\subsection{The Algorithm}
\label{sec:thinCPack}

We now summarize the algorithm for bin packing $\delta$-\thin{} items $I$
(see \cref{algo:thinCPack} for a more precise description).
First, use $\round$ on $I$, i.e., let $(\Itild, \Imed) \defeq \round(I)$.
Then enumerate all packings $P$ of compartments into bins as per \cref{sec:enum-configs}.
For each packing $P$, check if $\Itild$ can be fractionally packed into $P$
by solving the feasibility linear program (see \cref{sec:feas-lp}).
If yes, then use a solution to the feasibility linear program to
compute a (non-fractional) compartmental packing of $\Itild - D$
using $\greedyPack$ (see \cref{sec:greedy-cont}),
where $D$ is the set of items discarded by $\greedyPack$.
Then pack $\Imed \cup D$ into bins using the Next-Fit Decreasing Height (NFDH) algorithm.
Output the best bin packing of $I$ across all choices of $P$.

\begin{algorithm}[htb]
\caption{$\thinCPack_{\eps}(I)$: Packs a set $I$ of $\delta$-\thin{} rectangular items
into bins without rotating the items.}
\label{algo:thinCPack}
\begin{algorithmic}[1]
\State $(\Itild, \Imed) = \round_{\eps}(I)$.
\State Initialize $\Qbest$ to \Null.
\For{$P \in \iterPackings(\Itild)$}
    \Comment{$\iterPackings$ is defined in \cref{sec:enum-configs}.}
    \State $x^* = \opt(\FP_W(\Itild, P))$.
    \Comment{$\FP_W$ and $\FP_H$ are defined in \cref{sec:feas-lp}.}
    \LineComment{If $\FP_W(\Itild, P)$ is feasible,
        $x^*$ is an extreme-point solution to $\FP_W(\Itild, P)$.}
    \LineComment{If $\FP_W(\Itild, P)$ is infeasible, $x^*$ is \Null.}
    \State $y^* = \opt(\FP_H(\Itild, P))$.
    \If{$x^* \neq \Null$ and $y^* \neq \Null$}
        \Comment{if $\Itild$ can be packed into $P$}
        \State $(Q, D) = \greedyPack(\Itild, P, x^*, y^*)$.
        \Comment{$\greedyPack$ is defined in \cref{sec:greedy-cont}.}
        \State $Q_D = \operatorname{NFDH}(D \cup \Imed)$.
        \If{$Q \cup Q_D$ uses less bins than $\Qbest$}
            \State $\Qbest = Q \cup Q_D$.
        \EndIf
    \EndIf
\EndFor
\State \Return $\Qbest$
\end{algorithmic}
\end{algorithm}

Recall the function $f$ from \cref{eqn:thin-bp:remmed-func} in \cref{sec:thin-bp:remmed}.
Since $\epsSmall \defeq f(\epsLarge)$, we get
\begin{equation}
\label{eqn:epsSmall}
\epsSmall = f(\epsLarge)
= \frac{\eps\epsLarge}{104(1+1/\eps\epsLarge)^{2/\epsLarge-2}}
\le \frac{\eps\epsLarge}{104|\Tcal|}.
\end{equation}
The last inequality follows from the fact that
$|\Tcal| \le (1+1/\eps\epsLarge)^{2/\epsLarge-2}$.

\rthmNfdhWide*
\begin{proof}(See \cref{sec:thin-gpack:summary}.)\end{proof}

\begin{lemma}
\label{thm:nfdh-tall}
Let $I$ be a set of rectangular items where each item has width at most $\delta$.
Then the number of bins required by NFDH to pack $I$ is less than $2a(I)/(1-\delta) + 3$.
\end{lemma}
\begin{proof}
The bin packing version of NFDH first packs $I$ into shelves
and then packs the shelves into bins using Next-Fit.
Let the number of shelves be $p$.
Let $h_j$ be the height of the $j\Th$ shelf.
Let $S_j$ be the items in the $j\Th$ shelf.
For $j \in [p-1]$, in the $j\Th$ shelf, the total width of items is
more than $(1-\delta)$ and each item has height more than $h_{j+1}$.
Therefore, $a(S_j) > h_{j+1}(1-\delta)$.

Let $H$ be the sum of heights of all the shelves. Then
\begin{align*}
& a(I) > \sum_{i=1}^{p-1} a(S_j) > \sum_{i=1}^{p-1} h_{j+1}(1-\delta) > (1-\delta)(H - h_1)
\\ &\implies H < \frac{a(I)}{1-\delta} + 1.
\end{align*}

By \cref{thm:next-fit}, the number of bins is less than $2H + 1 < 2a(I)/(1-\delta) + 3$.
\end{proof}

\begin{theorem}
The number of bins used by $\thinCPack_{\eps}(\Itild)$ is less than
\[ (1+20\eps)\opt(I)
    + \frac{1}{13}\left(1 + \frac{1}{\eps\epsLarge}\right)^{2/\epsLarge - 2} + 23. \]
\end{theorem}
\begin{proof}
In an optimal fractional compartmental bin packing of $\Itild$,
let $P^*$ be the corresponding packing of empty compartments into bins.
Hence, $P^*$ contains $m \defeq \fcopt(\Itild)$ bins.
Since $\iterPackings(\Itild)$ iterates over all packings of compartments into bins,
$P^* \in \iterPackings(\Itild)$.
Since wide and tall items in $\Itild$ can be packed into the compartments of $P^*$,
we get that $x^*$ and $y^*$ are not \Null.
By \cref{thm:nfdh-wide,thm:nfdh-tall}, the number of bins used by NFDH to pack
$\Imed \cup D$ is less than $2a(\Imed \cup D)/(1-\delta) + 3 + 1/(1-\delta)$.
Therefore, the number of bins used by $\thinCPack(I)$ is less than
\begin{align*}
& m + \frac{2a(\Imed \cup D)}{1-\delta} + 3 + \frac{1}{1-\delta}
\\ &< m + \frac{2\eps}{1-\delta}a(I)
    + \frac{2\epsSmall}{1-\delta}\left(\frac{52|\Tcal|}{\epsLarge} m
        + 4\left(\frac{|\Tcal|^2}{2} + \frac{6|\Tcal|+2}{\eps\epsLarge}\right)\right)
    + 3 + \frac{1}{1-\delta}
    \tag{by \cref{thm:thin-bp:remmed-area,eqn:greedy-discard-area}}
\\ &= \left(1 + \frac{104\epsSmall|\Tcal|}{\epsLarge(1-\delta)}\right)m
    + \frac{2\eps}{1-\delta}a(I)
    + 3 + \frac{1}{1-\delta} + \frac{8\epsSmall}{1-\delta}\left(
        \frac{|\Tcal|^2}{2} + \frac{6|\Tcal|+2}{\eps\epsLarge}\right)
\\ &= \left(1 + \frac{\eps}{1-\delta}\right)m + \frac{2\eps}{1-\delta}a(I)
    + 3 + \frac{1}{13(1-\delta)}\left(\frac{\eps\epsLarge|\Tcal|}{2}
        + 19 + \frac{2}{|\Tcal|}\right).
    \tag{by \cref{eqn:epsSmall}}
\end{align*}
By \cref{thm:struct,thm:thin-bp:lingroup-opt-compare}, we get
\[ m = \fcopt(\Itild) < (1+4\eps)\fopt(\Itild) + 2
< (1+4\eps)(1+\eps)\opt(I) + 4 + 8\eps. \]
Therefore, the number of bins used by $\thinCPack(I)$ is less than
\begin{align*}
& \left((1+4\eps)(1+\eps)\left(1 + \frac{\eps}{1-\delta}\right)
    + \frac{2\eps}{1-\delta}\right)\opt(I)
\\ &\qquad + (4 + 8\eps)\left(1 + \frac{\eps}{1-\delta}\right) + 3
    + \frac{1}{13(1-\delta)}\left(\frac{\eps\epsLarge|\Tcal|}{2} + 19 + \frac{2}{|\Tcal|}\right)
\\ &\le (1+20\eps)\opt(I)
    + \frac{1}{13}\left(1 + \frac{1}{\eps\epsLarge}\right)^{2/\epsLarge - 2} + 23.
    \tag{since $\delta \le \epsLarge \le \eps \le 1/2$}
\end{align*}
\end{proof}
