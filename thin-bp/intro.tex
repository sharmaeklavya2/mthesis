\thinBPIntro

\subsection*{Overview of the Algorithm}

$\thinCPack$ takes a set $I$ of items as input and has the following outline:
\begin{enumerate}
\item Invoke the subroutine $\round(I)$ (described in \cref{sec:thin-bp:round}).
    $\round(I)$ removes some items $\Imed \subseteq I$ of low total area and
    rounds up the width or height of each remaining item
    so that the resulting items $\Itild$ have special properties
    that help us pack them easily.
\item Compute the optimal \emph{fractional compartmental} bin-packing of $\Itild$
    (we will define \emph{compartmental} and \emph{fractional} later).
\item Use this packing of $\Itild$ to obtain a packing of $I$
    that uses slightly more number of bins.
\end{enumerate}

Let $\opt(I)$ be the minimum number of bins needed to pack $I$.
To bound the AAR of $\thinCPack$, we will prove a structural theorem
in \cref{sec:thin-bp:struct}, i.e., we will prove that
the optimal fractional compartmental packing of $\Itild$ uses close to $\opt(I)$ bins.

We will focus on the case where the items cannot be rotated,
so we will assume \wLoG{} that the bin is a square of side length 1.
In \cref{sec:thin-bp:rot}, we show how to extend $\thinCPack$ to the case
where the items can be rotated by $90^{\circ}$.

\subsection*{Organization of the Chapter}

\begin{itemize}
\item In \cref{sec:thin-bp:round}, we describe the subroutine $\round$
    and define \emph{fractional} packing.
\item In \cref{sec:thin-bp:struct}, we define \emph{compartmental} packing
    and prove the structural theorem.
\item In \cref{sec:thin-bp:algo}, we describe the $\thinCPack$ algorithm.
\item In \cref{sec:thin-bp:rot}, we show how to extend $\thinCPack$ to
    handle item rotations.
\end{itemize}
