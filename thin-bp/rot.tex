\section{Handling Item Rotations}
\label{sec:thin-bp:rot}

In this section, we will briefly explain changes to $\thinCPack$ and its analysis
so that they work for the case where items can be rotated by $90^{\circ}$.

For the rotational version of the problem, we can assume \wLoG{} that
the height of each $\delta$-thin rectangle is at most $\delta$,
because otherwise we can rotate the item.
We assume \wLoG{} that the width of the bin is 1
and the height of the bin is at least 1,
because we can rotate the bin and scale its width and height equally.
We also assume that the height of the bin is a constant.

To handle the rotational case, we do not require any change in \cref{sec:thin-bp:round}.

The structural theorem in \cref{sec:thin-bp:struct} doesn't require any
conceptual modifications. The only change is that the number of tall compartments
in a tall cell can be more than $1/\epsLarge-1$.
Specifically, if the height of the bin is $H$, the number of tall compartments
in a tall cell is now upper-bounded by $H/\epsLarge$.
(Hence, for the number of compartments to be a constant,
we require the bin's height to be a constant).
This will increase the running time of $\iterPackings$,
but there will be no change to \cref{thm:struct}.

The feasibility linear program of \cref{sec:feas-lp} will have to change
to take item rotations into account. Instead of using two programs---%
$\FP_W$ and $\FP_H$---which fractionally pack wide and tall items separately,
we will use just one program which will also decide which items to rotate.
To do this, we allow an item to belong to both wide and tall configurations.
The number of constraints in the feasibility linear program will
be a constant that depends on $\eps$ and $\epsLarge$.

The $\greedyPack$ algorithm of \cref{sec:greedy-cont} will remain the same,
but the total area of discarded items will be slightly different because
the number of compartments in a bin can now be larger.
Finally, the AAR of $\thinCPack$ for the rotational version will be $1 + \Theta(1)\eps$
by the same kind of analysis as in \cref{sec:thinCPack}.
