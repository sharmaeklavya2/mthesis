\section{Classifying and Rounding Items}
\label{sec:thin-bp:round}

In this section, we will describe the algorithm $\round(I)$.
$\round(I)$ returns a pair $(\Itild, \Imed)$, where $\Itild$
is called the set of rounded items and
$\Imed \subseteq I$ is called the set of medium items.
We will show that $a(\Imed) \le \eps a(I)$ and
$\Itild$ is obtained by rounding up the width or height of each item in $I - \Imed$.

We assume that $\eps \le 1/2$ and that $\eps^{-1} \in \mathbb{Z}$.

\subsection{Removing Medium Items}
\label{sec:thin-bp:remmed}

We will choose $\Imed \subseteq I$ such that for two constants $\epsSmall$ and $\epsLarge$,
no item in $I - \Imed$ has its width or height in the interval $(\epsSmall, \epsLarge]$
and $\epsSmall \ll \epsLarge < 1$
(we will soon precisely define the meaning of $\epsSmall \ll \epsLarge$).
For $\thinCPack$ to work, we require $\delta \le \epsSmall$.

\begin{definition}
\label{defn:thin-bp:remmed}
Let $\mu_0 \in (0, 1]$ be a constant and let
$f: (0, 1] \mapsto (0, 1]$ be a function such that $\forall x \in (0, 1], f(x) < x$.
Let $T \defeq \ceil{2/\eps}$.
For $t \in [T]$, define $\mu_t \defeq f(\mu_{t-1})$ and define
\[ J_t \defeq \{i \in I: w(i) \in (\mu_t, \mu_{t-1}]
    \textrm{ or } h(i) \in (\mu_t, \mu_{t-1}]\}. \]
Define $\remMed(I, \eps, f, \mu_0)$ as the tuple $(J_r, \mu_r, \mu_{r-1})$,
where $r \defeq \argmin_{t=1}^T a(J_t)$.
\end{definition}

\begin{lemma}
\label{thm:thin-bp:remmed-area}
Let $(\Imed, \epsSmall, \epsLarge) \defeq \remMed(I, \eps, f, \mu_0)$.
Then $a(\Imed) \le \eps a(I)$.
\end{lemma}
\begin{proof}
Each item belongs to at most 2 sets $J_t$. Therefore,
\[ a(\Imed) = \min_{t=1}^T a(J_t)
\le \frac{1}{T} \sum_{t=1}^T a(J_t)
\le \frac{2}{\ceil{2/\eps}} a(I)
\le \eps a(I) \qedhere \]
\end{proof}

No item in $I - \Imed$ has width or height in the interval $(\epsSmall, \epsLarge]$.

Let $\mu_0 = \eps$. So, $\epsLarge \le \eps$ and $\epsSmall \defeq f(\epsLarge)$.
We choose $f$ to be
\begin{equation}
\label{eqn:thin-bp:remmed-func}
f(x) \defeq \frac{\eps x}{104(1+1/(\eps x))^{2/x-2}}.
\end{equation}
We will explain this choice later in \cref{sec:thinCPack}.
Intuitively, such an $f$ ensures that $\epsSmall = f(\epsLarge) \ll \epsLarge$.
Note that $f$ is independent of $I$, so $\epsLarge$ and $\epsSmall$ are constants.
Also note that $x^{-1} \in \mathbb{Z} \implies f(x)^{-1} \in \mathbb{Z}$,
so $\epsLarge^{-1}, \epsSmall^{-1} \in \mathbb{Z}$.

\subsection{Classifying Items}

Classify the items in $I - \Imed$ into three disjoint classes:
\begin{itemize}
\item Wide items: $W \defeq \{i \in I: w(i) > \epsLarge \textrm{ and } h(i) \le \epsSmall \}$.
\item Tall items: $H \defeq \{i \in I: w(i) \le \epsSmall \textrm{ and } h(i) > \epsLarge \}$.
\item Small items: $S \defeq \{i \in I: w(i) \le \epsSmall \textrm{ and } h(i) \le \epsSmall \}$.
\end{itemize}

\subsection{Linear Grouping}

We will now use \emph{linear grouping}~\cite{bp-aptas,kenyon1996strip}
to round up the widths of items in $W$ and the heights of items in $H$.
Arrange the items of $W$ in decreasing order of width and stack them
one-over-the-other (i.e., the widest item in $W$ is at the bottom).
Let $h_L$ be the height of the stack.
Let $y(i)$ be the $y$-coordinate of the bottom edge of item $i$.
Split the stack into sections of height $\eps\epsLarge h_L$ each.
For $j \in [1/\eps\epsLarge]$, let $w_j$ be the width of the
widest item intersecting the $j\Th$ section from the bottom, i.e.,
\[ w_j \defeq \max(\{w(i): i \in W \textrm{ and } (y(i), y(i) + h(i))
    \cap ((j-1)\eps\epsLarge h_L, j\eps\epsLarge h_L) \neq \emptyset\}). \]
Round up the width of each item $i$ to the smallest $w_j$ that is at least $w(i)$.
Let $W_j$ be the items whose width got rounded to $w_j$
and let $\Wtild_j$ be the resulting rounded items.
(There may be ties, i.e., there may exist $j_1 < j_2$ such that $w_{j_1} = w_{j_2}$.
In that case, define $W_{j_2} \defeq \Wtild_{j_2} = \emptyset$.
This ensures that all $W_j$ are disjoint.)
Let $\Wtild \defeq \bigcup_j \Wtild_j$.

Define $\Htild$ analogously. Let $\Itild \defeq \Wtild \cup \Htild \cup S$.

\begin{claim}
\label{thm:thin-bp:lingroup-distinct}
Items in $\Wtild$ have at most $1/\eps\epsLarge$ distinct widths.
Items in $\Htild$ have at most $1/\eps\epsLarge$ distinct heights.
\end{claim}

\begin{definition}[Fractional packing]
Suppose we are allowed to slice wide items in $\Itild$ using horizontal cuts,
slice tall items in $\Itild$ using vertical cuts and slice
small items in $\Itild$ using both horizontal and vertical cuts.
For any $\Xtild \subseteq \Itild$, a bin packing of the slices of $\Xtild$
is called a \emph{fractional packing} of $\Xtild$.
The optimal fractional packing of $\Xtild$ is denoted by $\fopt(\Xtild)$.
\end{definition}

\begin{lemma}
\label{thm:thin-bp:lingroup-opt-compare}
$\fopt(\Itild) < (1+\eps)\opt(I) + 2$.
\end{lemma}
\begin{proof}
Consider the optimal packing of $I$.
To convert this to a packing of $\Itild - (\Wtild_1 \cup \Htild_1)$,
unpack $W_1$ and $H_1$, and for each $j \in [1/\eps\epsLarge-1]$,
pack $\Wtild_{j+1}$ in the place of $W_j$
and pack $\Htild_{j+1}$ in the place of $H_j$,
possibly after slicing the items.
Therefore,
\begin{equation}
\label{eqn:thin-bp:lingroup-1}
\fopt(\Itild - (\Wtild_1 \cup \Htild_1)) \le \opt(I).
\end{equation}

We can pack $\Htild_1$ in a bin by stacking the items side-by-side on the base of bins.
We can pack $\Wtild_1$ in a bin by stacking the items one-over-the-other.
Let $w_L$ be the total width of items in $\Htild$. The number of bins used is
$\smallceil{\eps\epsLarge h_L} + \smallceil{\eps\epsLarge w_L}$. Also,
\[ \opt(I) \ge \opt(W \cup H)
\ge a(W) + a(H) \ge \epsLarge(h_L + w_L) \]
Therefore,
\begin{equation}
\label{eqn:thin-bp:lingroup-2}
\fopt(\Wtild_1 \cup \Htild_1)
\le \smallceil{\eps\epsLarge h_L} + \smallceil{\eps\epsLarge w_L} < \eps\opt(I) + 2.
\end{equation}
On combining \eqref{eqn:thin-bp:lingroup-1} and \eqref{eqn:thin-bp:lingroup-2}, we get
\[ \fopt(\Itild) \le \fopt(\Itild - (\Wtild_1 \cup \Htild_1))
    + \fopt(\Wtild_1 \cup \Htild_1)
< (1+\eps)\opt(I) + 2 \qedhere \]
\end{proof}
