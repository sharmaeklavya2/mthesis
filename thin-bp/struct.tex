\section{Structural Theorem}
\label{sec:thin-bp:struct}

In this section, we will define compartmental packing
and we will prove the structural theorem, which says that
the number of bins in the optimal fractional compartmental packing of $\Itild$
is roughly equal to $\fopt(\Itild)$.

For any rectangle $i$ packed in a bin, let $x_1(i)$ and $x_2(i)$ denote the $x$-coordinates
of its left and right edges, respectively, and let $y_1(i)$ and $y_2(i)$
denote the $y$-coordinates of its bottom and top edges, respectively.
Let $R$ be the set of distinct widths of items in $\Wtild$.
Given the way we rounded items, $|R| \le 1/\eps\epsLarge$.

Recall that $\epsLarge \le \eps \le 1/2$.

\subsection{Discretizing Horizontal Positions}

We will show that given a fractional packing of items in a bin,
we can remove a small fraction of tall and small items
and shift the remaining items leftwards so that the left and right edges
of each wide item belong to a constant-sized set $\Tcal$.

Let $T_0 \defeq \{0\}$ and $t_0 \defeq 1$. For any $j > 0$, define
\begin{itemize}
\item $t_j \defeq (1+1/\eps\epsLarge)^{2j}$.
\item $\delta_j \defeq \eps\epsLarge/t_{j-1}$.
\item $S_j \defeq T_{j-1} \cup \{k\delta_j: k \in \mathbb{Z}
    \textrm{ and } 0 \le k < 1/\delta_j\}$.
\item $T_j \defeq \{x + y: x \in S_j \textrm{ and } y \in R \cup \{0\}\}$.
\end{itemize}

\begin{observation}
\label{obs:shift}
For all $j > 0$, we have $T_{j-1} \subseteq S_j \subseteq T_j$
and $\delta_j^{-1} \in \mathbb{Z}$.
\end{observation}

\begin{lemma}
For all $j \ge 0$, $|T_j| \le t_j$.
\end{lemma}
\begin{proof}
We will prove this by induction. The base case holds because $|T_0| = t_0 = 1$.

Now assume $|T_{j-1}| \le t_{j-1}$. Then
\begin{align*}
|T_j| &\le (|R|+1)|S_j|
    \le \left(\frac{1}{\eps\epsLarge}+1\right)\left(|T_{j-1}| + \frac{1}{\delta_j}\right)
    \le \left(\frac{1}{\eps\epsLarge}+1\right)^2 t_{j-1}
    = t_j.
\end{align*}
Hence, by mathematical induction, $|T_j| \le t_j$ for all $j \ge 0$.
\end{proof}

Define $\Tcal \defeq T_{1/\epsLarge-1}$.
Therefore, $|\Tcal| \le t_{1/\epsLarge-1} = (1+1/\eps\epsLarge)^{2/\epsLarge - 2}$.

\begin{lemma}
\label{thm:disc-hor-pos}
Given a fractional packing of items $\Jtild \subseteq \Itild$ into a bin,
we can remove tall and small items of total area less than $\eps$
and shift some of the remaining items to the left such that for every wide item $i$,
we get $x_1(i), x_2(i) \in \Tcal$.
\end{lemma}
\begin{proof}
We will describe an algorithm for such a transformation.

For wide items $u$ and $v$, we say that $u \prec v$ iff
the right edge of $u$ is to the left of the left edge of $v$.
Formally $u \prec v \iff x_2(u) \le x_1(v)$.
We call $u$ a \emph{predecessor} of $v$.
Note that the relation $\prec$ is transitive.
A sequence $[i_1, i_2, \ldots, i_k]$ such that $i_1 \prec i_2 \prec \ldots \prec i_k$
is called a \emph{chain} ending at $i_k$.
For a wide item $i$, define $\level(i)$ as the number of items in the longest chain
ending at $i$. Formally,
\[ \level(i) \defeq \begin{cases}
1 & \textrm{ if } i \textrm{ has no predecessors}
\\ \displaystyle 1 + \max_{j \prec i} \level(j) & \textrm{ otherwise}
\end{cases}. \]
Let $W_j$ be the items at level $j$, i.e., $W_j \defeq \{i: \level(i) = j\}$.

\begin{figure}[htb]
\centering
\ifcsname pGameL\endcsname\else\newlength{\pGameL}\fi
\setlength{\pGameL}{0.3cm}
\tikzset{bin/.style={draw,thick}}
\tikzset{item/.style={draw,fill={textColor!15!bgColor}}}
\tikzset{myarrow/.style={draw,->,>={Stealth}}}
\tikzset{mynode/.style={pos=0.5,inner sep=0,minimum size=0.5cm,shape=circle,thick,draw}}
\tikzset{cutline/.style={draw,dashed}}
%\tikzset{binGrid/.style={draw,step=1\pGameL,{textColor!20!bgColor}}}
\begin{tikzpicture}
%\path[binGrid] (0\pGameL, 0\pGameL) grid (20\pGameL, 20\pGameL);

\path[cutline] (2\pGameL, 0\pGameL) -- +(0, 20\pGameL);
\path[cutline] (5\pGameL, 0\pGameL) -- +(0, 20\pGameL);
\path[cutline] (7\pGameL, 0\pGameL) -- +(0, 20\pGameL);
\path[cutline] (9\pGameL, 0\pGameL) -- +(0, 20\pGameL);
\path[cutline] (14\pGameL, 0\pGameL) -- +(0, 20\pGameL);
\path[cutline] (16\pGameL, 0\pGameL) -- +(0, 20\pGameL);

\path[item] (0\pGameL, 15\pGameL) rectangle +(7\pGameL, 2\pGameL) node[mynode] (wa) {$a$};
\path[item] (5\pGameL, 11\pGameL) rectangle +(9\pGameL, 2\pGameL) node[mynode] (wb) {$b$};
\path[item] (16\pGameL, 10\pGameL) rectangle +(4\pGameL, 2\pGameL) node[mynode] (wc) {$c$};
\path[item] (9\pGameL, 18\pGameL) rectangle +(5\pGameL, 2\pGameL) node[mynode] (wd) {$d$};
\path[item] (2\pGameL, 6\pGameL) rectangle +(7\pGameL, 2\pGameL) node[mynode] (we) {$e$};
\path[item] (9\pGameL, 2\pGameL) rectangle +(11\pGameL, 2\pGameL) node[mynode] (wf) {$f$};

\path[bin] (0\pGameL, 0\pGameL) rectangle (20\pGameL, 20\pGameL);

\path[myarrow] (wa) -- (wc);
\path[myarrow] (wa) -- (wd);
\path[myarrow] (wa) -- (wf);
\path[myarrow] (wb) -- (wc);
\path[myarrow] (wd) -- (wc);
\path[myarrow] (we) -- (wc);
\path[myarrow] (we) -- (wd);
\path[myarrow] (we) -- (wf);
\end{tikzpicture}

\caption[Relation $\prec$ among items in a bin]%
{Example illustrating the $\prec$ relationship between wide items in a bin.
An edge is drawn from $u$ to $v$ iff $u \prec v$.
Here $W_1 = \{a, e, b\}$, $W_2 = \{d, f\}$ and $W_3 = \{c\}$.}
\label{fig:precedence-graph}
\end{figure}

Note that the level of an item can be at most $1/\epsLarge-1$,
since each wide item has width more than $\epsLarge$.

Our algorithm will proceed in stages, where in the $j\Th$ stage,
we apply two transformations to the items in the bin.
In the first transformation, called \emph{strip-removal},
we will remove some tall and small items.
In the second transformation, called \emph{compaction},
we will first shift some tall and small items leftwards
and then shift each item in $W_j$ leftwards.

We will maintain the following invariant throughout the algorithm:\\
\textbf{Invariant}: \textsl{after $k$ stages, for each $j \in [k]$,
each item $i \in W_j$ has $x_1(i) \in S_j$ (and hence $x_2(i) \in T_j$).}
Note that the invariant is trivially true for $k=0$.

\begin{definition}[Strip-removal]
In the $j\Th$ stage, for each $x \in T_{j-1}$,
consider a strip of width $\delta_j$ and height 1 in the bin
whose left edge has coordinate $x$.
Discard the slices of tall and small items inside the strips.
This transformation is called \emph{strip-removal}.
\end{definition}

\begin{lemma}
Items discarded from a bin by strip-removal (across all stages)
have total area less than $\eps$.
\end{lemma}
\begin{proof}
In the $j\Th$ stage, we create $|T_{j-1}|$ strips,
and each strip has total area at most $\delta_j$.
Therefore, the area discarded in the $j\Th$ stage is at most
$|T_{j-1}|\delta_j \le t_{j-1}\delta_j = \eps\epsLarge$.
Since there can be at most $1/\epsLarge-1$ stages,
we discard an area of less than $\eps$ across all stages.
\end{proof}

\begin{definition}[Compaction]
In the $j\Th$ stage, move all tall and small items as much towards the left as possible
(imagine a gravitational force acting leftwards on the tall and small items)
while keeping the wide items fixed.
Then move each wide item $i \in W_j$ leftwards till $x_1(i) \in S_j$.
This transformation is called \emph{compaction}.
\end{definition}

\begin{lemma}
Compaction always succeeds, i.e., in the $j\Th$ stage, while moving
item $i \in W_j$ leftwards, no other item will block its movement.
\end{lemma}
\begin{proof}
Let $i \in W_j$. Let $z$ be the $x$-coordinate of the left edge of the strip
immediately to the left of item $i$, i.e.,
$z \defeq \max(\{x \in T_{j-1}: x \le x_1(i)\})$.

For any wide item $i'$, we have $x_2(i') \le x_1(i) \iff i' \prec i \iff \level(i') \le j-1$.
By our invariant, we get
\[ \level(i') \le j-1 \implies x_2(i') \in T_{j-1} \implies x_2(i') \le z. \]
Therefore, for every wide item $i'$, $x_2(i') \not\in (z, x_1(i)]$.

In the $j\Th$ strip-removal, we cleared the strip $[z, z+\delta_j] \times [0, 1]$.
If $x_1(i) \in [z, z+\delta_j]$, then $i$ can freely move to $z$,
and $z \in T_{j-1} \subseteq S_j$.
Since no wide item has its right edge in $(z, x_1(i)]$, if $x_1(i) > z + \delta_j$,
all the tall and small items in $[z+\delta_j, x_1(i)]$ will move leftwards by
at least $\delta_j$ during compaction.
Hence, there would be an empty space of width at least $\delta_j$
to the left of item $i$ (see \cref{fig:compaction-zoom}).
Therefore, we can move $i$ leftwards to make
$x_1(i)$ a multiple of $\delta_j$, and then $x_1(i)$ would belong to $S_j$.
\end{proof}

\begin{figure}[!htb]
\centering
\ifcsname myu\endcsname\else\newlength{\myu}\fi
\setlength{\myu}{0.8cm}
\tikzset{mypic/.pic={
\path[item] (3\myu, 1\myu) rectangle (7\myu, 1.5\myu);
\path[item-boundary] (7\myu, 1\myu) -- (3\myu, 1\myu) -- (3\myu, 1.5\myu) -- (7\myu, 1.5\myu);
\path[item] (4.8\myu, 2\myu) rectangle (7\myu, 2.5\myu);
\path[item-boundary] (7\myu, 2\myu) -- (4.8\myu, 2\myu) -- (4.8\myu, 2.5\myu) -- (7\myu, 2.5\myu);
\path[item] (4\myu, 3\myu) rectangle (7\myu, 3.5\myu) node[pos=0.5] {$i$};
\path[item-boundary] (7\myu, 3\myu) -- (4\myu, 3\myu) -- (4\myu, 3.5\myu) -- (7\myu, 3.5\myu);
\path[item] (-1\myu, 2\myu) rectangle (0\myu, 3\myu);
\path[item-boundary] (-1\myu, 2\myu) -- (0\myu, 2\myu) -- (0\myu, 3\myu) -- (-1\myu, 3\myu);
\path[item] (-1\myu, 4.5\myu) rectangle (6\myu, 5\myu) node[pos=0.5] {$k$};
\path[item-boundary] (-1\myu, 4.5\myu) -- (6\myu, 4.5\myu) -- (6\myu, 5\myu) -- (-1\myu, 5\myu);

\draw[thick]
    (-1\myu, 0\myu) -- (7\myu, 0\myu)
    (-1\myu, 6\myu) -- (7\myu, 6\myu);
\node[anchor=east] at (-1\myu, 3\myu) {$\cdots$};
\node[anchor=west] at (7\myu, 3\myu) {$\cdots$};
\draw[dashed,semithick]
    (0\myu, 0\myu) -- (0\myu, 6\myu)
    (4\myu, 0\myu) -- (4\myu, 6\myu)
    (6\myu, 3.5\myu) -- (6\myu, 6\myu);
\node[anchor=south] at (0\myu, 6\myu) {$z$};
\node[anchor=south] at (4\myu, 6\myu) {$x_1(i)$};
\node at (6.5\myu, 4.75\myu) {$C$};
}}
\begin{tikzpicture}[
item-boundary/.style={draw,semithick},
item/.style={fill={textColor!25!bgColor}},
myarrow/.style={->,>={Stealth},thick},
mybrace/.style = {decoration={amplitude=5pt,brace,mirror,raise=1pt},semithick,decorate},
]
\begin{scope}
\path[fill={textColor!10!bgColor}]
    (-1\myu, 0\myu) rectangle (0\myu, 6\myu)
    (1\myu, 0\myu) rectangle (7\myu, 6\myu);
\draw[very thin] (0\myu, 0\myu) -- (0\myu, 6\myu) (1\myu, 0\myu) -- (1\myu, 6\myu);
\draw[mybrace] (0\myu, 0\myu) -- node[below=5pt] {$\delta_j$} (1\myu, 0\myu);
\pic at (0\myu, 0\myu) {mypic};
\end{scope}
\draw[myarrow] (3\myu, -0.5\myu) -- (3\myu, -2.5\myu)
    node[pos=0.4,anchor=west,text width=4.5cm]
    {shift tall and small items leftwards by $\delta_j$};
\begin{scope}[yshift={-7.2cm}]
\path[fill={textColor!10!bgColor}] (-1\myu, 0\myu) rectangle (7\myu, 6\myu);
\draw[very thin,fill=bgColor]
    (2\myu, 1\myu) rectangle (3\myu, 1.5\myu)
    (3.8\myu, 2\myu) rectangle (4.8\myu, 2.5\myu)
    (3\myu, 3\myu) rectangle (4\myu, 3.5\myu);
\pic at (0\myu, 0\myu) {mypic};
\draw[mybrace] (2\myu, 1\myu) -- node[below=5pt] {$\delta_j$} (3\myu, 1\myu);
\end{scope}
\end{tikzpicture}

\caption[Creation of empty space during compaction.]%
{This figure shows a region in the bin in the vicinity of item $i \in W_j$.
It illustrates how shifting tall and small items during compaction in the $j\Th$ stage
creates a free space of width $\delta$ to the left of some wide items, including $i$.
Wide items are shaded dark and the lightly shaded region
potentially contains tall and small items.
Note that some tall and small items in the region $C$
may be unable to shift left because item $k$ is blocking them.
All other tall and small items in this figure to the right of $z$
can shift left by $\delta_j$.}
\label{fig:compaction-zoom}
\end{figure}

Since compaction in the $j\Th$ stage would force $x_1(i)$ to belong to $S_j$
for each $i \in W_j$, the invariant is maintained after each stage.
Therefore, after $1/\epsLarge-1$ stages, we get that for each wide item $i$,
$x_1(i) \in S_{1/\epsLarge-1} \subseteq \Tcal$ and $x_2(i) \in T_{1/\epsLarge-1} = \Tcal$.
\end{proof}

\subsection{Creating Compartments}

\begin{definition}[Compartmental packing]
\label{defn:thin-bp:compartmental}
Consider a bin with some items packed into it. A \emph{compartment} $C$
is defined as a rectangular region in the bin satisfying the following properties:
\begin{itemize}
\item $x_1(C), x_2(C) \in \Tcal$.
\item $y_1(C), y_2(C)$ are multiples of $\epsCont \defeq \eps\epsLarge/6|\Tcal|$.
\item $C$ does not contain both wide items and tall items.
\item If $C$ contains tall items, then $x_1(C)$ and $x_2(C)$
    are consecutive values in $\Tcal$.
\end{itemize}
If a compartment $C$ contains a wide item, it is called a \emph{wide compartment}.
Otherwise it is called a \emph{tall compartment}.

A packing of items $\Jtild$ into a bin is said to be \emph{compartmental}
iff there is a set of non-overlapping \emph{compartments} in the bin
such that each wide or tall item lies completely inside some compartment,
and there are at most $\nW \defeq 3(1/\epsLarge-1)|\Tcal| + 1$ wide compartments in the bin
and there are at most $\nH \defeq (1/\epsLarge-1)|\Tcal|$ tall compartments in the bin.

A packing of items into bins is called compartmental iff
each bin in the packing is compartmental.
\end{definition}

\begin{lemma}
\label{thm:empty-to-rects}
Let there be a set $I$ of rectangles packed inside a bin.
Then there is a polynomial-time algorithm that can decompose the empty space in the bin
into at most $3|I|+1$ rectangles by making horizontal cuts only.
\end{lemma}
\begin{proof}
Extend the top and bottom edge of each rectangle leftwards and rightwards
till they hit another rectangle or an edge of the bin.
This decomposes the empty region into rectangles $R$.
See \cref{fig:thin-bp:empty-space-to-rects}.

For each rectangle $i \in I$, the top edge of $i$ is the bottom edge of a rectangle in $R$,
the bottom edge of $i$ is the bottom edge of two rectangles in $R$.
Apart from possibly the rectangle in $R$ whose bottom edge is at the bottom of the bin,
the bottom edge of every rectangle in $R$ is either the bottom or top edge of a rectangle in $I$.
Therefore, $|R| \le 3|I| + 1$.
\end{proof}

\begin{figure}[htb]
\centering
\ifcsname myu\endcsname\else\newlength{\myu}\fi
\setlength{\myu}{0.8cm}
\begin{tikzpicture}[
item/.style={draw,fill={textColor!15!bgColor}},
cutline/.style={draw,dashed},
]
\path[item] (1\myu, 1\myu) rectangle (3\myu, 2\myu);
\path[item] (5\myu, 2\myu) rectangle (6\myu, 4\myu);
\path[item] (2\myu, 3\myu) rectangle (4\myu, 5\myu);
\draw[thick] (0\myu, 0\myu) rectangle (7\myu, 6\myu);

\draw[cutline] (0\myu, 1\myu) -- (7\myu, 1\myu);
\draw[cutline] (0\myu, 2\myu) -- (7\myu, 2\myu);
\draw[cutline] (0\myu, 3\myu) -- (5\myu, 3\myu);
\draw[cutline] (4\myu, 4\myu) -- (7\myu, 4\myu);
\draw[cutline] (0\myu, 5\myu) -- (7\myu, 5\myu);

\path (0\myu, 0\myu) -- node {1} (7\myu, 1\myu);
\path (0\myu, 1\myu) -- node {2} (1\myu, 2\myu);
\path (3\myu, 1\myu) -- node {3} (7\myu, 2\myu);
\path (0\myu, 2\myu) -- node {4} (5\myu, 3\myu);
\path (6\myu, 2\myu) -- node {5} (7\myu, 4\myu);
\path (4\myu, 3\myu) -- node {6} (5\myu, 4\myu);
\path (0\myu, 3\myu) -- node {7} (2\myu, 5\myu);
\path (4\myu, 4\myu) -- node {8} (7\myu, 5\myu);
\path (0\myu, 5\myu) -- node {9} (7\myu, 6\myu);
\end{tikzpicture}

\caption{Using horizontal cuts to partition the empty space
around the 3 items into 9 rectangular regions.}
\label{fig:thin-bp:empty-space-to-rects}
\end{figure}

\begin{lemma}
\label{thm:thin-bp:compartmentalize}
Let $\Jtild$ be a packing of items into a bin such that for each wide item $i$,
$x_1(i), x_2(i) \in \Tcal$. Then by removing wide and small items of area less than $\eps$,
we can get a compartmental packing of the remaining items.
\end{lemma}
\begin{proof}
Draw vertical lines in the bin at the $x$-coordinates in $\Tcal - \{0\}$.
This splits the bin into $|\Tcal|$ columns (see \cref{fig:compartmentalize:begin}).
Each column has 0 or more wide items crossing it.
These wide items divide the column into cells.
A cell is called tall iff it contains a tall item (see \cref{fig:compartmentalize:tall-cells}).
There can be at most $1/\epsLarge-1$ tall cells in a column,
so there can be at most $(1/\epsLarge - 1)|\Tcal|$ tall cells in the bin.

\begin{figure}[!htb]
\begin{subfigure}{0.45\textwidth}
\centering
\ifcsname pGameL\endcsname\else\newlength{\pGameL}\fi
\setlength{\pGameL}{0.2cm}
\definecolor{myBlue}{HTML}{00b0f0}
\definecolor{myGreen}{HTML}{92d050}
\definecolor{myLightBlue}{HTML}{ccf1ff}
\definecolor{myLightGreen}{HTML}{e6f5d6}
\definecolor{shadedLightBlue}{HTML}{b4d9e4}
\definecolor{shadedLightGreen}{HTML}{ceddc0}

\tikzset{bin/.style={draw,thick}}
\tikzset{binGrid/.style={draw,step=1\pGameL,{textColor!20!bgColor}}}
\tikzset{witem/.style={draw,fill=myGreen}}
\tikzset{hitem/.style={draw,fill=myBlue}}
\tikzset{cutline/.style={draw,dashed,thick}}
\begin{tikzpicture}
%\path[binGrid] (0\pGameL, 0\pGameL) grid (30\pGameL, 30\pGameL);
\path[hitem] (0\pGameL, 22\pGameL) rectangle +(1\pGameL, 8\pGameL);
\path[hitem] (2\pGameL, 20\pGameL) rectangle +(2\pGameL, 8\pGameL);
\path[hitem] (7\pGameL, 21\pGameL) rectangle +(1\pGameL, 9\pGameL);
\path[hitem] (8\pGameL, 20\pGameL) rectangle +(1\pGameL, 9\pGameL);
\path[hitem] (10\pGameL, 22\pGameL) rectangle +(2\pGameL, 8\pGameL);
\path[hitem] (12\pGameL, 17\pGameL) rectangle +(2\pGameL, 8\pGameL);
\path[hitem] (14\pGameL, 20\pGameL) rectangle +(2\pGameL, 10\pGameL);
\path[hitem] (17\pGameL, 19\pGameL) rectangle +(2\pGameL, 11\pGameL);
\path[hitem] (0\pGameL, 1\pGameL) rectangle +(1\pGameL, 8\pGameL);
\path[hitem] (2\pGameL, 0\pGameL) rectangle +(1\pGameL, 12\pGameL);
\path[hitem] (4\pGameL, 1\pGameL) rectangle +(2\pGameL, 8\pGameL);
\path[hitem] (7\pGameL, 0\pGameL) rectangle +(2\pGameL, 9\pGameL);
\path[hitem] (11\pGameL, 4\pGameL) rectangle +(1\pGameL, 6\pGameL);
\path[hitem] (26\pGameL, 10\pGameL) rectangle +(2\pGameL, 9\pGameL);
\path[hitem] (16\pGameL, 12\pGameL) rectangle +(1\pGameL, 12\pGameL);
\path[hitem] (17\pGameL, 7\pGameL) rectangle +(1\pGameL, 11\pGameL);
\path[hitem] (18\pGameL, 7\pGameL) rectangle +(2\pGameL, 7\pGameL);
\path[hitem] (21\pGameL, 9\pGameL) rectangle +(2\pGameL, 13\pGameL);
\path[hitem] (25\pGameL, 8\pGameL) rectangle +(1\pGameL, 16\pGameL);
\path[hitem] (29\pGameL, 2\pGameL) rectangle +(1\pGameL, 8\pGameL);
\path[hitem] (27\pGameL, 2\pGameL) rectangle +(2\pGameL, 8\pGameL);
\path[hitem] (28\pGameL, 10\pGameL) rectangle +(1\pGameL, 9\pGameL);

\path[witem] (0\pGameL, 12\pGameL) rectangle +(10\pGameL, 2\pGameL);
\path[witem] (0\pGameL, 17\pGameL) rectangle +(10\pGameL, 2\pGameL);
\path[witem] (5\pGameL, 16\pGameL) rectangle +(10\pGameL, 1\pGameL);
\path[witem] (0\pGameL, 15\pGameL) rectangle +(10\pGameL, 1\pGameL);
\path[witem] (0\pGameL, 14\pGameL) rectangle +(15\pGameL, 1\pGameL);
\path[witem] (15\pGameL, 5\pGameL) rectangle +(10\pGameL, 2\pGameL);
\path[witem] (5\pGameL, 10\pGameL) rectangle +(10\pGameL, 2\pGameL);
\path[witem] (15\pGameL, 0\pGameL) rectangle +(10\pGameL, 1\pGameL);
\path[witem] (15\pGameL, 1\pGameL) rectangle +(15\pGameL, 1\pGameL);
\path[witem] (10\pGameL, 2\pGameL) rectangle +(10\pGameL, 2\pGameL);
\path[witem] (20\pGameL, 25\pGameL) rectangle +(10\pGameL, 2\pGameL);

\path[bin] (0\pGameL, 0\pGameL) rectangle (30\pGameL, 30\pGameL);
\path[cutline] foreach \x in {5,10,15,20,25}{
    (\x\pGameL, -1\pGameL) -- (\x\pGameL, 32\pGameL)
};
\foreach \index/\x in {1/2.5,2/7.5,3/12.5,4/17.5,5/22.5,6/27.5}{
    \node[anchor=south] at (\x\pGameL, 30\pGameL) {\index};
}
\end{tikzpicture}

\caption{A packing of items in a bin. Wide items are green and tall items are blue.
Draw vertical lines at $x$-coordinates from $\Tcal - \{0\}$.
They divide the bin into columns. In this figure, we have 6 columns.}
\label{fig:compartmentalize:begin}
\end{subfigure}
\hfill
\begin{subfigure}{0.45\textwidth}
\centering
\ifcsname pGameL\endcsname\else\newlength{\pGameL}\fi
\setlength{\pGameL}{0.2cm}
\definecolor{myBlue}{HTML}{00b0f0}
\definecolor{myGreen}{HTML}{92d050}
\definecolor{myLightBlue}{HTML}{ccf1ff}
\definecolor{myLightGreen}{HTML}{e6f5d6}
\definecolor{shadedLightBlue}{HTML}{b4d9e4}
\definecolor{shadedLightGreen}{HTML}{ceddc0}

\tikzset{bin/.style={draw,thick}}
\tikzset{binGrid/.style={draw,step=1\pGameL,{black!20}}}
\tikzset{witem/.style={draw,fill=myGreen}}
\tikzset{hitem/.style={draw={black!40},fill=shadedLightBlue}}
\tikzset{tcell/.style={fill={black!10}}}
\tikzset{tcell2/.style={draw,semithick}}
\tikzset{cutline/.style={draw,dashed,thick}}
\begin{tikzpicture}
%\path[binGrid] (0\pGameL, 0\pGameL) grid (30\pGameL, 30\pGameL);
\path[tcell] (0\pGameL, 19\pGameL) rectangle +(5\pGameL, 11\pGameL);
\path[tcell] (5\pGameL, 19\pGameL) rectangle +(5\pGameL, 11\pGameL);
\path[tcell] (10\pGameL, 17\pGameL) rectangle +(5\pGameL, 13\pGameL);
\path[tcell] (0\pGameL, 0\pGameL) rectangle +(5\pGameL, 12\pGameL);
\path[tcell] (5\pGameL, 0\pGameL) rectangle +(5\pGameL, 10\pGameL);
\path[tcell] (10\pGameL, 4\pGameL) rectangle +(5\pGameL, 6\pGameL);
\path[tcell] (15\pGameL, 7\pGameL) rectangle +(5\pGameL, 23\pGameL);
\path[tcell] (20\pGameL, 7\pGameL) rectangle +(5\pGameL, 18\pGameL);
\path[tcell] (25\pGameL, 2\pGameL) rectangle +(5\pGameL, 23\pGameL);
\path[hitem] (0\pGameL, 22\pGameL) rectangle +(1\pGameL, 8\pGameL);
\path[hitem] (2\pGameL, 20\pGameL) rectangle +(2\pGameL, 8\pGameL);
\path[hitem] (7\pGameL, 21\pGameL) rectangle +(1\pGameL, 9\pGameL);
\path[hitem] (8\pGameL, 20\pGameL) rectangle +(1\pGameL, 9\pGameL);
\path[hitem] (10\pGameL, 22\pGameL) rectangle +(2\pGameL, 8\pGameL);
\path[hitem] (12\pGameL, 17\pGameL) rectangle +(2\pGameL, 8\pGameL);
\path[hitem] (14\pGameL, 20\pGameL) rectangle +(2\pGameL, 10\pGameL);
\path[hitem] (17\pGameL, 19\pGameL) rectangle +(2\pGameL, 11\pGameL);
\path[hitem] (0\pGameL, 1\pGameL) rectangle +(1\pGameL, 8\pGameL);
\path[hitem] (2\pGameL, 0\pGameL) rectangle +(1\pGameL, 12\pGameL);
\path[hitem] (4\pGameL, 1\pGameL) rectangle +(2\pGameL, 8\pGameL);
\path[hitem] (7\pGameL, 0\pGameL) rectangle +(2\pGameL, 9\pGameL);
\path[hitem] (11\pGameL, 4\pGameL) rectangle +(1\pGameL, 6\pGameL);
\path[hitem] (26\pGameL, 10\pGameL) rectangle +(2\pGameL, 9\pGameL);
\path[hitem] (16\pGameL, 12\pGameL) rectangle +(1\pGameL, 12\pGameL);
\path[hitem] (17\pGameL, 7\pGameL) rectangle +(1\pGameL, 11\pGameL);
\path[hitem] (18\pGameL, 7\pGameL) rectangle +(2\pGameL, 7\pGameL);
\path[hitem] (21\pGameL, 9\pGameL) rectangle +(2\pGameL, 13\pGameL);
\path[hitem] (25\pGameL, 8\pGameL) rectangle +(1\pGameL, 16\pGameL);
\path[hitem] (29\pGameL, 2\pGameL) rectangle +(1\pGameL, 8\pGameL);
\path[hitem] (27\pGameL, 2\pGameL) rectangle +(2\pGameL, 8\pGameL);
\path[hitem] (28\pGameL, 10\pGameL) rectangle +(1\pGameL, 9\pGameL);

\path[witem] (0\pGameL, 12\pGameL) rectangle +(10\pGameL, 2\pGameL);
\path[witem] (0\pGameL, 17\pGameL) rectangle +(10\pGameL, 2\pGameL);
\path[witem] (5\pGameL, 16\pGameL) rectangle +(10\pGameL, 1\pGameL);
\path[witem] (0\pGameL, 15\pGameL) rectangle +(10\pGameL, 1\pGameL);
\path[witem] (0\pGameL, 14\pGameL) rectangle +(15\pGameL, 1\pGameL);
\path[witem] (15\pGameL, 5\pGameL) rectangle +(10\pGameL, 2\pGameL);
\path[witem] (5\pGameL, 10\pGameL) rectangle +(10\pGameL, 2\pGameL);
\path[witem] (15\pGameL, 0\pGameL) rectangle +(10\pGameL, 1\pGameL);
\path[witem] (15\pGameL, 1\pGameL) rectangle +(15\pGameL, 1\pGameL);
\path[witem] (10\pGameL, 2\pGameL) rectangle +(10\pGameL, 2\pGameL);
\path[witem] (20\pGameL, 25\pGameL) rectangle +(10\pGameL, 2\pGameL);

\path[tcell2] (0\pGameL, 19\pGameL) rectangle +(5\pGameL, 11\pGameL) node[pos=0.5] {1};
\path[tcell2] (5\pGameL, 19\pGameL) rectangle +(5\pGameL, 11\pGameL) node[pos=0.5] {2};
\path[tcell2] (10\pGameL, 17\pGameL) rectangle +(5\pGameL, 13\pGameL) node[pos=0.5] {3};
\path[tcell2] (0\pGameL, 0\pGameL) rectangle +(5\pGameL, 12\pGameL) node[pos=0.5] {4};
\path[tcell2] (5\pGameL, 0\pGameL) rectangle +(5\pGameL, 10\pGameL) node[pos=0.5] {5};
\path[tcell2] (10\pGameL, 4\pGameL) rectangle +(5\pGameL, 6\pGameL) node[pos=0.5] {6};
\path[tcell2] (15\pGameL, 7\pGameL) rectangle +(5\pGameL, 23\pGameL) node[pos=0.5] {7};
\path[tcell2] (20\pGameL, 7\pGameL) rectangle +(5\pGameL, 18\pGameL) node[pos=0.5] {8};
\path[tcell2] (25\pGameL, 2\pGameL) rectangle +(5\pGameL, 23\pGameL) node[pos=0.5] {9};
\path[bin] (0\pGameL, 0\pGameL) rectangle (30\pGameL, 30\pGameL);
\end{tikzpicture}

\caption{Wide items divide each column into \emph{cells}.
Each cell containing a tall item is called a \emph{tall cell}.
There are 9 tall cells in this figure, which are shaded gray.}
\label{fig:compartmentalize:tall-cells}
\end{subfigure}
\caption{Creating tall cells in a bin}
\label{fig:compartmentalize-1}
\end{figure}

By \cref{thm:empty-to-rects}, we can use horizontal cuts to partition the space outside
tall cells into at most $3(1/\epsLarge-1)|\Tcal| + 1$ rectangular regions
(this can slice some wide items). See \cref{fig:compartmentalize:boxes}.
If a region contains a wide item, call it a box.

\begin{figure}[!htb]
\begin{subfigure}[t]{0.45\textwidth}
\centering
\ifcsname pGameL\endcsname\else\newlength{\pGameL}\fi
\setlength{\pGameL}{0.2cm}
\definecolor{myBlue}{HTML}{00b0f0}
\definecolor{myGreen}{HTML}{92d050}
\definecolor{myLightBlue}{HTML}{ccf1ff}
\definecolor{myLightGreen}{HTML}{e6f5d6}
\definecolor{shadedLightBlue}{HTML}{b4d9e4}
\definecolor{shadedLightGreen}{HTML}{ceddc0}

\tikzset{bin/.style={draw,thick}}
\tikzset{binGrid/.style={draw,step=1\pGameL,{textColor!20!bgColor}}}
\tikzset{witem/.style={draw={textColor!40!bgColor},fill=shadedLightGreen}}
\tikzset{hitem/.style={draw={textColor!30!bgColor},fill=myLightBlue}}
\tikzset{tcell/.style={draw}}
\tikzset{box/.style={fill={textColor!10!bgColor}}}
\tikzset{box2/.style={draw,thick}}
\begin{tikzpicture}
%\path[binGrid] (0\pGameL, 0\pGameL) grid (30\pGameL, 30\pGameL);

\path[box] (0\pGameL, 17\pGameL) rectangle +(10\pGameL, 2\pGameL);
\path[box] (0\pGameL, 12\pGameL) rectangle +(15\pGameL, 5\pGameL);
\path[box] (5\pGameL, 10\pGameL) rectangle +(10\pGameL, 2\pGameL);
\path[box] (20\pGameL, 25\pGameL) rectangle +(10\pGameL, 5\pGameL);
\path[box] (15\pGameL, 4\pGameL) rectangle +(10\pGameL, 3\pGameL);
\path[box] (10\pGameL, 2\pGameL) rectangle +(15\pGameL, 2\pGameL);
\path[box] (10\pGameL, 0\pGameL) rectangle +(20\pGameL, 2\pGameL);

\path[witem] (0\pGameL, 12\pGameL) rectangle +(10\pGameL, 2\pGameL);
\path[witem] (0\pGameL, 17\pGameL) rectangle +(10\pGameL, 2\pGameL);
\path[witem] (5\pGameL, 16\pGameL) rectangle +(10\pGameL, 1\pGameL);
\path[witem] (0\pGameL, 15\pGameL) rectangle +(10\pGameL, 1\pGameL);
\path[witem] (0\pGameL, 14\pGameL) rectangle +(15\pGameL, 1\pGameL);
\path[witem] (15\pGameL, 5\pGameL) rectangle +(10\pGameL, 2\pGameL);
\path[witem] (5\pGameL, 10\pGameL) rectangle +(10\pGameL, 2\pGameL);
\path[witem] (15\pGameL, 0\pGameL) rectangle +(10\pGameL, 1\pGameL);
\path[witem] (15\pGameL, 1\pGameL) rectangle +(15\pGameL, 1\pGameL);
\path[witem] (10\pGameL, 2\pGameL) rectangle +(10\pGameL, 2\pGameL);
\path[witem] (20\pGameL, 25\pGameL) rectangle +(10\pGameL, 2\pGameL);

\path[hitem] (0\pGameL, 22\pGameL) rectangle +(1\pGameL, 8\pGameL);
\path[hitem] (2\pGameL, 20\pGameL) rectangle +(2\pGameL, 8\pGameL);
\path[hitem] (7\pGameL, 21\pGameL) rectangle +(1\pGameL, 9\pGameL);
\path[hitem] (8\pGameL, 20\pGameL) rectangle +(1\pGameL, 9\pGameL);
\path[hitem] (10\pGameL, 22\pGameL) rectangle +(2\pGameL, 8\pGameL);
\path[hitem] (12\pGameL, 17\pGameL) rectangle +(2\pGameL, 8\pGameL);
\path[hitem] (14\pGameL, 20\pGameL) rectangle +(2\pGameL, 10\pGameL);
\path[hitem] (17\pGameL, 19\pGameL) rectangle +(2\pGameL, 11\pGameL);
\path[hitem] (0\pGameL, 1\pGameL) rectangle +(1\pGameL, 8\pGameL);
\path[hitem] (2\pGameL, 0\pGameL) rectangle +(1\pGameL, 12\pGameL);
\path[hitem] (4\pGameL, 1\pGameL) rectangle +(2\pGameL, 8\pGameL);
\path[hitem] (7\pGameL, 0\pGameL) rectangle +(2\pGameL, 9\pGameL);
\path[hitem] (11\pGameL, 4\pGameL) rectangle +(1\pGameL, 6\pGameL);
\path[hitem] (26\pGameL, 10\pGameL) rectangle +(2\pGameL, 9\pGameL);
\path[hitem] (16\pGameL, 12\pGameL) rectangle +(1\pGameL, 12\pGameL);
\path[hitem] (17\pGameL, 7\pGameL) rectangle +(1\pGameL, 11\pGameL);
\path[hitem] (18\pGameL, 7\pGameL) rectangle +(2\pGameL, 7\pGameL);
\path[hitem] (21\pGameL, 9\pGameL) rectangle +(2\pGameL, 13\pGameL);
\path[hitem] (25\pGameL, 8\pGameL) rectangle +(1\pGameL, 16\pGameL);
\path[hitem] (29\pGameL, 2\pGameL) rectangle +(1\pGameL, 8\pGameL);
\path[hitem] (27\pGameL, 2\pGameL) rectangle +(2\pGameL, 8\pGameL);
\path[hitem] (28\pGameL, 10\pGameL) rectangle +(1\pGameL, 9\pGameL);


\path[tcell] (0\pGameL, 19\pGameL) rectangle +(5\pGameL, 11\pGameL);
\path[tcell] (5\pGameL, 19\pGameL) rectangle +(5\pGameL, 11\pGameL);
\path[tcell] (10\pGameL, 17\pGameL) rectangle +(5\pGameL, 13\pGameL);
\path[tcell] (0\pGameL, 0\pGameL) rectangle +(5\pGameL, 12\pGameL);
\path[tcell] (5\pGameL, 0\pGameL) rectangle +(5\pGameL, 10\pGameL);
\path[tcell] (10\pGameL, 4\pGameL) rectangle +(5\pGameL, 6\pGameL);
\path[tcell] (15\pGameL, 7\pGameL) rectangle +(5\pGameL, 23\pGameL);
\path[tcell] (20\pGameL, 7\pGameL) rectangle +(5\pGameL, 18\pGameL);
\path[tcell] (25\pGameL, 2\pGameL) rectangle +(5\pGameL, 23\pGameL);

\path[box2] (0\pGameL, 17\pGameL) rectangle +(10\pGameL, 2\pGameL) node[pos=0.5] {1};
\path[box2] (0\pGameL, 12\pGameL) rectangle +(15\pGameL, 5\pGameL) node[pos=0.5] {2};
\path[box2] (5\pGameL, 10\pGameL) rectangle +(10\pGameL, 2\pGameL) node[pos=0.5] {3};
\path[box2] (20\pGameL, 25\pGameL) rectangle +(10\pGameL, 5\pGameL) node[pos=0.5] {4};
\path[box2] (15\pGameL, 4\pGameL) rectangle +(10\pGameL, 3\pGameL) node[pos=0.5] {5};
\path[box2] (10\pGameL, 2\pGameL) rectangle +(15\pGameL, 2\pGameL) node[pos=0.5] {6};
\path[box2] (10\pGameL, 0\pGameL) rectangle +(20\pGameL, 2\pGameL) node[pos=0.5] {7};

\path[bin] (0\pGameL, 0\pGameL) rectangle (30\pGameL, 30\pGameL);
\end{tikzpicture}

\caption{Partition the space outside tall cells into rectangular regions by
extending the horizontal edges of tall cells (see \cref{thm:empty-to-rects}).
Each rectangular region containing a wide item is called a \emph{box}.
There are 7 boxes in this figure, which are shaded gray.}
\label{fig:compartmentalize:boxes}
\end{subfigure}
\hfill
\begin{subfigure}[t]{0.45\textwidth}
\centering
\ifcsname pGameL\endcsname\else\newlength{\pGameL}\fi
\setlength{\pGameL}{0.2cm}
\definecolor{myBlue}{HTML}{00b0f0}
\definecolor{myGreen}{HTML}{92d050}
\definecolor{myLightBlue}{HTML}{ccf1ff}
\definecolor{myLightGreen}{HTML}{e6f5d6}
\definecolor{shadedLightBlue}{HTML}{b4d9e4}
\definecolor{shadedLightGreen}{HTML}{ceddc0}

\tikzset{bin/.style={draw,thick}}
\tikzset{binGrid/.style={draw,step=1\pGameL,{black!20}}}
\tikzset{witem/.style={draw={black!40},fill=shadedLightGreen}}
\tikzset{hitem/.style={draw={black!30},fill=myLightBlue}}
\tikzset{tcell/.style={draw}}
\tikzset{box/.style={fill={black!10}}}
\tikzset{box2/.style={draw,semithick}}
\begin{tikzpicture}
%\path[binGrid] (0\pGameL, 0\pGameL) grid (30\pGameL, 30\pGameL);

\path[box] (0\pGameL, 12\pGameL) rectangle +(15\pGameL, 4\pGameL);
\path[box] (5\pGameL, 10\pGameL) rectangle +(10\pGameL, 2\pGameL);
\path[box] (20\pGameL, 26\pGameL) rectangle +(10\pGameL, 4\pGameL);
\path[box] (15\pGameL, 4\pGameL) rectangle +(10\pGameL, 2\pGameL);
\path[box] (10\pGameL, 2\pGameL) rectangle +(15\pGameL, 2\pGameL);
\path[box] (10\pGameL, 0\pGameL) rectangle +(20\pGameL, 2\pGameL);

\path[witem] (0\pGameL, 12\pGameL) rectangle +(10\pGameL, 2\pGameL);
\path[witem] (0\pGameL, 15\pGameL) rectangle +(10\pGameL, 1\pGameL);
\path[witem] (0\pGameL, 14\pGameL) rectangle +(15\pGameL, 1\pGameL);
\path[witem] (15\pGameL, 5\pGameL) rectangle +(10\pGameL, 1\pGameL);
\path[witem] (5\pGameL, 10\pGameL) rectangle +(10\pGameL, 2\pGameL);
\path[witem] (15\pGameL, 0\pGameL) rectangle +(10\pGameL, 1\pGameL);
\path[witem] (15\pGameL, 1\pGameL) rectangle +(15\pGameL, 1\pGameL);
\path[witem] (10\pGameL, 2\pGameL) rectangle +(10\pGameL, 2\pGameL);
\path[witem] (20\pGameL, 26\pGameL) rectangle +(10\pGameL, 1\pGameL);

\path[hitem] (0\pGameL, 22\pGameL) rectangle +(1\pGameL, 8\pGameL);
\path[hitem] (2\pGameL, 20\pGameL) rectangle +(2\pGameL, 8\pGameL);
\path[hitem] (7\pGameL, 21\pGameL) rectangle +(1\pGameL, 9\pGameL);
\path[hitem] (8\pGameL, 20\pGameL) rectangle +(1\pGameL, 9\pGameL);
\path[hitem] (10\pGameL, 22\pGameL) rectangle +(2\pGameL, 8\pGameL);
\path[hitem] (12\pGameL, 17\pGameL) rectangle +(2\pGameL, 8\pGameL);
\path[hitem] (14\pGameL, 20\pGameL) rectangle +(2\pGameL, 10\pGameL);
\path[hitem] (17\pGameL, 19\pGameL) rectangle +(2\pGameL, 11\pGameL);
\path[hitem] (0\pGameL, 1\pGameL) rectangle +(1\pGameL, 8\pGameL);
\path[hitem] (2\pGameL, 0\pGameL) rectangle +(1\pGameL, 12\pGameL);
\path[hitem] (4\pGameL, 1\pGameL) rectangle +(2\pGameL, 8\pGameL);
\path[hitem] (7\pGameL, 0\pGameL) rectangle +(2\pGameL, 9\pGameL);
\path[hitem] (11\pGameL, 4\pGameL) rectangle +(1\pGameL, 6\pGameL);
\path[hitem] (26\pGameL, 10\pGameL) rectangle +(2\pGameL, 9\pGameL);
\path[hitem] (16\pGameL, 12\pGameL) rectangle +(1\pGameL, 12\pGameL);
\path[hitem] (17\pGameL, 7\pGameL) rectangle +(1\pGameL, 11\pGameL);
\path[hitem] (18\pGameL, 7\pGameL) rectangle +(2\pGameL, 7\pGameL);
\path[hitem] (21\pGameL, 9\pGameL) rectangle +(2\pGameL, 13\pGameL);
\path[hitem] (25\pGameL, 8\pGameL) rectangle +(1\pGameL, 16\pGameL);
\path[hitem] (29\pGameL, 2\pGameL) rectangle +(1\pGameL, 8\pGameL);
\path[hitem] (27\pGameL, 2\pGameL) rectangle +(2\pGameL, 8\pGameL);
\path[hitem] (28\pGameL, 10\pGameL) rectangle +(1\pGameL, 9\pGameL);


\path[tcell] (0\pGameL, 19\pGameL) rectangle +(5\pGameL, 11\pGameL);
\path[tcell] (5\pGameL, 19\pGameL) rectangle +(5\pGameL, 11\pGameL);
\path[tcell] (10\pGameL, 17\pGameL) rectangle +(5\pGameL, 13\pGameL);
\path[tcell] (0\pGameL, 0\pGameL) rectangle +(5\pGameL, 12\pGameL);
\path[tcell] (5\pGameL, 0\pGameL) rectangle +(5\pGameL, 10\pGameL);
\path[tcell] (10\pGameL, 4\pGameL) rectangle +(5\pGameL, 6\pGameL);
\path[tcell] (15\pGameL, 7\pGameL) rectangle +(5\pGameL, 23\pGameL);
\path[tcell] (20\pGameL, 7\pGameL) rectangle +(5\pGameL, 18\pGameL);
\path[tcell] (25\pGameL, 2\pGameL) rectangle +(5\pGameL, 23\pGameL);

\path[box2] (0\pGameL, 12\pGameL) rectangle +(15\pGameL, 4\pGameL) node[pos=0.5] {2};
\path[box2] (5\pGameL, 10\pGameL) rectangle +(10\pGameL, 2\pGameL) node[pos=0.5] {3};
\path[box2] (20\pGameL, 26\pGameL) rectangle +(10\pGameL, 4\pGameL) node[pos=0.5] {4};
\path[box2] (15\pGameL, 4\pGameL) rectangle +(10\pGameL, 2\pGameL) node[pos=0.5] {5};
\path[box2] (10\pGameL, 2\pGameL) rectangle +(15\pGameL, 2\pGameL) node[pos=0.5] {6};
\path[box2] (10\pGameL, 0\pGameL) rectangle +(20\pGameL, 2\pGameL) node[pos=0.5] {7};

\path[bin] (0\pGameL, 0\pGameL) rectangle (30\pGameL, 30\pGameL);
\end{tikzpicture}

\caption{For each box, discard some items and shift horizontal edges
to make their $y$-coordinates multiples of $\epsCont$.
Boxes that continue to contain a wide item are now wide compartments.}
\label{fig:compartmentalize:boxes2}
\end{subfigure}
\caption{Obtaining wide compartments}
\label{fig:compartmentalize-2}
\end{figure}

For each box $i$, slice and discard some items from the bottom of the box
and increase $y_1(i)$ so that it becomes a multiple of $\epsCont$.
Then slice and discard some items from the top of the box
and reduce $y_2(i)$ so that it becomes a multiple of $\epsCont$.
The total area of items discarded is less than $2\epsCont$.
If $i$ continues to contain a wide item, it becomes a wide compartment.
Now all wide items belong to some wide compartment
(see \cref{fig:compartmentalize:boxes2}).

Each column has 0 or more wide compartments crossing it.
These wide compartments divide the column into rectangular regions.
Each region that contains a tall item is a tall compartment
(see \cref{fig:compartmentalize-3}).

\begin{figure}[!htb]
\centering
\ifcsname pGameL\endcsname\else\newlength{\pGameL}\fi
\setlength{\pGameL}{0.2cm}
\definecolor{myBlue}{HTML}{00b0f0}
\definecolor{myGreen}{HTML}{92d050}
\definecolor{myLightBlue}{HTML}{ccf1ff}
\definecolor{myLightGreen}{HTML}{e6f5d6}
\definecolor{shadedLightBlue}{HTML}{b4d9e4}
\definecolor{shadedLightGreen}{HTML}{ceddc0}

\tikzset{bin/.style={draw,thick}}
\tikzset{binGrid/.style={draw,step=1\pGameL,{textColor!20!bgColor}}}
\tikzset{witem/.style={draw={textColor!30!bgColor},fill=myLightGreen}}
\tikzset{hitem/.style={draw={textColor!30!bgColor},fill=myLightBlue}}
\tikzset{tcell/.style={draw,semithick}}
\tikzset{box2/.style={draw}}
\begin{tikzpicture}
%\path[binGrid] (0\pGameL, 0\pGameL) grid (30\pGameL, 30\pGameL);

\path[witem] (0\pGameL, 12\pGameL) rectangle +(10\pGameL, 2\pGameL);
\path[witem] (0\pGameL, 15\pGameL) rectangle +(10\pGameL, 1\pGameL);
\path[witem] (0\pGameL, 14\pGameL) rectangle +(15\pGameL, 1\pGameL);
\path[witem] (15\pGameL, 5\pGameL) rectangle +(10\pGameL, 1\pGameL);
\path[witem] (5\pGameL, 10\pGameL) rectangle +(10\pGameL, 2\pGameL);
\path[witem] (15\pGameL, 0\pGameL) rectangle +(10\pGameL, 1\pGameL);
\path[witem] (15\pGameL, 1\pGameL) rectangle +(15\pGameL, 1\pGameL);
\path[witem] (10\pGameL, 2\pGameL) rectangle +(10\pGameL, 2\pGameL);
\path[witem] (20\pGameL, 26\pGameL) rectangle +(10\pGameL, 1\pGameL);

\path[hitem] (0\pGameL, 22\pGameL) rectangle +(1\pGameL, 8\pGameL);
\path[hitem] (2\pGameL, 20\pGameL) rectangle +(2\pGameL, 8\pGameL);
\path[hitem] (7\pGameL, 21\pGameL) rectangle +(1\pGameL, 9\pGameL);
\path[hitem] (8\pGameL, 20\pGameL) rectangle +(1\pGameL, 9\pGameL);
\path[hitem] (10\pGameL, 22\pGameL) rectangle +(2\pGameL, 8\pGameL);
\path[hitem] (12\pGameL, 17\pGameL) rectangle +(2\pGameL, 8\pGameL);
\path[hitem] (14\pGameL, 20\pGameL) rectangle +(2\pGameL, 10\pGameL);
\path[hitem] (17\pGameL, 19\pGameL) rectangle +(2\pGameL, 11\pGameL);
\path[hitem] (0\pGameL, 1\pGameL) rectangle +(1\pGameL, 8\pGameL);
\path[hitem] (2\pGameL, 0\pGameL) rectangle +(1\pGameL, 12\pGameL);
\path[hitem] (4\pGameL, 1\pGameL) rectangle +(2\pGameL, 8\pGameL);
\path[hitem] (7\pGameL, 0\pGameL) rectangle +(2\pGameL, 9\pGameL);
\path[hitem] (11\pGameL, 4\pGameL) rectangle +(1\pGameL, 6\pGameL);
\path[hitem] (26\pGameL, 10\pGameL) rectangle +(2\pGameL, 9\pGameL);
\path[hitem] (16\pGameL, 12\pGameL) rectangle +(1\pGameL, 12\pGameL);
\path[hitem] (17\pGameL, 7\pGameL) rectangle +(1\pGameL, 11\pGameL);
\path[hitem] (18\pGameL, 7\pGameL) rectangle +(2\pGameL, 7\pGameL);
\path[hitem] (21\pGameL, 9\pGameL) rectangle +(2\pGameL, 13\pGameL);
\path[hitem] (25\pGameL, 8\pGameL) rectangle +(1\pGameL, 16\pGameL);
\path[hitem] (29\pGameL, 2\pGameL) rectangle +(1\pGameL, 8\pGameL);
\path[hitem] (27\pGameL, 2\pGameL) rectangle +(2\pGameL, 8\pGameL);
\path[hitem] (28\pGameL, 10\pGameL) rectangle +(1\pGameL, 9\pGameL);


\path[tcell] (0\pGameL, 16\pGameL) rectangle +(5\pGameL, 14\pGameL) node[pos=0.5] {1};
\path[tcell] (5\pGameL, 16\pGameL) rectangle +(5\pGameL, 14\pGameL) node[pos=0.5] {2};
\path[tcell] (10\pGameL, 16\pGameL) rectangle +(5\pGameL, 14\pGameL) node[pos=0.5] {3};
\path[tcell] (0\pGameL, 0\pGameL) rectangle +(5\pGameL, 12\pGameL) node[pos=0.5] {4};
\path[tcell] (5\pGameL, 0\pGameL) rectangle +(5\pGameL, 10\pGameL) node[pos=0.5] {5};
\path[tcell] (10\pGameL, 4\pGameL) rectangle +(5\pGameL, 6\pGameL) node[pos=0.5] {6};
\path[tcell] (15\pGameL, 6\pGameL) rectangle +(5\pGameL, 24\pGameL) node[pos=0.5] {7};
\path[tcell] (20\pGameL, 6\pGameL) rectangle +(5\pGameL, 20\pGameL) node[pos=0.5] {8};
\path[tcell] (25\pGameL, 2\pGameL) rectangle +(5\pGameL, 24\pGameL) node[pos=0.5] {9};

\path[box2] (0\pGameL, 12\pGameL) rectangle +(15\pGameL, 4\pGameL);
\path[box2] (5\pGameL, 10\pGameL) rectangle +(10\pGameL, 2\pGameL);
\path[box2] (20\pGameL, 26\pGameL) rectangle +(10\pGameL, 4\pGameL);
\path[box2] (15\pGameL, 4\pGameL) rectangle +(10\pGameL, 2\pGameL);
\path[box2] (10\pGameL, 2\pGameL) rectangle +(15\pGameL, 2\pGameL);
\path[box2] (10\pGameL, 0\pGameL) rectangle +(20\pGameL, 2\pGameL);

\path[bin] (0\pGameL, 0\pGameL) rectangle (30\pGameL, 30\pGameL);
\end{tikzpicture}

\caption[Obtaining tall compartments]%
{Wide compartments divide each column into rectangular regions.
Each such region containing a tall item is a tall compartment.
There are 9 tall compartments in this figure.}
\label{fig:compartmentalize-3}
\end{figure}

Therefore, by removing wide and small items of area less than
$6|\Tcal|\epsCont/\epsLarge \le \eps$, we get a compartmental packing of items
where there are at most $(1/\epsLarge-1)|\Tcal|$ tall compartments
and at most $3(1/\epsLarge-1)|\Tcal| + 1$ wide compartments.
\end{proof}

\subsection{Existence of Near-Optimal Compartmental Packing}

For a set $\Itild$ of rounded items, define $\fcopt(\Itild)$ as the number of bins in
the optimal fractional compartmental packing of $\Itild$.

\begin{theorem}
\label{thm:struct}
Let $\Itild$ be a set of $\delta$-\thin{} rounded items. Then
$\fcopt(\Itild) < (1+4\eps)\fopt(\Itild) + 2$.
\end{theorem}
\begin{proof}
Consider a fractional packing of $\Itild$ into $m \defeq \fopt(\Itild)$ bins.
By \cref{thm:disc-hor-pos,thm:thin-bp:compartmentalize}, in each bin,
we can discard items of area at most $2\eps$ from the bin
and get a compartmental packing of the remaining items.

Let $X$ be the set of wide and small discarded items
and let $Y$ be the set of tall discarded items.
For each item $i \in X$, if $w(i) \le 1/2$, slice it using a horizontal cut in the middle
and place the pieces horizontally next to each other to get a new item
of width $2w(i)$ and height $h(i)/2$. Repeat until $w(i) > 1/2$.
Now pack the items in bins by stacking them one-over-the-other
so that for each item $i \in X$, $x_1(i) = 0$.
This will require less than $2a(X) + 1$ bins,
and the packing will be compartmental.

Similarly, we can get a compartmental packing of $Y$ into $2a(Y) + 1$ bins.
Since $a(X \cup Y) < 2\eps m$, we will require less than $4\eps m + 2$ bins.
Therefore, the total number of compartmental bins used to pack $\Itild$
is less than $(1 + 4\eps)m + 2$.
\end{proof}
